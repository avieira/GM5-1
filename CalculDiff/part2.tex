\part{Équations aux dérivées partielles d'ordre 1}
\section{Équations homogènes}
\begin{equation} \label{EH}\tag{EH} \sum_{i=1}^n \frac{\partial h}{\partial x_i} f_i(x)=0 \end{equation}
$(f_1,...,f_n)^T$, $f_i=f_i(x)$ : champ de vecteur donné. $h$ cherché : \[h : \begin{array}{c c c} \mathbb{R}^n &\to& \mathbb{R} \\ x &\mapsto& h(x) \end{array}\]

\Def{Problème de Cauchy}{Soit $M$ une hypersurface dans $\mathbb{R}^n$, une sous-variété de $\mathbb{R}^n$ de dimension $n-1$ : \[M=\{x\in\mathbb{R}^n, \phi(x)=0\} \text{ où } \phi :\mathbb{R}^n\to \mathbb{R},\ rg\frac{\partial \phi}{\partial x}=1,\ \forall x\in M\]

Fixons $M$ une hypersurface et $ B:M\to \mathbb{R}$. Le problème de Cauchy est : \\
Trouver une solution de (\ref{EH}) tel que \[\restriction{h}{M}=b\]}

On remarque que $h$ est une intégrale première de l'équation $\dot{x}=f(x)$ \\
De même, en posant $y=\sigma(x)$, avec $\sigma : \mathbb{R}^n\to \mathbb{R}^n$ un difféomorphisme, on remarque que pour la fonction $\tilde{h}$ définit par : \[h(x)=\tilde{h}(y)\] on a $\tilde{h}$ une intégrale première de $\dot{y}=\tilde{f}(y)$ définit par : 
\[\tilde{f}(y(t))=\frac{\partial \sigma}{\partial x} \left(\sigma^{-1}(y(t))\right)f\left(\sigma^{-1}(y(t))\right)\]

\Lem{}{Si $f(x_0)\neq 0$, $\exists y=\sigma(x)$ un difféomorphisme local, tel que : \[\frac{\partial \sigma}{\partial x} \left(\sigma^{-1}(y)\right) f\left(\sigma^{-1}(y) \right)=\tilde{f}(y)=\frac{\partial}{\partial y_1}\]}

\Lem{}{Étant donné $(M,f)$ tels que $f(x_0)\notin T_{x_0}M$, $\exists y=\sigma(x)$ un difféomorphisme local, tel que : \[M=\{y_1=0\} \text{ et } \frac{\partial \sigma}{\partial x} \left(\sigma^{-1}(y)\right) f\left(\sigma^{-1}(y) \right)=\tilde{f}(y)=\frac{\partial}{\partial y_1}\]}

\Theo{}{\begin{enumerate}
    \item Si $f(x_0)\neq 0$ alors (EH) possède des solutions dans un voisinage de $x_0$
    \item Si $f(x_0)\notin T_{x_0}M$ alors dans un voisinage de $x_0$, le problème de Cauchy possède une solution unique.
\end{enumerate}} 

\section{Équations non homogènes}
On s'intéresse maintenant aux équations non homogènes : \begin{equation} \label{ENH}\tag{ENH} \sum_{i=1}^n \frac{\partial h}{\partial x_i} f_i(x)=\eta(x) \end{equation}
\Theo{}{\begin{enumerate}
	\item Si $f(p)\neq 0$, alors il existe localement, autour de $p$, des solutions.
	\item Fixons $M$ une hypersurface et $b:M\to \mathbb{R}$. Si $f(p)\not\in T_p M$, alors autour de $p$, il existe une solution du problème de Cauchy (\ref{ENH})$\oplus$ $\restriction{h}{M}=b$\\
En plus, \[h(\gamma_t(p))=b(p)+\int_0^t \eta(\gamma_{\tau}(p)) d\tau\]
\end{enumerate}}

\section{Équations quasi-linéaires}
\begin{equation} \tag{EQL} \label{EQL} \nabla h . f(x,h)=\eta(x,h) \end{equation}

On va chercher la solution sous une forme implicite, ie \begin{equation} \tag{Impl} \label{Impl} \psi(x,h)=0\end{equation}
Si on veut $h(x_0)=h_0$ :
\Theo{des fonctions implicites}{Si $\frac{\partial \psi}{\partial h}(x_0,h_0)\neq 0$, alors (\ref{Impl}) possède une solution unique $h(x)$ satisfaisant cette dernière, ie $\psi(x,h(x))=0$.}

\[\frac{\partial}{\partial x_i}=\frac{\partial \psi}{\partial x_i} + \frac{\partial \psi}{\partial x_i} \frac{\partial h}{\partial x_i} = 0\]
\[\frac{\partial h}{\partial x_i} h(x,h(x))= -\frac{\frac{\partial \psi}{\partial x_i}}{\frac{\partial \psi}{\partial h}}\]
On réintroduit ça dans (\ref{EQL}) et on obtient :
\begin{equation} \tag{EHIm} \label{EHIm} \sum_{i=1}^n \frac{\partial \psi}{\partial x_i} f_i + \frac{\partial \psi}{\partial h} \eta = 0 \end{equation}
On obtient donc une équation homogène dans l'espace $(x,h)$ de dimension $n+1$ !
