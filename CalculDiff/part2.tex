\part{Équations aux dérivées partielles d'ordre 1}
\section{Équations homogènes}
\begin{equation} \label{EH}\tag{EH} \sum_{i=1}^n \frac{\partial h}{\partial x_i} f_i(x)=0 \end{equation}
$(f_1,...,f_n)^T$, $f_i=f_i(x)$ : champ de vecteur donné. $h$ cherché : \[h : \begin{array}{c c c} \mathbb{R}^n &\to& \mathbb{R} \\ x &\mapsto& h(x) \end{array}\]

\Def{Problème de Cauchy}{Soit $M$ une hypersurface dans $\mathbb{R}^n$, une sous-variété de $\mathbb{R}^n$ de dimension $n-1$ : \[M=\{x\in\mathbb{R}^n, \phi(x)=0\} \text{ où } \phi :\mathbb{R}^n\to \mathbb{R},\ rg\frac{\partial \phi}{\partial x}=1,\ \forall x\in M\]

Fixons $M$ une hypersurface et $ B:M\to \mathbb{R}$. Le problème de Cauchy est : \\
Trouver une solution de (\ref{EH}) tel que \[\restriction{h}{M}=b\]}

On remarque que $h$ est une intégrale première de l'équation $\dot{x}=f(x)$ \\
De même, en posant $y=\sigma(x)$, avec $\sigma : \mathbb{R}^n\to \mathbb{R}^n$ un difféomorphisme, on remarque que pour la fonction $\tilde{h}$ définit par : \[h(x)=\tilde{h}(y)\] on a $\tilde{h}$ une intégrale première de $\dot{y}=\tilde{f}(y)$ définit par : 
\[\tilde{f}(y(t))=\frac{\partial \sigma}{\partial x} \left(\sigma^{-1}(y(t))\right)f\left(\sigma^{-1}(y(t))\right)\]

\Lem{}{Si $f(x_0)\neq 0$, $\exists y=\sigma(x)$ un difféomorphisme local, tel que : \[\frac{\partial \sigma}{\partial x} \left(\sigma^{-1}(y)\right) f\left(\sigma^{-1}(y) \right)=\tilde{f}(y)=\frac{\partial}{\partial y_1}\]}

\Lem{}{Étant donné $(M,f)$ tels que $f(x_0)\notin T_{x_0}M$, $\exists y=\sigma(x)$ un difféomorphisme local, tel que : \[M=\{y_1=0\} \text{ et } \frac{\partial \sigma}{\partial x} \left(\sigma^{-1}(y)\right) f\left(\sigma^{-1}(y) \right)=\tilde{f}(y)=\frac{\partial}{\partial y_1}\]}

\Theo{}{\begin{enumerate}
    \item Si $f(x_0)\neq 0$ alors (EH) possède des solutions dans un voisinage de $x_0$
    \item Si $f(x_0)\notin T_{x_0}M$ alors dans un voisinage de $x_0$, le problème de Cauchy possède une solution unique.
\end{enumerate}} 

\section{Équations non homogènes}
On s'intéresse maintenant aux équations non homogènes : \begin{equation} \label{ENH}\tag{ENH} \sum_{i=1}^n \frac{\partial h}{\partial x_i} f_i(x)=\eta(x) \end{equation}
\Theo{}{\begin{enumerate}
	\item Si $f(p)\neq 0$, alors il existe localement, autour de $p$, des solutions.
	\item Fixons $M$ une hypersurface et $b:M\to \mathbb{R}$. Si $f(p)\not\in T_p M$, alors autour de $p$, il existe une solution du problème de Cauchy (\ref{ENH})$\oplus$ $\restriction{h}{M}=b$\\
En plus, \[h(\gamma_t(p))=b(p)+\int_0^t \eta(\gamma_{\tau}(p)) d\tau\]
\end{enumerate}}

\section{Équations quasi-linéaires}
\begin{equation} \tag{EQL} \label{EQL} \nabla h . f(x,h)=\eta(x,h) \end{equation}

On va chercher la solution sous une forme implicite, ie \begin{equation} \tag{Impl} \label{Impl} \psi(x,h)=0\end{equation}
Si on veut $h(x_0)=h_0$ :
\Theo{des fonctions implicites}{Si $\frac{\partial \psi}{\partial h}(x_0,h_0)\neq 0$, alors (\ref{Impl}) possède une solution unique $h(x)$ satisfaisant cette dernière, ie $\psi(x,h(x))=0$.}

\[\frac{\partial \psi}{\partial x_i}=\frac{\partial \psi}{\partial x_i} + \frac{\partial \psi}{\partial x_i} \frac{\partial h}{\partial x_i} = 0\]
\[\frac{\partial h}{\partial x_i} = -\frac{\frac{\partial \psi}{\partial x_i}}{\frac{\partial \psi}{\partial h}}\]
On réintroduit ça dans (\ref{EQL}) et on obtient :
\begin{equation} \tag{EHIm} \label{EHIm} \sum_{i=1}^n \frac{\partial \psi}{\partial x_i} f_i + \frac{\partial \psi}{\partial h} \eta = 0 \end{equation}
On obtient donc une équation homogène dans l'espace $(x,h)$ de dimension $n+1$ !

\section{Systèmes d'EDP d'ordre 1}
\begin{equation} \label{systEDP} \tag{S} \sum_{i=1}^n \frac{\partial h}{\partial x_i}f_i^j(x)=\eta^j(x),\ 1\leq j\leq k \end{equation}
On réécrit cela : \[L_{f^j}h=\eta^j,\ 1\leq j\leq k\]

On construit la distribution $\mathcal{D}=span\{f^1,...,f^k\}$.
\begin{itemize}
	\item Soit $\mathcal{D}$ est involutive (et dans ce cas là, tout va bien !)
	\item Soit $\mathcal{D}$ ne l'est pas.
\end{itemize}

Dans le deuxième cas, $\exists f^r, f^q$; $[f^r,f^q]\not\in \mathcal{D}$. Dans ce cas là, on pose $f^{k+1}=[f^r,f^q]$ et : 
\[\eta^{k+1}=L_{[f^r,f^q]}h=L_{f^r}\eta^q-L_{f^q}\eta^r\]
On rajoute $L_{f^{k+1}}h=\eta^{k+1}$ dans (\ref{systEDP}) et on construit une nouvelle distribution $\mathcal{D}$. On recommence, $\mathcal{D}$ involutive ou non.\\

\bigskip
On note $\overline{\mathcal{D}}$ l'adhérence involutive de $\mathcal{D}$, ie la plus petite distribution involutive contenant $\mathcal{D}$.\\
Schématiquement, en notant $[A,B]=\{[a,b],a\in A, b\in B\}$
\[\mathcal{D}^0=\mathcal{D}\]
\[\mathcal{D}^{j+1}=\mathcal{D}^j+[\mathcal{D}^j, \mathcal{D}^j] \]
\[\overline{\mathcal{D}}=\bigcup_{j=0}^\infty \mathcal{D}^j\]
On note symboliquement $\infty$, mais on sait que cela sera fini ($n$ champs dans la distribution pour qu'elle soit involutive).

\bigskip
On peut toujours supposer $\mathcal{D}$ involutive. 
\[\forall f^i, f^j,\ [f^j, f^i]=\sum_{q=1}^k \alpha_q^{ji} f^q,\]

\Theo{Fröbenius}{Considérons (\ref{systEDP}) où $\mathcal{D}=span\{f_1,...,f_k\}$ involutive.
\begin{enumerate}
	\item (\ref{systEDP}) possède des solutions si et seulement si les conditions d'intégrabilité sont satisfaites : 
\begin{equation} \tag{CI} L_{f^j}\eta^i - L_{f^i}\eta^j = \sum_{q=1}^k \alpha_q^{ji}\eta^q\ 1\leq j<i\leq k \end{equation}
Le nombre de conditions d'intégrabilité vaut $\binom{k}{2}$
	\item Soit $M$ une sous-variété de $\mathbb{R}^n$ de dimension $n-k$. Le problème de Cauchy (\ref{systEDP})$\oplus$ ($\restriction{h}{M}=b$) possède une solution unique localement autour de $p\in\mathbb{R}^n$ si \[\mathcal{D}(p)\oplus T_pM=\mathbb{R}^n\]
\end{enumerate}}
