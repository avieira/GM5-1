\section{Calcul variationnel}
Ici : recherche d'optimum non plus dans un espace de réels, mais dans un espace de fonctions. On cherche $y^*$ tel que :
	\[I(y^*)=\min_{y\in\mathcal{F}}I(y)\]

Considérons $y:[x_1,x_2]\to \mathbb{R}$ et $L:[x_1,x_2]\times\mathbb{R}\times\mathbb{R}\to\mathbb{R}$. Parmis tous les $y$, dérivable et tel que $y(x_1)=y_1$ et $y(x_2)=y_2$, trouver la courbe minimisant : \begin{equation} \label{P} \tag{P} I(y)=\int_{x_1}^{x_2} L(x,y(x),y'(x)) dx\end{equation}

\Theo{Euler-Lagrange}{Si $y\in\mathcal{C}^1[x_1,x_2]$ minimise $\int_{x_1}^{x_2} L(x,y,y')dx$ parmi toutes les fonctions telles que $y(x_1)=y_1$ et $y(x_2)=y_2$ où $L\in\mathcal{C}^2$, alors $y$ satisfait : \[\frac{\partial L}{\partial y} - \frac{d}{dx}\frac{\partial L}{\partial y'}=0\]}

\textbf{Idée de la démonstration :} On prend $y$ minimisant $I$, et on pose $Y=y+\varepsilon \eta$, avec $\eta(x_1)=\eta(x_2)=0$, puis on reprend $I$ dépendant de $\varepsilon$. $I$ est minimal pour $\varepsilon=0$, on dérive, on trouve ce qu'il faut !

\Def{}{$g$ est une intégrale première de l'équation d'Euler-Lagrange si $g$ est contante le long des solutions de l'équation d'Euler-Lagrange.}

\Prop{}{\begin{enumerate}
	\item Si $L=L(x,y')$, alors $\frac{\partial L}{\partial y'}=C$
	\item Si $L=L(y,y')$ alors $L-y'\frac{\partial L}{\partial y}=C$.
\end{enumerate}}

\Def{Topologie dans $\mathcal{C}([x_1,x_2])$}{On définit une topologie dans $\mathcal{C}^0([x_1,x_2])$ :
\[\forall y\in\mathcal{C}^0; \|y\|_{\mathcal{C}^0} = \max_{x\in[x_1,x_2]}\|y(x)\|_2\]
\[\forall y\in\mathcal{C}^0([x_1,x_2]),\ V_{\varepsilon}^0(y)= \left\{\tilde{y}\in\mathcal{C}^0\left([x_1,x_2]\right); \|y-\tilde{y}\|_{\mathcal{C}^0}<\varepsilon\right\}\]
On fait de même dans $\mathcal{C}^1([x_1,x_2])$ : 
\[\forall y\in\mathcal{C}^1; \|y\|_{\mathcal{C}^1} = \max_{x\in[x_1,x_2]}\|y(x)\|_2+\max_{x\in[x_1,x_2]}\|y'(x)\|_2\]
\[\forall y\in\mathcal{C}^1([x_1,x_2]),\ V_{\varepsilon}^1(y)= \left\{\tilde{y}\in\mathcal{C}^1\left([x_1,x_2]\right); \|y-\tilde{y}\|_{\mathcal{C}^1}<\varepsilon\right\}\]}

\Def{}{On considère que le problème \ref{P} admet comme solution $y^*$.\\
$y^*$ est un minimum fort strict s'il existe un voisinage dans $\mathcal{C}^0([x_1,x_2])$ (ie $V_{\varepsilon}^0(y^*)$) tel que : 
		\[I(y^*)<I(y) \forall y\in V_{\varepsilon}^0 (y^*)\]
C'est un maximum fort strict si : \[I(y^*)>I(y) \forall y\in V_{\varepsilon}^0(y^*)\]

\[\exists V_{\varepsilon}^1(y^*);\ I(y^*)<I(y) \Rightarrow y* \text{ minimum faible strict}\]
\[\exists V_{\varepsilon}^1(y^*);\ I(y^*)>I(y) \Rightarrow y* \text{ maximum faible strict}\]}

\Def{}{Soit $D=[x_1,x_2]\times \mathbb{R}$.\\
$y(x,C), C\in\mathbb{R}$ est un champ d'extrémales, si : 
\begin{enumerate}
	\item $(x,y(x,C))\in D,\ \forall C\in\mathbb{R}$
	\item $\forall C\in\mathbb{R},$ $y(x)$ satisfait les équations d'Euler-Lagrange.
\end{enumerate}

Ce champ est dit propre si $\forall (x_0,y_0)\in D,\ \exists !y(x,C)$ extrémale.\\
Ce champ est dit central si $y(X,C)=y_1,\ \forall C\in\mathbb{R}$ et $y(x,C)\neq y(x,\tilde{C})$, $\forall C\neq \tilde{C}$, $\forall x\neq \tilde{x}$.}

\Theo{Jacobi-Weierstrass}{Supposons $L\in\mathcal{C}^3$. Considérons toujours le même problème de minimisation \ref{P}.\\
Si $y^*$ satisfait :
\begin{enumerate}
	\item $y^*(x_1)=y_1$ et $y^*(x_2)=y_2$
	\item $y^*$ peut être plongé dans un champ d'extrémale soit propre soit central
	\item $\frac{\partial^2 L}{\partial y'^2}(x,y^*,(y^*)')>0$ (resp $<0$)\\
Alors $y^*$ est un minimum (resp. maximum) faible
	\item $\frac{\partial^2 L}{\partial y'^2}(x,y,y')>0$ (resp $<0$) $\forall y\in V_{\varepsilon}^0(y^*)$\\
Alors $y^*$ est un minimum (resp. maximum) fort.
\end{enumerate}}
