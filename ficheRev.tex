\documentclass{article}
\input{../preambule}
%\usepackage[8pt]{extsizes}

\hypersetup{colorlinks=true, urlcolor=bleu, linkcolor=red}

%Def = Definition
%Theo = Théorème
%Prop = Propriété
%Coro = Corollaire
%Lem = Lemme

\makeatletter
\@addtoreset{section}{part}
\makeatother

\begin{document}
\part{Automatique non lin\'{e}aire}
\begin{enumerate}
\section{Outils mathématiques}
	\item Définition $\mathcal{C}^{\infty}$, difféomorphisme, difféomorphisme local.
	\item Théorème sur difféomorphisme local
	\item Définition vecteur tangent, espace tangent, champ de vecteurs
	\item Lemme : $(\phi_*f)(p)=?$
	\item Proposition : diagramme qui commute
	\item Proposition : Lien entre flot original et flot transformé ?
	\item Définition de $V^{\infty}(X)$
	\item Défintion du crochet de Lie, autre formulation
\section{Controlabilité}
	\item Proposition : à quelle condition revient-on au point de départ ?
	\item 7 points équivalents sur la controlabilité (critère de Kalmann)
	\item Définition d'accessible, fortement accessible.
	\item Proposition sur le linéairisé et l'accessibilité locale \item Définition d'une l'algèbre de Lie de $\Pi$
	\item Expression de $\mathcal{L}$ avec les contrôles
	\item Propriété des crochets
	\item Définition de sous-variété plongée de dimension $n$
	\item Lemme sur vecteurs tangents et leur crochets
	\item Théorème de Sussman-Jevdjevic (Dim de $\mathcal{L}(x_0)$)
	\item Définition d'une distribution, de rang constant, $\mathcal{C}^{\infty}$, intégrale, involutive
	\item Définition: Opérateur associé à un champ de vecteur
	\item Théorème de Frobenius
	\item Définition d'Orbite
	\item Lemme : relation d'équivalence pour l'orbite, propriété en résultant
	\item Définition de sous-variété immersée
	\item Théorème sur orbite, sous-variété immersée, distribution, algèbre de Lie de $\Pi$
	\item Définition: Idéal de Lie
	\item Théorème sur l'accessibilité forte et l'idéal de Lie
	\item Propriété : rapport entre $\mathcal{L}$ et $\mathcal{L}_0$
	\item 3 formes normale d'accessibilité
\section{Controlabilité complète}
	\item Définition: Complètement controlable, reversible
	\item Proprité si reversible entre orbite et $R$
	\item Définition: Connexe
	\item Théorème si reversible et connexe avec controlabilité complète
	\item Définition: Stable à la Poisson
	\item Théorème: Bonnard-Crouch sur systèmes affines
\section{Linéarisation}
	\item Définition de S-équivalent, localement S-équivalent, S-linéarisable
	\item Définition de $ad_j^jg$
	\item Théorème : linéarisable autour d'un point d'équilibre
	\item Théorème S-linéarisable
	\item Définition de F-équivalence
	\item Théorème de Jakuleczyk-Respondek sur la F-linéarisation
	\item Définition: Forme de Brunovsky
\section{Observabilité}
	\item Définition indistingable, observable, localement observable, espace d'observation
	\item Définition Codistribution
	\item Théorème de Hermann-Kremer : implication pour localement observable
	\item Théorème : 4 points avec dimension de la codistribution constante
	\item Définition découplable, matrice de découplage
	\item Théorème sur découplage avec matrice de découplage
\end{enumerate}

\part{Calcul différentiel}
\begin{enumerate}
\section{Calcul variationnel}
\subsection{Euler-Lagrange}
	\item Théorème d'Euler-Lagrange
	\item Définition d'une intégrale première
	\item Propriété si $L(x,y')$ et si $L(y,y')$.
	\item Définition: Topologie dans $\mathcal{C}([x_1 , x_2])$
	\item Définition de minimum faible/fort
	\item Définition de champ d'extremales
	\item Théorème: Jacobi-Weierstrass
\subsection{Hamiltonien}
	\item Définition de l'Hamiltonien, système hamiltonien (SH)
	\item Proposition : équivalence à $I$ intégrale première
	\item H intégrale première ?
	\item Définition du crochet de Poisson
	\item Dans 2 cas : I intégrale première de (SH)
	\item Si L invariant par rapport aux translations spatiales ? (avec L puis I)
	\item $I$ invariant par rapport à une transformation
	\item Théorème d'Emmy Noether
\section{EDP d'ordre 1}
	\item Définition : problème de Cauchy
	\item Condition pour que le système (EH) admette des solutions / une solution unique
	\item De même pour (ENH)
	\item Théorème des fonctions implicites
	\item Si EQL, que faire ?
	\item Système d'EDP : définition de $\mathcal{D}$, involutive ? Si oui, comment exprimer tous les crochets ?
	\item Équivalence à (S) possède des solutions, une solution unique pour le problème de Cauchy.
\end{enumerate}

\part{Sobolev}
\begin{enumerate}
\section{Rappels}
	\item Définition: Hölderienne
	\item Théorème: Unicité et existence
	\item Théorème: estimation de Schender
\subsection{$L^p$}
	\item Définition Dual, Bidual, reflexif
	\item Théorème: représentation de Riesz-Fréchet
	\item Inégalité de Holder
	\item Corollaire : convergence entre $L^p$ et $L^{p'}$
	\item $1 \leq p<q\leq +\infty$, inclusion, inégalité des normes
	\item Théorème: inégalité d’interpolation
	\item Lemme de Fatou, convergence dominée de Lebesgue
	\item Convergence dans $L^p$ et dans $O$
	\item Définition: Séparable, $L^p$ ?
	\item Théorème: représentation de Green (un $p$ à remplacer par $p'$)
	\item 3 propriétés de la convolution
	\item Définition : suite régularisante, convergence avec convolution
	\item Densité de $L^p$
	\item Lemme: de Urysohn, corollaire
	\item Théorème : prolongement d'une fonction $L^p$ en dehors de O
\subsection{Distributions}
	\item Convergence dans les fonctions tests, définition distribution
	\item Distribution régulière, dérivée d'une distribution
\section{Espaces de Sobolev}
	\item Définition d'un espace de Sobolev, notation pour $p=2$
	\item Équivalene à la norme dans un espace de Sobolev
	\item Banach, Hilbert ?
	\item Séparable, reflexif ?
	\item Restriction à un sous-ouvert, dérivé d'un produit
	\item Lemme : dérivé d'un produit de convolution avec $W^{1,p}$
	\item Densité de $W^{1,p}(\mathbb{R}^N)$
	\item Définition: ouvert à frontière lipschitzienne
	\item Théorème: de prolongement
	\item Définition : $\mathcal{D}(\overline{\Omega})$
	\item Densité de $W^{1,p}(\Omega)$
	\item Théorème: chain rule, Stampacchia
	\item Corollaire sur valeur du gradient sur les lignes de niveau
	\item Theo : si gradient nul sur tout un domaine ?
	\item Théorème : inclusion de $W^{1,p}(\mathbb{R})$, continue ? Hölderienne ?
	\item Théorème: de Rademacher
	\item Théorème: de trace
	\item Définition de $H^{\frac{1}{2}}(\partial\Omega)$.
	\item Banach ? Densité ? Linéaire continue ? 
	\item Théorème : intégration par partie
	\item Définition : $W_0^{1,p}(O)$
	\item Propriété avec le prolongement par $0$
	\item Si $u\in W^{1,p}(\Omega)$ à support compact inclu dans $\Omega$ ?
	\item Inégalité de Poincaré, corollaire avec norme équivalente sur $H^1_0$
	\item Définition de $H^{-1}$. Décomposition dans $L^2$, décomposition de la norme. Conséquence : inclusion de $L^2$
\subsection{Inclusion continue}
	\item Définition inclusion continue
	\item Théorème: Inclusions continues de Sobolev dans $\mathbb{R}^N$, dans $\Omega$
	\item Rapport entre $W^{1,p}(\Omega)$ et lipschitzienne
	\item Inclusions continues de Sobolev dans $O$
\subsection{Inclusion compacte}
	\item Définition : application compacte, inclusion compacte
	\item Théorème: Ascoli-Arzela
	\item $\Omega$ ouvert borné de $\mathbb{R}^N$, inclusion compact de Sobolev
	\item Théorème: de Rellich-Komdrochov ($1\leq p<N$), exposant critique des inclusions
	\item $p=N$, autres inclusions compactes
\section{Pblm variationnel}
	\item Définition $M(\alpha,\beta,\Omega)$, forme bilinéaire bornée, elliptique
	\item Théorème de Lax-Milgram
	\item Propriété : équivalence des solutions entre problème et variationnel
	\item Théorème : solution unique, estimations
\end{enumerate}
\end{document}
