\section*{Introduction}
On s'intéresse aux problèmes de la forme : \begin{equation} \label{pblm} \tag{P}
\left\{ \begin{array}{c}
	Lu=-\sum_{i,j=1}^N a_{ij}\frac{\partial^2 u}{\partial x_i \partial x_j} + \sum_{i=1}^N b_i \frac{\partial u}{\partial x_i} + cu = f \text{ sur } \Omega\subset \mathbb{R}^N \text{ borné ouvert}\\
	u=g \text{ sur } \partial\Omega
\end{array}\right.
\end{equation}

\Def{Hölderienne}{$f$ hölderienne d'exposant $\alpha$ si : \[\exists c>0; \forall x,y, |f(x)-f(y)|\leq c|x-y|^\alpha, 0<\alpha<1\]}

\Theo{Unicité et existence}{Soit $\partial\Omega$ de classe $\mathcal{C}^1$, L uniformément elliptique : \[\exists \alpha>0; \forall x\in\overline{\Omega}, \forall \xi\in\mathbb{R}^N, \sum_{i,j=1}^N a_{ij}(x) \xi_i \xi_j \geq \alpha |\xi|^2\]
On suppose $a_{ij}, b_i, c\in\mathcal{C}^{0,\alpha}(\Omega)$ (continue et hölderienne), $\alpha\in]0,1[, c\geq 0$.\\
$f\in\mathcal{C}^{0,\alpha}(\overline{\Omega}), g\in\mathcal{C}^0(\partial \Omega)$.\\
Alors $\exists ! u$ solution de (\ref{pblm}) tel que $u\in\mathcal{C}^{2,\alpha}(\Omega)\cap\mathcal{C}^0(\overline{\Omega})$. }

\Theo{estimation de Schender}{Si de plus, $\partial \Omega$ de classe $\mathcal{C}^{2,\alpha}$, $g\in\mathcal{C}^{2,\alpha}(\partial \Omega)$, alors $u\in\mathcal{C}^{2,\alpha}(\overline{\Omega})$ et on a : 
\[\|u\|_{\mathcal{C}^{2,\alpha}(\overline{\Omega})}\leq c\left( \|f\|_{\mathcal{C}^{0,\alpha}(\overline{\Omega})} + \|g\|_{\mathcal{C}^{2,\alpha}(\partial\Omega)} \right)\]}

\part{Rappels divers}
\section{Les espaces $L^p$}
\subsection{Rappels d'analyse fonctionnelle}
\Def{Dual}{Soit $X$ un evn. On appelle dual de $X$ l'espace \[X'=\mathcal{L}(X,\mathbb{R})\]
Si $\phi\in X'$ et $x\in X$, on note souvent : \[\phi(x)=\langle \phi, x\rangle_{X'X}\]
appelé crochet de dualité.}

\Def{Bidual}{Soit $X$ un evn. On appelle bidual de $X$ l'espace \[X''=(X')'\] qui est un Banach.}

\textbf{Remarque : } On peut identifier $X$ avec un sous-espace de $X''$ à travers une isométrie, de la manière suiva,te : $\forall x\in X$, on définit : \[f_x : x'\in X' \mapsto \langle x',x\rangle_{X'X}\in\mathbb{R}\]
$f_x$ est dans $X''$ car linéaire, et $|\langle x',x\rangle|\leq \|x\|_X \|x'\|_{X'}$ donc $f_x$ est borné.

On peut montrer que : \[\mathcal{F} : x\in X \mapsto f_x\in X''\] est une isométrie, ie $\|x\|_X=\|f_x\|_{X''},\ \forall x\in X$. Donc on identifie $x$ avec $f_x$ et on écrit $X\subset X''$.\\
Question : a-t-on $X=X''$ ? autrement dit, $\mathcal{F}$ est-elle surjective ? En général, non.

\Def{Reflexif}{Si $\mathcal{F}$ est surjective, on dit que $C$ est reflexif.}

\Theo{représentation de Riesz-Fréchet}{Soit $H$ de Hilbert. \[\forall F\in H', \exists ! \tau(F)\in H; \forall x\in H, \langle F,x\rangle_{H'H}=(\tau(F), x)_H\]
De plus, l'application \begin{eqnarray*} \Phi : H'&\to&H \\ F&\mapsto& \tau(F) \end{eqnarray*} est une isométrie.}

\subsection{Les espaces $L^p$}
Dans la suite, $O$ est un ouvert de $\mathbb{R}^N$, $N\geq 2$\\
$\Omega$ est un ouvert borné de $\mathbb{R}^N$\\
$dx$ la mesure de Lebesgue\\

\Def{}{Soit $1\leq p< +\infty$.
\[L^p(O)=\{f:O\to\mathbb{R} \text{ mesurable }; \int|f|^p dx<\infty\}\]
\[L^\infty(O)=\{f:O\to\mathbb{R} \text{ mesurable }; |f|<\infty \text{ p.p. dans } O\}\]
\[\forall 1\leq p\leq+\infty, L^p_{loc}(O)=\{f\in L^p(\omega), \forall \omega \text{ ouvert borné}, \bar{\omega}\subset O\}\]}

\Prop{}{$L^p(O)$ est de Banach muni de la norme : \[\|f\|_{L^p(O)}=\left| \begin{array}{r c l} 
	\left(\int_O |f|^p dx \right)^{\frac{1}{p}} &\text{ si }& p<\infty\\
	\inf\{C; |f|\leq C \text{ pp}\} &\text{ si }& p=\infty
\end{array}\right.\]}

\Rem{}{Si $p=2$, $L^2(O)$ est un Hilbert par rapport au produit scalaire \[(f,g)_{L^2(O)}=\int_O f(x)g(x)dx\]}

\Prop{inégalité de Holder}{Soit $1\leq p\leq +\infty$. On pose 
\[p'=\left| \begin{array}{c c c}
	\frac{p}{p-1} &\text{ si }& 1<p<+\infty\\
	1 &\text{ si }& p=+\infty\\
	+\infty &\text{ si }& p=1
\end{array}\right.\]
appelé le conjugué.\\
\[\forall f\in L^p(O), \forall g\in L^{p'}(O), \int_O |f(x)g(x)|dx \leq \|f\|_{L^p(O)} \|g\|_{L^{p'}(O)}\]}

\Coro{}{$1\leq p\leq +\infty$, $p'$ son conjugué.\\
Si $f_n\to f$ dans $L^p(O)$ et $g\in L^{p'}(O)$ alors : \[\lim_{n\to +\infty} \int_O f_n g dx = \int_O fg dx\]}

\Coro{}{$1\leq p < q \leq +\infty$, $\Omega$ ouvert borné de $\mathbb{R}^N$. Alors $L^q(\Omega)\subset L^p(\Omega)$ et $\|f\|_{L^p(\Omega)}\leq c \|f\|_{L^q(\Omega)}$ où $c=c(|\Omega|, p, q)$.}

\Lem{inégalité de Young}{Soient $a,b\geq 0$ et $1<p<+\infty$. Alors \[ab\leq \frac{1}{p}a^p + \frac{1}{p'}b^{p'}\] avec $p'$ le conjugué de $p$.}

\Theo{inégalité d'interpolation}{Soit $1\leq p\leq r<+\infty$.\\
Si $f\in L^p(O)\cap L^r(O)$ alors $f\in L^q(O)$, $\forall p\leq q\leq r$.\\
De plus, \[\|f\|_{L^q(O)}\leq \|f\|^{\alpha}_{L^p(O)}\|f\|^{1-\alpha}_{L^r(O)}\]avec $\alpha\in[0,1]$ tel que $\frac{\alpha}{p} + \frac{1-\alpha}{r} = \frac{1}{q}$}

\subsection{2 rappels de mesure}
\Lem{de Fatou}{Soit $\{f_n\}\subset L^1(O)$ positives bornées dans $L^1(O)$. On pose \[f(x)=\liminf_{n\to +\infty} f_n(x) \text{ p.p. dans } O\]
Alors $f\in L^1(O)$ et \[\|f\|_{L^1(O)} \leq \liminf_{n\to +\infty} \|f_n\|_{L^1(O)}\]}

\Theo{convergence dominée de Lebesgue}{$\{f_n\}\subset L^1(O)$ telle que : 
\begin{enumerate}
	\item $f_n \to f$ presque partout dans $O$
	\item $\exists h\in L^1(O)$ telle que $|f_n(x)|\leq h(x)$ presque partout dans $O$, $\forall n\in \mathbb{N}$.
\end{enumerate}
alors $f_n \xrightarrow{L^1(O)} f$.}

\Prop{}{$1\leq p \leq +\infty$ tel que $f_n \xrightarrow{L^p} f$.\\
Alors $\exists \{f_{n_k}\}$ une sous-suite telle que $f_{n_k}\to f$ presque partout dans $O$.}

\subsection{Supportabilité}
\Def{Séparable}{Soit $B$ un espace de Banach.\\
$B$ est dit séparable s'il existse $A\subset B$ avec $A$ au plus dénombrable tel que $\overline{A}=B$.}

\Prop{}{$L^p(O)$ est séparable si $1\leq p < +\infty$.}

\subsection{Caractérisation du dual}
\Theo{représentation de Green}{$1\leq p<+\infty$, $p'$ son conjugué.\\
Si $f\in \left(L^p(O) \right)'$, alors $\exists ! g_f\in L^p(O)$ tel que \[\forall v\in L^{p'}(O), \langle f,v\rangle_{(L^p(O))'L^p(O)} = \int_O g_f(x) v(x) dx\]
De plus, \[\begin{array}{c c c c} \Phi : & \left(L^p(O)\right)' &\to& L^p(O) \\ &f&\mapsto& g_f \end{array}\] est une isométrie.}

\textbf{Remarque : } On peut donc identifier $f$ avec $g_f$.\\
De plus, $\Phi$ est surjective. On identifie donc $(L^p)'$ avec $L^{p'}$ si $1\leq p\leq +\infty$.
\begin{itemize}
	\item $1<p<+\infty$, $(L^p)'=L^{p'}$
	\item $p=1$, $(L^1)'=L^{\infty}$
	\item $p=+\infty$, $L^1 \subset (L^{\infty})'$
\end{itemize}
Ceci implique en particulier que $L^p(O)$ reflexif si $1<p<+\infty$. Mais $L^1$ et $L^{\infty}$ non reflexifs.

\section{Densité dans $L^p$}
\subsection{Notion de support}
\Def{}{$\phi : O \to \mathbb{R}$ continue. \[supp(\phi)=\{x\in O ; \phi(x)\neq 0\}\] (fermé de $O$)}

\Def{}{\[\mathcal{D}(O)=\{v:O\to \mathbb{R}; v\in\mathcal{C}^{\infty}(O) \text{ et } supp(v) \text{ est un compact de } \mathbb{R}^n \text{ contenu dans } O\}\]
\[\mathcal{C}^0_C(O)=\{v:O\to \mathbb{R}; v\in\mathcal{C}^0(O) \text{ et } supp(v) \text{ est un compact de } \mathbb{R}^n \text{ contenu dans } O\}\]}

\Prop{}{$1\leq p\leq +\infty$, $f\in L^p(O)$.\\
On pose \[\mathcal{A}=\{A \text{ ouvert de } O; f=0 \text{ p.p. dans } A\}\]
Alors si $w=\bigcup_{A\in\mathcal{A}} A$, on a $f=0$ p.p. dans $A$.}

\Def{}{On pose alors $supp(f)=O\backslash w$.}

\Def{}{\[L^p_c(O)=\{f\in L^p(O); supp(f) \text{ est un compact de } \mathbb{R}^n \text{ inclu dans } O\}\]}

\subsection{Convolution}
\Def{}{$1\leq p\leq +\infty$, $f\in L^1(\mathbb{R}^N)$, $g\in L^p(\mathbb{R}^n)$. 
On définit le produit de convolution par : \[\forall x\in \mathbb{R}^n, (f*g)(x)=\int_{\mathbb{R}}f(x-y)g(y)dy \text{ p.p.}\]}

\Prop{}{\begin{enumerate}
	\item $f\in L^1(\mathbb{R}^N)$, $g\in L^p(\mathbb{R}^N)$. \\
$f*g$ est bien définie et $f*g\in L^p(\mathbb{R}^N)$, et : \[\|f*g\|_{L^p(\mathbb{R}^n)} \leq \|f\|_{L^1(\mathbb{R}^N)} \|g\|_{L^p(\mathbb{R}^N)}\]
	\item $f,g\in L^1(\mathbb{R}^N)$, $f*g=g*f$
	\item Si $f\in\mathcal{D}(\mathbb{R}^N)$, $g\in L^p(\mathbb{R}^N)$, alors $f*g\in \mathcal{C}^{\infty}(\mathbb{R}^N)$ (mais pas nécessairement à support compact).
		\[\frac{\partial}{\partial x_i} (f*g) = \frac{\partial f}{\partial x_i} * g\]
Si de plus, $g\in L^p_c(\mathbb{R}^N)$, alors $f*g\in\mathcal{D}(\mathbb{R}^N)$ et $supp(f*g)\subset supp(f) + supp(g)$.
\end{enumerate}}

\subsubsection{Suites régularisantes}
\Def{}{$B(0,1)\subset \mathbb{R}^N$. Soit $\rho \in \mathcal{D}(\mathbb{R}^N)$, $\rho\geq 0$, $\|\rho\|_{L^1(\mathbb{R}^N)}=1$, $supp(\rho)\subset \overline{B(0,1)}$.\\
$\forall n\in\mathbb{N}$, on pose $\rho_n(x)=n^N \rho(nx)$, $\forall x\in\mathbb{R}^N$. $\{\rho_n\}_n$ s'appelle une suite régularisante.}

\Theo{}{$1\leq p<+\infty$, $f\in L^p(\mathbb{R}^N)$. $\forall \{\rho_n\}_n$ suite régularisante : \[\underbrace{\rho_n * f}_{\in\mathcal{C}^{\infty}(\mathbb{R}^N)} \to f \text{ dans } L^p(\mathbb{R}^N)\]}

\Theo{}{$\mathcal{D}(\mathbb{R}^N)$ est dense dans $L^p(\mathbb{R}^N)$, $\forall 1\leq p<+\infty$. (Faux pour $L^{\infty}$ !)}

\Lem{de Urysohn}{$O$ ouvert de $\mathbb{R}^N$, $K$ compact de $\mathbb{R}^N$, $K\subset O$.\\
Alors $\exists \psi\in\mathcal{D}(O)$ telle que $\psi\equiv 1$ sur $K$ et $0\leq \psi<1$.}

\Coro{}{$\forall O\subset \mathbb{R}^N$, $\exists \{\psi_n\}\subset \mathcal{D}(O)$ tel que \[\forall n\in\mathbb{N}, 0\leq \psi_n\leq 1, \psi_n \to 1 \text{ p.p. dans } O\]}

\Theo{}{$1\leq p<\infty$.\\
Soit $v\in L^p(\mathbb{R}^N)$. On prolonge $v$ par zéro : \[\tilde{v}=\left\{ \begin{array}{c c c}
v &\text{ dans }& O \\
0 &\text{ sinon}&
\end{array}\right.\]
Donc $\tilde{v}\in L^p(\mathbb{R}^N)$}

\Theo{}{$f\in L^1_{loc}(O)$ tel que \[\int_O f(x)\phi(x) dx = 0\ \forall \phi\in\mathcal{D}(O)\]
alors $f=0$ presque partout dans $O$.}

\section{Distributions}
\Def{Convergence des suites dans $\mathcal{D}(O)$}
{$\{\phi_n\}\subset \mathcal{D}(O)$, $\phi\in\mathcal{D}(O)$\\
$\phi_n\to \phi$ dans $\mathcal{D}(O)$ si : \begin{enumerate}
	\item $\exists K$ compact, $K\subset O$; \[\forall n, supp(\phi_n)\subset K\] \[supp(\phi)\subset K\]
	\item $\forall \alpha=(\alpha_1,...,\alpha_n)\in\mathbb{N}^n$, $\partial^{\alpha} \phi_n \to \partial^{\alpha} \phi$ uniformément dans $K$
\end{enumerate}}

\textbf{Remarque :} $\mathcal{D}(O)$ n'est pas métrisable, cela ne définit pas une topologie mais on peut en définir une telle que la convergence des suites dans cette topologie soit celle-ci.\\

\Def{}{Une application $T:\mathcal{D}(O)\to\mathbb{R}$ est une distribution si : \begin{enumerate}
	\item $T$ linéaire
	\item Si $\phi_n\to\phi$ dans $\mathcal{D}(O)$, alors $T(\phi_n)\to T(\phi)$
\end{enumerate}
L'ensemble des distributions sur $O$ est noté $\mathcal{D}'(O)$. On notera : \[\langle T,\phi\rangle_{\mathcal{D}'(O)\mathcal{D}(O)}=T(\phi)\]}

\textbf{Remarque :} L'application $\Phi : f\in L^1_{loc}(O) \to T_f\in\mathcal{D}'(O)$ est injective et linéaire car si $T_f(\phi)=O \forall \phi\in\mathcal{D}(O)$ alors $f=0$\\
Donc on identifie $f$ et $T_f$ et on écrit : \[L^1_{loc}(O)\subset \mathcal{D}'(O)\]

\Def{Distribution régulière}{$T\in\mathcal{D}'(O)$ est une régulière si : \[\exists f\in L^1_{loc}(O); T=T_f\]}

\textbf{Remarque :} On peut montrer qu'il existe des distributions non régulières.

\Def{Dérivée d'une distribution}{Soit $T\in\mathcal{D}'(O)$. On appelle dérivée de $T$ (au sens des distributions) par rapport à la ième variable et on la note $\frac{\partial T}{\partial x_i}$ la distribution définie par : \[\forall \phi\in\mathcal{D}(O), \langle \frac{\partial T}{\partial x_i} ,\phi\rangle_{\mathcal{D}'(O)\mathcal{D}(O)}=-\langle T, \frac{\partial \phi}{\partial x_i}\rangle_{\mathcal{D}'(O)\mathcal{D}(O)}\]}
