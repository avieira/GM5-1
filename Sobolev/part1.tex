\section*{Introduction}
On s'intéresse aux problèmes de la forme : \begin{equation} \label{pblm} \tag{P}
\left\{ \begin{array}{c}
	Lu=-\sum_{i,j=1}^N a_{ij}\frac{\partial^2 u}{\partial x_i \partial x_j} + \sum_{i=1}^N b_i \frac{\partial u}{\partial x_i} + cu = f \text{ sur } \Omega\subset \mathbb{R}^N \text{ borné ouvert}\\
	u=g \text{ sur } \partial\Omega
\end{array}\right.
\end{equation}

\Def{Hölderienne}{$f$ hölderienne d'exposant $\alpha$ si : \[\exists c>0; \forall x,y, |f(x)-f(y)|\leq c|x-y|^\alpha, 0<\alpha<1\]}

\Theo{Unicité et existence}{Soit $\partial\Omega$ de classe $\mathcal{C}^1$, L uniformément elliptique : \[\exists \alpha>0; \forall x\in\overline{\Omega}, \forall \xi\in\mathbb{R}^N, \sum_{i,j=1}^N a_{ij}(x) \xi_i \xi_j \geq \alpha |\xi|^2\]
On suppose $a_{ij}, b_i, c\in\mathcal{C}^{0,\alpha}(\Omega)$ (continue et hölderienne), $\alpha\in]0,1[, c\geq 0$.\\
$f\in\mathcal{C}^{0,\alpha}(\overline{\Omega}), g\in\mathcal{C}^0(\partial \Omega)$.\\
Alors $\exists ! u$ solution de (\ref{pblm}) tel que $u\in\mathcal{C}^{2,\alpha}(\Omega)\cap\mathcal{C}^0(\overline{\Omega})$. }

\Theo{estimation de Schender}{Si de plus, $\partial \Omega$ de classe $\mathcal{C}^{2,\alpha}$, $g\in\mathcal{C}^{2,\alpha}(\partial \Omega)$, alors $u\in\mathcal{C}^{2,\alpha}(\overline{\Omega})$ et on a : 
\[\|u\|_{\mathcal{C}^{2,\alpha}(\overline{\Omega})}\leq c\left( \|f\|_{\mathcal{C}^{0,\alpha}(\overline{\Omega})} + \|g\|_{\mathcal{C}^{2,\alpha}(\partial\Omega)} \right)\]}

\section{Les espaces $L^p$}
\subsection{Rappels d'analyse fonctionnelle}
\Def{Dual}{Soit $X$ un evn. On appelle dual de $X$ l'espace \[X'=\mathcal{L}(X,\mathbb{R})\]
Si $\phi\in X'$ et $x\in X$, on note souvent : \[\phi(x)=\langle \phi, x\rangle_{X'X}\]
appelé crochet de dualité.}

\Def{Bidual}{Soit $X$ un evn. On appelle bidual de $X$ l'espace \[X''=(X')'\] qui est un Banach.}

\textbf{Remarque : } On peut identifier $X$ avec un sous-espace de $X''$ à travers une isométrie, de la manière suiva,te : $\forall x\in X$, on définit : \[f_x : x'\in X' \mapsto \langle x',x\rangle_{X'X}\in\mathbb{R}\]
$f_x$ est dans $X''$ car linéaire, et $|\langle x',x\rangle|\leq \|x\|_X \|x'\|_{X'}$ donc $f_x$ est borné.

On peut montrer que : \[\mathcal{F} : x\in X \mapsto f_x\in X''\] est une isométrie, ie $\|x\|_X=\|f_x\|_{X''},\ \forall x\in X$. Donc on identifie $x$ avec $f_x$ et on écrit $X\subset X''$.\\
Question : a-t-on $X=X''$ ? autrement dit, $\mathcal{F}$ est-elle surjective ? En général, non.

\Def{Reflexif}{Si $\mathcal{F}$ est surjective, on dit que $C$ est reflexif.}

\Theo{représentation de Riesz-Fréchet}{Soit $H$ de Hilbert. \[\forall F\in H', \exists ! \tau(F)\in H; \forall x\in H, \langle F,x\rangle_{H'H}=(\tau(F), x)_H\]
De plus, l'application \begin{eqnarray*} \Phi : H'&\to&H \\ F&\mapsto& \tau(F) \end{eqnarray*} est une isométrie.}

\subsection{Les espaces $L^p$}
Dans la suite, $O$ est un ouvert de $\mathbb{R}^N$, $N\geq 2$\\
$\Omega$ est un ouvert borné de $\mathbb{R}^N$\\
$dx$ la mesure de Lebesgue\\

\Def{}{Soit $1\leq p< +\infty$.
\[L^p(O)=\{f:O\to\mathbb{R} \text{ mesurable }; \int|f|^p dx<\infty\}\]
\[L^p(O)=\{f:O\to\mathbb{R} \text{ mesurable }; |f|<\infty \text{ p.p. dans } O\}\]
\[\forall 1\leq p\leq+\infty, L^p_{loc}(O)=\{f\in L^p(\omega), \forall \omega \text{ ouvert borné}, \bar{\omega}\subset O\}\]}

\Prop{}{$L^p(O)$ est de Banach muni de la norme : \[\|f\|_{L^p(O)}=\left| \begin{array}{r c l} 
	\left(\int_O |f|^p dx \right)^{\frac{1}{p}} &\text{ si }& p<\infty\\
	\inf\{C; |f|\leq C \text{ pp}\} &\text{ si }& p=\infty
\end{array}\right.\]}

\Rem{}{Si $p=2$, $L^2(O)$ est un Hilbert par rapport au produit scalaire \[(f,g)_{L^2(O)}=\int_O f(x)g(x)dx\]}

\Prop{inégalité de Holder}{Soit $1\leq p\leq +\infty$. On pose 
\[p'=\left| \begin{array}{c c c}
	\frac{p}{p-1} &\text{ si }& 1<p<+\infty\\
	1 &\text{ si }& p=+\infty\\
	+\infty &\text{ si }& p=1
\end{array}\right.\]
appelé le conjugué.\\
\[\forall f\in L^p(O), \forall g\in L^{p'}(O), \int_O |f(x)g(x)|dx \leq \|f\|_{L^p(O)} \|g\|_{L^{p'}(O)}\]}

\Coro{}{$1\leq p\leq +\infty$, $p'$ son conjugué.\\
Si $f_n\to f$ dans $L^p(O)$ et $g\in L^{p'}(O)$ alors : \[\lim_{n\to +\infty} \int_O f_n g dx = \int_O fg dx\]}

\Coro{}{$1\leq p < q \leq +\infty$, $\Omega$ ouvert borné de $\mathbb{R}^N$. Alors $L^q(\Omega)\subset L^p(\Omega)$ et $\|f\|_{L^p(\Omega)}\leq c \|f\|_{L^q(\Omega)}$ où $c=c(|\Omega|, p, q)$.}


