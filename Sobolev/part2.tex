\part{Espaces de Sobolev}
\Def{}{$1\leq p\leq +\infty$. On définit, pour $O$ ouvert de $\mathbb{R}^N$ : \[W^{1,p}(O)=\{v\in L^p(O); \frac{\partial v}{\partial x_i}\in L^p(O), \forall i=1,...,N\}\]
où $\frac{\partial v}{\partial x_i}$ est donnée au sens des distributions.\\
On munit cet espace de la norme : \[\|w\|_{W^{1,p}(O)}=\|w\|_{L^p(O)} + \sum_{i=1}^N \left\|\frac{\partial w}{\partial x_i}\right\|_{L^p(O)}\]
Pour $p=2$, on note $W^{1,p}(O)=H^1(O)$.}

\Prop{}{$1\leq p<+\infty$. La norme $\|\bullet\|_{W^{1,p}(O)}$ est équivalente à la norme : \[\|u\|=\left( \|u\|^p_{L^p(O)} + \|\nabla u\|^p_{L^p(O)}\right)^{\frac{1}{p}} \]
où \[\|\nabla u\|^p_{L^p(O)}=\sum_{i=1}^N \left\|\frac{\partial u}{\partial x_i} \right\|^p_{L^p(O)}\]}

\textbf{Remarque :} Puisque les constantes de l'inégalité sont indépendantes de l'ouvert et ne dépend que de $W$ et $p$, on utilisera l'une des deux indifférement.

\Prop{}{\begin{itemize}
	\item $1\leq p\leq +\infty$, $W^{1,P}(O)$ est un espace de Banach avec la norme associée
	\item $H^1(O)$ est un Hilbert par rapport au produit scalaire : \[(u,v)_{H^1(O)}=(u,v)_{L^2(O)} + \sum_{i=1}^N \left( \frac{\partial u}{\partial x_i}, \frac{\partial v}{\partial x_i}\right)_{L^2(O)}\]
\end{itemize}}

\Prop{}{$W^{1,p}(O)$ séparable si $1\leq p<+\infty$, réflexif si $1<p<+\infty$}

\Prop{}{\begin{itemize}
	\item $1\leq p<+\infty$, $\forall O_1\subset O$, $u\in W^{1,p}(O)\Rightarrow u\in W^{1,p}(O_1)$
	\item $\psi\in\mathcal{D}(O)$, $u\in W^{1,p}(O)$, alors $\psi u\in W^{1,p}(O)$ et \[\frac{\partial (\psi u)}{\partial x_i} = u\frac{\partial \psi}{\partial x_i} + \psi \frac{\partial u}{\partial x_i}\]
\end{itemize}}

\Lem{}{$1\leq p\leq +\infty$, $\phi\in\mathcal{D}(\mathbb{R}^N)$, $u\in W^{1,p}(\mathbb{R}^N)$.
\[\phi * u \in \mathcal{C}^{\infty}(\mathbb{R}^N) \text{ et } \frac{\partial}{\partial x_i} (\phi * u)=\phi * \frac{\partial u}{\partial x_i}\]}

\Theo{}{$1\leq p<+\infty$\\
$\mathcal{D}(\mathbb{R}^N)$ est dense dans $W^{1,p}(\mathbb{R}^N)$}

\section{Restriction à un ouvert}
\Def{ouvert à frontière lipschitzienne}{Soit $N\geq 2$, $\Omega$ ouvert borné.\\
On définit un système de coordonnées locales de la manière suivante : \\
On suppose qu'il existe $m\in \mathbb{N}^*$ et m fonctions \[\psi_i : Q=]-1,1[^{N-1}\times\mathbb{R}\to\mathbb{R}\]
et $\exists r>0$ tel que : \[\begin{array}{c c c c}
\psi_i : & U=Q\times]-r,r[&\to& \psi_i(U) \\
	& (y',y_N) &\mapsto& (y', y_N+\psi_i(y'))
\end{array}\]
alors $\psi_i$ est un homéomorphisme entre $U$ et $\psi_i(U)$ et $\forall i$ :
\begin{eqnarray*}
	\Gamma_i&=&\psi_i(Q\times\{0\})\subset\partial\Omega\\
	U_i^+&=&\psi_i(Q\times ]0,r[)\subset\Omega\\
	U_i^-&=&\psi_i(Q\times ]-r,0[)\subset\Omega\\
\end{eqnarray*}
et \[\partial \Omega=\bigcup_{i=1}^m \Gamma_i\]
On dit que $\partial \Omega$ est lipschitienne (resp. $\mathcal{C}^k$) s'il existe un système de coordonnées locales tel que $\forall i$, $\psi_i$ est lipschitzienne (resp. $\mathcal{C}^k$)}

\Theo{de prolongement}{$1\leq p\leq +\infty$\\
Soit $\Omega\subset \mathbb{R}^N$ et on suppose 3 cas : 
\begin{itemize}
	\item $N=1$ : $\Omega$ est un intervalle ouvert de $\mathbb{R}$ (borné ou non)
	\item $N\geq 2$ : \begin{itemize}
		\item $\Omega$ est le demi-espace $\mathbb{R}^{n-1}\times \mathbb{R}^*_+$
		\item $\Omega$ ouvert borné avec $\partial\Omega$ lipschitzienne
	\end{itemize}
\end{itemize}
Alors il existe un opérateur de prolongement $p$ linéaire et continu \[p : W^{1,p}(\Omega)\to W^{1,p}(\mathbb{R}^N)\]
tel que : \begin{enumerate}
	\item $Pu=u$ sur $\Omega$
	\item $\|Pu\|_{L^p(\mathbb{R}^N)}\leq c\|u\|_{L^p(\Omega)}$\\
		$\|Pu\|_{W^{1,p}(\mathbb{R}^N)}\leq c\|u\|_{W^{1,p}(\Omega)}$\\
	où $c=c(\Omega,p)$.
\end{enumerate}}

\Def{}{$\Omega\subset\mathbb{R}^N$ comme dans le théorème de prolongement.\\
On note $\mathcal{D}(\overline{\Omega})$ (resp. $\mathcal{C}^1_c(\overline{\Omega})$) l'ensemble des restrictions à $\overline{\Omega}$ de fonctions de $\mathcal{D}(\mathbb{R}^N)$ (resp. $\mathcal{C}^1_c(\mathbb{R}^N)$).\\
Si $\Omega=\mathbb{R}^{N-1}\times\mathbb{R}^+_*$, on note $\mathcal{D}(\mathbb{R}^{N-1}\times\mathbb{R}^+)$}

\textbf{Remarque :} $\mathcal{D}(\Omega) \subsetneq \mathcal{D}(\overline{\Omega})$ car les fonctions de $\mathcal{D}'\overline{\Omega})$ ne s'annulent pas forcément sur $\partial\Omega$.

\Theo{}{$\Omega$ ouvert de $\mathbb{R}^N$ comme dans le théorème de prolongement, $1\leq p<+\infty$.\\
Alors $\mathcal{D}(\overline{\Omega})$ est dense dans $W^{1,p}(\Omega)$.}

\Theo{chain rule}{$1\leq p\leq +\infty$, $\Omega\subset\mathbb{R}^N$ comme dans le théorème de prolongement.\\
Soit $G\in\mathcal{C}^1(\mathbb{R})$ tel que $G(0)=0$ et $\forall s$, $|G'(s)|\leq M$\\
Alors $\forall u\in W^{A,p}(\Omega)$, $G(u)\in W^{1,p}(\Omega)$ et on a (au sens des distributions) : \[\nabla G(u) = G'(u) \nabla u\]}

\Theo{Stampacchia}{$1\leq p\leq +\infty$, $\Omega\subset\mathbb{R}^N$ comme dans le théorème de prolongement.\\
$\forall u\in W^{1,p}(\Omega)$, on pose \[u_+=\max\{u,0\},\ u_-=\min\{u,0\},\ _=u_-+u_+\]
Alors $u_+$, $u_-$ et $|u|$  appartiennent ) $W^{1,p}(\Omega)$ et on a presque partout : 
\[\nabla u_+ = \left| \begin{array}{c c c}
	\nabla u &\text{ où }& u>0 \\
	0 &\text{ où }& u\leq 0
\end{array}\right.\]
\[\nabla u_- = \left| \begin{array}{c c c}
	0 &\text{ où }& u\geq 0 \\
	\nabla u &\text{ où }& u< 0
\end{array}\right.\]
\[\nabla| u| = \left| \begin{array}{c c c}
	\nabla u &\text{ où }& u>0 \\
	0 &\text{ où }& u=0 \\
	-\nabla u &\text{ où }& u<0
\end{array}\right.\]
}
