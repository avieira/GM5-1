\part{Espaces de Sobolev}
\Def{}{$1\leq p\leq +\infty$. On définit, pour $O$ ouvert de $\mathbb{R}^N$ : \[W^{1,p}(O)=\{v\in L^p(O); \frac{\partial v}{\partial x_i}\in L^p(O), \forall i=1,...,N\}\]
où $\frac{\partial v}{\partial x_i}$ est donnée au sens des distributions.\\
On munit cet espace de la norme : \[\|w\|_{W^{1,p}(O)}=\|w\|_{L^p(O)} + \sum_{i=1}^N \left\|\frac{\partial w}{\partial x_i}\right\|_{L^p(O)}\]
Pour $p=2$, on note $W^{1,p}(O)=H^1(O)$.}

\Prop{}{$1\leq p<+\infty$. La norme $\|\bullet\|_{W^{1,p}(O)}$ est équivalente à la norme : \[\|u\|=\left( \|u\|^p_{L^p(O)} + \|\nabla u\|^p_{L^p(O)}\right)^{\frac{1}{p}} \]
où \[\|\nabla u\|^p_{L^p(O)}=\sum_{i=1}^N \left\|\frac{\partial u}{\partial x_i} \right\|^p_{L^p(O)}\]}

\textbf{Remarque :} Puisque les constantes de l'inégalité sont indépendantes de l'ouvert et ne dépend que de $W$ et $p$, on utilisera l'une des deux indifférement.

\Prop{}{\begin{itemize}
	\item $1\leq p\leq +\infty$, $W^{1,P}(O)$ est un espace de Banach avec la norme associée
	\item $H^1(O)$ est un Hilbert par rapport au produit scalaire : \[(u,v)_{H^1(O)}=(u,v)_{L^2(O)} + \sum_{i=1}^N \left( \frac{\partial u}{\partial x_i}, \frac{\partial v}{\partial x_i}\right)_{L^2(O)}\]
\end{itemize}}

\Prop{}{$W^{1,p}(O)$ séparable si $1\leq p<+\infty$, réflexif si $1<p<+\infty$}

\Prop{}{\begin{itemize}
	\item $1\leq p<+\infty$, $\forall O_1\subset O$, $u\in W^{1,p}(O)\Rightarrow u\in W^{1,p}(O_1)$
	\item $\psi\in\mathcal{D}(O)$, $u\in W^{1,p}(O)$, alors $\psi u\in W^{1,p}(O)$ et \[\frac{\partial (\psi u)}{\partial x_i} = u\frac{\partial \psi}{\partial x_i} + \psi \frac{\partial u}{\partial x_i}\]
\end{itemize}}

\Lem{}{$1\leq p\leq +\infty$, $\phi\in\mathcal{D}(\mathbb{R}^N)$, $u\in W^{1,p}(\mathbb{R}^N)$.
\[\phi * u \in \mathcal{C}^{\infty}(\mathbb{R}^N) \text{ et } \frac{\partial}{\partial x_i} (\phi * u)=\phi * \frac{\partial u}{\partial x_i}\]}

\Theo{}{$1\leq p<+\infty$\\
$\mathcal{D}(\mathbb{R}^N)$ est dense dans $W^{1,p}(\mathbb{R}^N)$}

\section{Restriction à un ouvert}
\Def{ouvert à frontière lipschitzienne}{Soit $N\geq 2$, $\Omega$ ouvert borné.\\
On définit un système de coordonnées locales de la manière suivante : \\
On suppose qu'il existe $m\in \mathbb{N}^*$ et m fonctions \[\psi_i : Q=]-1,1[^{N-1}\times\mathbb{R}\to\mathbb{R}\]
et $\exists r>0$ tel que : \[\begin{array}{c c c c}
\psi_i : & U=Q\times]-r,r[&\to& \psi_i(U) \\
	& (y',y_N) &\mapsto& (y', y_N+\psi_i(y'))
\end{array}\]
alors $\psi_i$ est un homéomorphisme entre $U$ et $\psi_i(U)$ et $\forall i$ :
\begin{eqnarray*}
	\Gamma_i&=&\psi_i(Q\times\{0\})\subset\partial\Omega\\
	U_i^+&=&\psi_i(Q\times ]0,r[)\subset\Omega\\
	U_i^-&=&\psi_i(Q\times ]-r,0[)\subset\Omega\\
\end{eqnarray*}
et \[\partial \Omega=\bigcup_{i=1}^m \Gamma_i\]
On dit que $\partial \Omega$ est lipschitienne (resp. $\mathcal{C}^k$) s'il existe un système de coordonnées locales tel que $\forall i$, $\psi_i$ est lipschitzienne (resp. $\mathcal{C}^k$)}

\Theo{de prolongement}{$1\leq p\leq +\infty$\\
Soit $\Omega\subset \mathbb{R}^N$ et on suppose 3 cas : 
\begin{itemize}
	\item $N=1$ : $\Omega$ est un intervalle ouvert de $\mathbb{R}$ (borné ou non)
	\item $N\geq 2$ : \begin{itemize}
		\item $\Omega$ est le demi-espace $\mathbb{R}^{n-1}\times \mathbb{R}^*_+$
		\item $\Omega$ ouvert borné avec $\partial\Omega$ lipschitzienne
	\end{itemize}
\end{itemize}
Alors il existe un opérateur de prolongement $p$ linéaire et continu \[p : W^{1,p}(\Omega)\to W^{1,p}(\mathbb{R}^N)\]
tel que : \begin{enumerate}
	\item $Pu=u$ sur $\Omega$
	\item $\|Pu\|_{L^p(\mathbb{R}^N)}\leq c\|u\|_{L^p(\Omega)}$\\
		$\|Pu\|_{W^{1,p}(\mathbb{R}^N)}\leq c\|u\|_{W^{1,p}(\Omega)}$\\
	où $c=c(\Omega,p)$.
\end{enumerate}}

\Def{}{$\Omega\subset\mathbb{R}^N$ comme dans le théorème de prolongement.\\
On note $\mathcal{D}(\overline{\Omega})$ (resp. $\mathcal{C}^1_c(\overline{\Omega})$) l'ensemble des restrictions à $\overline{\Omega}$ de fonctions de $\mathcal{D}(\mathbb{R}^N)$ (resp. $\mathcal{C}^1_c(\mathbb{R}^N)$).\\
Si $\Omega=\mathbb{R}^{N-1}\times\mathbb{R}^+_*$, on note $\mathcal{D}(\mathbb{R}^{N-1}\times\mathbb{R}^+)$}

\textbf{Remarque :} $\mathcal{D}(\Omega) \subsetneq \mathcal{D}(\overline{\Omega})$ car les fonctions de $\mathcal{D}'\overline{\Omega})$ ne s'annulent pas forcément sur $\partial\Omega$.

\Theo{}{$\Omega$ ouvert de $\mathbb{R}^N$ comme dans le théorème de prolongement, $1\leq p<+\infty$.\\
Alors $\mathcal{D}(\overline{\Omega})$ est dense dans $W^{1,p}(\Omega)$.}

\Theo{chain rule}{$1\leq p\leq +\infty$, $\Omega\subset\mathbb{R}^N$ comme dans le théorème de prolongement.\\
Soit $G\in\mathcal{C}^1(\mathbb{R})$ tel que $G(0)=0$ et $\forall s$, $|G'(s)|\leq M$\\
Alors $\forall u\in W^{A,p}(\Omega)$, $G(u)\in W^{1,p}(\Omega)$ et on a (au sens des distributions) : \[\nabla G(u) = G'(u) \nabla u\]}

\Theo{Stampacchia}{$1\leq p\leq +\infty$, $\Omega\subset\mathbb{R}^N$ comme dans le théorème de prolongement.\\
$\forall u\in W^{1,p}(\Omega)$, on pose \[u_+=\max\{u,0\},\ u_-=\min\{u,0\},\ _=u_-+u_+\]
Alors $u_+$, $u_-$ et $|u|$  appartiennent ) $W^{1,p}(\Omega)$ et on a presque partout : 
\[\nabla u_+ = \left| \begin{array}{c c c}
	\nabla u &\text{ où }& u>0 \\
	0 &\text{ où }& u\leq 0
\end{array}\right.\]
\[\nabla u_- = \left| \begin{array}{c c c}
	0 &\text{ où }& u\geq 0 \\
	\nabla u &\text{ où }& u< 0
\end{array}\right.\]
\[\nabla| u| = \left| \begin{array}{c c c}
	\nabla u &\text{ où }& u>0 \\
	0 &\text{ où }& u=0 \\
	-\nabla u &\text{ où }& u<0
\end{array}\right.\]
}

\Coro{}{$1\leq p\leq +\infty$, $\Omega\subset\mathbb{R}^N$ comme dans le théorème de prolongement.\\
$w\in W^{1,p}(\Omega) \Rightarrow \nabla u=0$ p.p. sur les lignes de niveau, ie $\forall \alpha$, $\nabla u=0$ p.p. sur $\{u=\alpha\}$}

\Theo{}{$1\leq p\leq +\infty$, $\Omega\subset\mathbb{R}^N$ comme dans le théorème de prolongement, connexe.\\
Si $u\in W^{1,p}(\Omega); \nabla u=0$ dans $\Omega$, alors $u$ est constante.}

\section{Amélioration de la régularité}
Est-ce que la condition $\nabla u\in (L^p(\Omega))^n$ "améliore" vraiment la régularité ou juste la sommabilité de $u$ ?\\
Le théorème suivant repond pour $N=1$ où on gagne beaucoup. Pour $N\geq 2$, la réponse est donnée par les théorèmes d'inclusion de Sobolec où on "gagne moins".

\Theo{}{$1\leq p<+\infty$. On a : \[W^{1,p}(\mathbb{R})\subset \mathcal{C}^0_b(\mathbb{R})\left(=\mathcal{C}^0(\mathbb{R})\cap L^{\infty}(\mathbb{R})\right)\]
et \[\forall u\in W^{1,p}(\mathbb{R}),\ \|u\|_{L^{\infty}(\mathbb{R})}\leq c(p)\|u\|_{W^{1,p}(\mathbb{R})}\]
De plus, si $p>1$, \[\forall x,y\in\mathbb{R},\ |u(x)-u(y)|\leq |x-y|^{\frac{p-1}{p}} \|u\|_{L^p(\mathbb{R})}\] ie $u$ est hölderienne d'exposant $\frac{p-1}{p}=\frac{1}{p'}$}

\textbf{Remarques :} \begin{enumerate}
	\item $\forall I\in\mathbb{R}$, $W^{1,p}(I)\subset \mathcal{C}^0(\bar{I})\cap L^{\infty}(I)$. En particulier, si $u\in W^{1,p}(]a,b[)$, on a $u(a)$ et $u(b)$ bien définis. Cela donne un sens aux conditions de Dirichlet.
	\item $W^{1,p}(\mathbb{R})\subset \mathcal{C}^0_b(\mathbb{R})$ même pour $p=+\infty$
	\item Pour $N\geq 2$, l'inclusion montrée n'est pas vraie en général $\forall p$
\end{enumerate}

\subsection{Notion de trace}
\Theo{de Rademacher}{$f:a\subset\mathbb{R}^N\to\mathbb{R}$, $A$ ouvert, $f$ lipschitzienne sur $A$.\\
$f$ est alors différentiable presque partout et $\nabla f$ est égal à son gradient au sens des distributions presque partout. De plus, $\nabla f\in\left( L^{\infty}(A)\right)^N$.}

\Theo{partition de l'unité}{$F\subset \mathbb{R}^N$, $\geq 2$, $F$ compact.\\
$A_1$, ..., $A_m$ $m$ ouverts de $\mathbb{R}^N$ tel que $F\subset \cup_{i=1}^m A_i$\\
Donc $\forall i=1,...,m$, $\exists \gamma_i\in\mathcal{D}(A_i)$ avec $0\leq \gamma_i\leq 1$ et \[\sum_{i=1}^m \gamma_i (x)=1\ \forall x\in F\]}

\Def{}{$N\geq 2$, $\Omega\subset\mathbb{R}^N$ borné, $\partial\Omega$ lipschitzienne.\\
Soit $\gamma_1,...,\gamma_n$ donnés par le théorème précédent correspondant à $F=\partial\Omega$ et $A_i=V_i$ dans la définition de Ne\v{c}as.\\
Soit $u$ mesurable sur $\partial\Omega$. On dit que $u$ est intégrable sur $\partial\Omega$ si $\forall i=1...m$, les fonctions 
	\[u(y',\psi_i(y'))\gamma_i(y',\psi_i(y'))\sqrt{1+|\nabla\psi_i(y')|^2}\]
est intégrable sur Q.\\
On pose ensuite \[\int_{\partial\Omega}u(x)ds = \sum_{i=1}^m \int_{\Gamma_i}u(x)\gamma_i(x) ds\]
où \[\int_{\Gamma_i}u(x)\gamma_i(x) ds=\int_Q u(y',\psi_i(y'))\gamma_i(y',\psi_i(y'))\sqrt{1+|\nabla\psi_i(y')|^2} dy'\]}

\textbf{Remarque :} On peut montrer que la définition est indépendant des coordonnées locales et des $\gamma_i$

\Def{}{$\Omega$ borné de $\mathbb{R}^N$, $\partial\Omega$ lipschitzienne, $N\geq 2$.\\
On définit $L^p(\partial\Omega$, $1\leq<+\infty$ par : \[L^p(\partial\Omega)=\{u:\partial\Omega\to\mathbb{R} \text{ mesurables égales p.p tel que } \int_{\partial\Omega} |u|^p ds<+\infty\}\]
et \[L^{\infty}(\partial\Omega)=\{f:\partial\Omega\to\mathbb{R}; \exists c>0; |f|\leq c \text{ p.p sur } \partial\Omega\}\]
On munit ces espaces des normes : \[p<+\infty\ :\ \|u\|_{L^p(\partial\Omega)}=\left( \int_{\partial\Omega} |u|^p ds\right)^{\frac{1}{p}}\]
\[p=+\infty\ :\ \|u\|_{L^\infty(\partial\Omega)}=\inf \{c; |f|<c \text{ p.p. sur } \partial\Omega\}\]}

\Prop{}{$L^p(\partial\Omega)$ est de Banach $\forall 1\leq p\leq +\infty$, Hilbert pour $p=2$.}

Pour la suite, on prend $p=2$.

\Theo{de trace}{$N\geq 2$.
\begin{enumerate}
	\item $\exists!\gamma : H^1(\mathbb{R}^{N-1}\times\mathbb{R}^*_+)\to L^2(\mathbb{R}^{N-1})$ linéaire continue appelée trace, tel que 
\[\gamma(u)=\restriction{u}{\mathbb{R}^{N-1}}\ \forall u\in H^1\left(\mathbb{R}^{N-1}\times\mathbb{R}_*^+\right)\cap\mathcal{C}^0\left(\mathbb{R}^{N-1}\times\mathbb{R}_+\right)\]
	\item Si $\Omega$ est un ouvert borné de $\mathbb{R}^N$, $\partial\Omega$ lipschitzienne, alors $\exists!\gamma : H^1(\Omega)\to L^2(\partial\Omega)$ linéaire continue, tel que 
\[\gamma(u)=\restriction{u}{\partial\Omega}\ \forall u\in H^1(\Omega)\cap\mathcal{C}^0\left(\overline{\Omega}\right)\]
\end{enumerate}}

\textbf{Problème :} On peut montrer que $\gamma$ n'est pas surjective sur $L^2(\partial\Omega)$.

\Def{}{On pose \[H^{\frac{1}{2}}(\partial\Omega)=\gamma\left(H^1(\Omega)\right)\subset L^2(\partial\Omega)\]}

\Theo{}{$\Omega$ borné, $\partial\Omega$ lipschitzienne (ou $\Omega=\mathbb{R}^{N-1}\times\mathbb{R}^*_+$). Alors : 
\begin{enumerate}
	\item $H^{\frac{1}{2}}(\partial\Omega)$ est un Banach par rapport à : \[\|u\|^2_{H^{\frac{1}{2}}(\partial\Omega)}=\int_{\partial\Omega} |u|^2 ds+\int_{\partial\Omega}\int_{\partial\Omega} \frac{|u(x)-u(y)|}{|x-y|^{N-1}} ds_x ds_y\]
	\item $\{\restriction{u}{\partial\Omega}, u\in\mathcal{C}^{\infty}(\mathbb{R}^N)\}$ dense dans $H^{\frac{1}{2}}(\partial\Omega)$
	\item $\gamma:H^1(\Omega)\to H^{\frac{1}{2}}(\partial\Omega)$ est linéaire continue, ie 
\[\|\gamma(u)\|_{H^{\frac{1}{2}}(\partial\Omega)}\le c\|u\|_{H^1(\Omega)}\]
	\item Il existe un relevement continue de la trace, ie $\exists g$ linéaire continue tel que 
\[g:\begin{array}{c c c} H^{\frac{1}{2}}(\partial\Omega)&\to&H^1(\Omega)\\ u&\mapsto& U\end{array}\] avec $\gamma(U)=u$.
\end{enumerate}}

\Theo{}{$\Omega$ borné de $\mathbb{R}^N$, $N\geq 2$, $\partial\Omega$ lipschitzienne. On note $n(x)$ le vecteur normal unitaire à $\partial\Omega$.\\
Alors $\forall u, v\in H^1(\Omega)$ : \[\int_{\Omega}u\frac{\partial v}{\partial x_i}dx = \int_{\partial\Omega} \gamma(u)\gamma(v) n_i ds - \int_{\Omega} v\frac{\partial u}{\partial x_i} dx,\ i=1..n\]}

Dans la suite, on noter $\gamma(u)$ simplement $u$, en retenant que c'est la trace.

\Def{}{$1\leq p\leq +\infty$.\\
$W^{1,p}_0(O)$ est la fermeture de $\mathcal{D}(O)$ dans la norme $W^{1,p}(O)$.}

\Rem{}{\begin{itemize}
	\item $W^{1,p}_0(O)$ est un espace fermé de $W^{1,p}(O)$
	\item $H^1_0(O)$ de Hilbert
	\item D'après le théorème de densité dans $\mathbb{R}^N$, on a \[W^{1,p}_0(\mathbb{R}^N)=W^{1,p}(\mathbb{R}^N)\]
\end{itemize}}

\Prop{}{$1\leq p\leq +\infty$. Si $u\in W^{1,p}_0(O)$, alors son prolongement par 0 : \[\tilde{u}=\left|\begin{array}{c c} u &\text{ dans } O \\ 0 &\text{ sinon} \end{array}\right.\]
vérifie $\tilde{u}\in W^{1,p}(\mathbb{R}^N)$ et $\tilde{u}\in W^{1,p}_0(O_1)$, $\forall O\subset O_1$.\\
De plus, \[\|u\|_{W^{1,p}(O)}=\|\tilde{u}\|_{W^{1,p}(O_1)}=\|\tilde{u}\|_{W^{1,p}(\mathbb{R}^N)}\]}

\Prop{}{$1\leq p\leq +\infty$, $\Omega$ intervalle de $\mathbb{R}$ si $N=1$ ou $\Omega$ borné, $\partial\Omega$ lipschitzienne si $N\geq 2$.\\
Si $u\in W^{1,p}(\Omega)$, $u$ à support compact inclu dans $\Omega$, alors $u\in W^{1,p}_0(\Omega)$.}

\textbf{Remarque :} On peut remarquer que l'hypothèse $\partial\Omega$ lipschitzienne n'est pas nécessaire.
