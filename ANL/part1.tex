\section*{Introduction}
On s'intéresse aux équations de la forme :
\[\Pi : \left\{\begin{array}{l c l}
	\dot{x}(t)&=&F(x(t),u(t))\\
	x(0)&=&x_0
\end{array}\right.\]
$u\in\mathbb{R}^m$ le contrôle\\
$y\in\mathbb{R}^p$ les observations\\
$x\in\mathbb{R}^n$ l'état.

\bigskip
La solution à cette équation est en générale lisse, de dimension finie mais elle n'est pas uniquement déterminée par la condition initiale $x_0$.\\
Comment choisir $u$ ?
\begin{itemize}
	\item $u=u(t)$ : $\Pi$ est une équation différentielle non autonome. On peut avoir unicité des solutions. Contrôle en boucle ouverte.
	\item $u=u(x)$ : $\Pi$ est une équation différentielle autonome, uncité des solutions. Contrôle en boucle fermée / par bouclage / par feedback.
\end{itemize}

\section{Outils mathématiques}
\subsection{Difféomorphismes}
\Def{}{$h:X\to\mathbb{R},\ X\subset \mathbb{R}$ ouvert est $\mathcal{C}^{\infty}$ si $\frac{\partial^ih}{\partial x_1^{i_1}...\partial x_n^{i_n}}$, $i=\sum_{j=1}^n i_j$ existent et sont continues pour tout n-uplet $(i_j)_j$.\\
Si maintenant, $h:X\to Y,\ Y\subset\mathbb{R}^m$ ouvert, h est $\mathcal{C}^{\infty}$ si $h_1,...,h_m$ est $\mathcal{C}^{\infty}$.}

\Def{Difféomorphisme}{$h$ est un difféomorphisme si :
\begin{itemize}
	\item $h$ est injective : $x\neq \tilde{x} \Rightarrow h(x)\neq h(\tilde{x})$
	\item $h$ est surjective : $\forall y\in Y, \exists x\in X;\ y=h(x)$
	\item $h$ et $h^{-1}$ sont $\mathcal{C}^{\infty}$.
\end{itemize}}

\Def{Difféormorphisme local}{$h:X\to Y$ est un difféormorphisme local autour de $x_0$ et $y_0$ s'il existe $X_{x_0}$ et $Y_{y_0}$, deux voisinages ouverts, tels que : \begin{itemize}
	\item $h(X_0)=Y_0$
	\item $\restriction{h}{X_0}$ est un difféomorphisme.
\end{itemize}}

\Theo{}{Supposons $h:X\to Y$ tel que $h(x_0)=y_0$ et :
\begin{enumerate}
	\item $h\in\mathcal{C}^{\infty}$
	\item $\frac{\partial h}{\partial x}(x_0)$ inversible
\end{enumerate}
alors $h$ est un difféomorphisme local en $x_0$ et $y_0$.}

\subsection{Application tangente}
\Def{Vecteur tangent}{Soit $\gamma:]-\varepsilon;\varepsilon[\to X$. Le vecteur tangent à $\gamma(t)$ en $p=\gamma(0)$ est $\dot{\gamma}(0)$.}

\Def{Espace tangent en p}{Soit $X\subset\mathbb{R}^n$ un ouvert. On appelle espace tangent en p :
	\[T_pX=\{\dot{\gamma}(0), \gamma \text{ une sourbe passant par } p\}\]}

\Def{Champ de vecteurs}{Un champ de vecteur $f$ sur $X$ est : \[p\in X\mapsto f(p)\in T_pX\]
On note : \[f(x)=\begin{pmatrix} f_1(x)\\\vdots\\f_n(x)\end{pmatrix}=\sum_{i=1}^n f_i(x)\frac{\partial}{\partial x_i}\]}

A chaque $f$, un champ de vecteurs, on associe l'équation différentielle $\dot{x}=f(x)$. $f$ est $\mathcal{C}^{\infty}$.\\
On va noter $V^{\infty}(X)$ l'ensemble des champs de vecteurs $\mathcal{C}^{\infty}$. 

\bigskip
Soit $\gamma_t(x_0)=x(t,x_0)=x_t(x_0)$ la solution passant par $x_0$. 

\Propo{}{$\gamma_t$ satisfait : \begin{enumerate}
	\item $\gamma_0=id$
	\item $\gamma_s\circ\gamma_t(x_0)=\gamma_{s+t}(x_0)=\gamma_{t+s}(x_0)$
	\item $\gamma_t^{-1}=\gamma_{-t}$
\end{enumerate}}

On prend à présent $h$, un difféormorphisme :
	\[h:X\to Y,\ \dim X=\dim Y=n\]
$h\circ \gamma$ est une courbe dans $Y$.
\begin{eqnarray*}
	\frac{d}{dt}(h\circ \gamma)(0)&=&Dh(\gamma(0)) \frac{d}{dt}\gamma(0)\\
					&=& Dh(p) v
\end{eqnarray*}
Si on note $w$ le vecteur tangent dans $Y$, on a :
	\[W=\frac{\partial h}{\partial x} v \in T_{h(p)}Y\]
	\[W_i=\sum_{j=1}^n \frac{\partial h_i}{\partial x_j} v_j\]

En partant de $H$, non linéaire, on arrive à $Dh=\frac{\partial h}{\partial x}=h_*$ une application linéaire qui tranforme $T_pX$ en $T_{h(p)}Y$. Cette application linéaire est appelée application tangente.

\Lem{}{On a : \[(\phi_* f)(p)=D\phi(\phi^{-1}(p)) f(\phi^{-1}(p))\]}

\Propo{}{Le diagramme suivant commute :\\
\begin{center}
\begin{tikzpicture}
	\node (Pi) at (0,0) {$\dot{x}=f(x)$};
	\node[below= of Pi] (solPi) {$x(t)$};
	\node[right=5cm of Pi] (PiTr) {$\dot{y}=(\phi_*f)(y)$};
	\node[below= of PiTr] (solPiTr) {$y(t)=\phi(x(t))$};
	\draw[->] (Pi)--(PiTr) node[midway, below]{$\phi_*$ transforme};
	\draw[->] (solPi)--(solPiTr) node[midway, below]{$\phi$ transforme};
	\draw[->] (Pi)--(solPi) node[midway, right]{résolution};
	\draw[->] (PiTr)--(solPiTr) node[midway, right]{résolution};
\end{tikzpicture}
\end{center}}

\Propo{}{Soit $\gamma_t$ le flot de $\dot{x}=f(x)$. Alors $\sigma_t$, le flot de $\dot{y}=(\phi_*f)(y)$ est : \[\sigma_t=\phi\circ\gamma_t\circ\phi^{-1}\]}

\textbf{Idée de la démonstration :} On vérifie simplement que le flot $\sigma_t$ vérifie bien l'équation $\frac{d}{dt} \sigma_t = (\phi_* f)(\sigma_t)$.

\subsection{Crochet de Lie}
\Def{}{On note $V^{\infty}(X)$ l'ensemble de tous les champs de vecteurs sur $X$ de classe $\mathcal{C}^{\infty}$.}

\Def{Crochet de Lie}{Soit $f,g\in V^{\infty}(X)$. On définit : \[[f,g](p)=\restriction{\frac{\partial}{\partial t} (\gamma_{-t}^f)_* g(p)}{t=0}\]}

\Lem{}{On a également, en coordonnées $x=(x_1,...,x_n)$ : \[[f,g](p)=\frac{\partial g}{\partial x}(p) f(p) - \frac{\partial f}{\partial x}(p)g(p)\]}

\section{Controlabilité des systèmes}
$\dot{x}=u_1f(x)+u_2g(x)$, où $x\in X\subset \mathbb{R}^n$, ouvert. $u_1, u_2\in\mathbb{R}$.

\Propo{}{\[\forall p\in X,\ \forall t,s\in\mathbb{R},\  \gamma_s^{-g}\circ\gamma_t^{-f}\circ\gamma_s^g\circ\gamma_t^f(p)=p \Leftrightarrow [f,g]\equiv 0\]}

\Lem{}{Soient $f,g\in V^{\infty}(X)$ et $\phi$ un difféomorphisme. Alors : \[\phi_*[f,g]=[\phi_*f,\phi_*g]\]}

\subsection{Les systèmes linéaires}
$\dot{x}=Ax+Bu$, $x\in\mathbb{R}^n$, $u\in\mathbb{R}^m$, $A\in\mathcal{M}_{n\times n}(\mathbb{R})$, $B\in\mathcal{M}_{n\times m}(\mathbb{R})$.

On note $R_T(x_0)$ l'ensemble des points accessibles depuis $x_0$ au temps $T$. \\
$R(x_0)=\bigcup_{t>0} R_t(x_0)$ : ensemble d'accessibilité depuis $x_0$. 

\Theo{}{Les conditions suivantes sont équivalentes : 
\begin{enumerate}
	\item $R(0)=\mathbb{R}^n$
	\item $\exists T>0;\ R_T(0)=\mathbb{R}^n$
	\item $\forall T>0,\ R_T(0)=\mathbb{R}^n$
	\item $\forall x_0\in\mathbb{R}^n,\ R(x_0)=\mathbb{R}^n$
	\item $\exists T>0; \forall x_0\in\mathbb{R}^n, R_T(x_0)=\mathbb{R}^n$
	\item $\exists T>0; \forall x_0\in\mathbb{R}^n, R_T(x_0)=\mathbb{R}^n$
	\item Rg$(B,...,A^{n-1}B)=n$
\end{enumerate}}

\subsection{Systèmes non linéaires}
On note \ref{Pi} le problème  : 
\begin{equation} \label{Pi} \tag{$\Pi$} \dot{x}=F(x,u),\ x\in X,\text{ ouvert de } \mathbb{R}^n,\ u\in U\subset \mathbb{R}^m \end{equation}
$U$ est la classe des contrôles admissibles.\\
\[PC_U\subset U \subset \mathcal{M}_U\]
Où $PC_U$ est l'ensemble des contrôles constants par morceaux à valeur dans $U$ et $\mathcal{M}_U$ est l'ensemble des contrôles mesurables à valeur dans $U$.

\[R_T(x_0)=\left\{x(T,u,x_0), u\in U([0,T])\right\}\]
où $x(T,u,x_0)$ est la trajectoire de $\dot{x}=F(x,u)$ oassabt par $x_0$ en $t=0$. 

\Def{}{\ref{Pi} est accessible en $x_0$ si $\widering{R(x_0)}=\emptyset$\\
\ref{Pi} est fortement accessible en $x_0$ si $\forall T>0,\ \widering{R_T(x_0)}=\emptyset$}

Supposons que $(x_e,u_e)$ soit un point d'équilibre, et on linéarise \ref{Pi} autour de ce point d'équilibre : 
\begin{eqnarray*}
	z&=&x-x_e\\
	v&=&u-u_e
\end{eqnarray*}

On aura donc : \[\dot{z}=\underbrace{\frac{\partial F}{\partial x} (x_e,u_e)}_{=A}(x-x_e)+\underbrace{\frac{\partial F}{\partial u}(x_e,u_e)}_{=B}(u-u_e) + ... \]

\Propo{}{Si (A,B) est contrôlable en $(x_0,u_0)$, alors \[x_0\in \widering{R_T(x_0)}, \forall T>0\] (controlabilité locale)\\
Cela implique que si \ref{Pi} est fortement accessible en $x_0$, donc il est accessible en $x_0$.}

On pose $\mathcal{F}=\{F_u=F(\bullet,u),u\in \mathcal{U}\}$, appelée collection de champ de vecteurs.

\Def{Algèbre de Lie}{L'algèbre de Lie $\mathcal{L}$ de $\Pi$ est le plus petit espace vectoriel (sur $\mathbb{R}$) tel que : 
\begin{enumerate}
	\item $\mathcal{F}\subset\mathcal{L}$
	\item $\mathcal{L}$ est fermée par rapport au $[\bullet,\bullet]$, ie : \[f,g\in\mathcal{L}\Rightarrow [f,g]\in\mathcal{L}\]
\end{enumerate}}

\Theo{de Sussman-Jevdjevic}{Considérons \ref{Pi} où $X\subset\mathbb{R}^n$, ouvert, et $x_0\in X$. 
\begin{enumerate}
	\item Si dim $\mathcal{L}(x_0)=n \Rightarrow$ \ref{Pi} est accessible en $x_0$
	\item Si \ref{Pi} est analytique, alors dim $\mathcal{L}(x_0)=n \Leftrightarrow$ \ref{Pi} accessible en $x_0$.
\end{enumerate}}

\Propo{}{Pour \ref{Pi}, on a : \[\mathcal{L}=vect\left\{ [F_{u_1},...[F_{u_{k-1}},F_{u_k}]], k\geq 1, u_j\in\mathcal{U}\right\}\]}

\Def{Algèbre de Lie}{Une algèbre de Lie A est un espace vectoriel A munie d'une opération $[\bullet,\bullet]:A\times A\to A$ tel que : 
\begin{enumerate}
	\item $[\bullet,\bullet]$ est bilinéaire
	\item $[\bullet,\bullet]$ est antisymétrique
	\item $[\bullet,\bullet]$ satisfait l'identité de Jacobi : \[\forall a,b,c\in A,\ \left[ a,\left[b,c\right]\right]=\left[ \left[a,b\right],c\right] + \left[ b,\left[a,c\right]\right]\]
\end{enumerate}}

\Def{Sous-variété}{Une sous-variété dans $\mathbb{R}^d$ de dimension $n$ est : \[X=\left\{ x\in\mathbb{R}^d; \phi(x)=0\right\}\]
où $\phi:\mathbb{R}^d\to\mathbb{R}^k$ tel que rg$\frac{d\phi}{dx}(x)=k$, $\forall x\in X$ et $n=d-k$.\\
(Par force, $d\geq k$)}
