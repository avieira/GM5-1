\documentclass{article}
\input{../../preambule}

\def\restriction#1#2{\mathchoice
              {\setbox1\hbox{${\displaystyle #1}_{\scriptstyle #2}$}
              \restrictionaux{#1}{#2}}
              {\setbox1\hbox{${\textstyle #1}_{\scriptstyle #2}$}
              \restrictionaux{#1}{#2}}
              {\setbox1\hbox{${\scriptstyle #1}_{\scriptscriptstyle #2}$}
              \restrictionaux{#1}{#2}}
              {\setbox1\hbox{${\scriptscriptstyle #1}_{\scriptscriptstyle #2}$}
              \restrictionaux{#1}{#2}}}
\def\restrictionaux#1#2{{#1\,\smash{\vrule height .8\ht1 depth .85\dp1}}_{\,#2}}


\title{Automatique non linéaire}

\hypersetup{colorlinks=true, urlcolor=bleu, linkcolor=red}

%Def = Definition
%Theo = Théorème
%Prop = Propriété
%Coro = Corollaire
%Lem = Lemme

\makeatletter
\@addtoreset{section}{part}
\makeatother

\begin{document}

\maketitle
\setcounter{tocdepth}{4}
\tableofcontents
\newpage

\section*{Introduction}
On s'intéresse aux problèmes de la forme : \begin{equation} \label{pblm} \tag{P}
\left\{ \begin{array}{c}
	Lu=-\sum_{i,j=1}^N a_{ij}\frac{\partial^2 u}{\partial x_i \partial x_j} + \sum_{i=1}^N b_i \frac{\partial u}{\partial x_i} + cu = f \text{ sur } \Omega\subset \mathbb{R}^N \text{ borné ouvert}\\
	u=g \text{ sur } \partial\Omega
\end{array}\right.
\end{equation}

\Def{Hölderienne}{$f$ hölderienne d'exposant $\alpha$ si : \[\exists c>0; \forall x,y, |f(x)-f(y)|\leq c|x-y|^\alpha, 0<\alpha<1\]}

\Theo{Unicité et existence}{Soit $\partial\Omega$ de classe $\mathcal{C}^1$, L uniformément elliptique : \[\exists \alpha>0; \forall x\in\overline{\Omega}, \forall \xi\in\mathbb{R}^N, \sum_{i,j=1}^N a_{ij}(x) \xi_i \xi_j \geq \alpha |\xi|^2\]
On suppose $a_{ij}, b_i, c\in\mathcal{C}^{0,\alpha}(\Omega)$ (continue et hölderienne), $\alpha\in]0,1[, c\geq 0$.\\
$f\in\mathcal{C}^{0,\alpha}(\overline{\Omega}), g\in\mathcal{C}^0(\partial \Omega)$.\\
Alors $\exists ! u$ solution de (\ref{pblm}) tel que $u\in\mathcal{C}^{2,\alpha}(\Omega)\cap\mathcal{C}^0(\overline{\Omega})$. }

\Theo{estimation de Schender}{Si de plus, $\partial \Omega$ de classe $\mathcal{C}^{2,\alpha}$, $g\in\mathcal{C}^{2,\alpha}(\partial \Omega)$, alors $u\in\mathcal{C}^{2,\alpha}(\overline{\Omega})$ et on a : 
\[\|u\|_{\mathcal{C}^{2,\alpha}(\overline{\Omega})}\leq c\left( \|f\|_{\mathcal{C}^{0,\alpha}(\overline{\Omega})} + \|g\|_{\mathcal{C}^{2,\alpha}(\partial\Omega)} \right)\]}

\part{Rappels divers}
\section{Les espaces $L^p$}
\subsection{Rappels d'analyse fonctionnelle}
\Def{Dual}{Soit $X$ un evn. On appelle dual de $X$ l'espace \[X'=\mathcal{L}(X,\mathbb{R})\]
Si $\phi\in X'$ et $x\in X$, on note souvent : \[\phi(x)=\langle \phi, x\rangle_{X'X}\]
appelé crochet de dualité.}

\Def{Bidual}{Soit $X$ un evn. On appelle bidual de $X$ l'espace \[X''=(X')'\] qui est un Banach.}

\textbf{Remarque : } On peut identifier $X$ avec un sous-espace de $X''$ à travers une isométrie, de la manière suiva,te : $\forall x\in X$, on définit : \[f_x : x'\in X' \mapsto \langle x',x\rangle_{X'X}\in\mathbb{R}\]
$f_x$ est dans $X''$ car linéaire, et $|\langle x',x\rangle|\leq \|x\|_X \|x'\|_{X'}$ donc $f_x$ est borné.

On peut montrer que : \[\mathcal{F} : x\in X \mapsto f_x\in X''\] est une isométrie, ie $\|x\|_X=\|f_x\|_{X''},\ \forall x\in X$. Donc on identifie $x$ avec $f_x$ et on écrit $X\subset X''$.\\
Question : a-t-on $X=X''$ ? autrement dit, $\mathcal{F}$ est-elle surjective ? En général, non.

\Def{Reflexif}{Si $\mathcal{F}$ est surjective, on dit que $C$ est reflexif.}

\Theo{représentation de Riesz-Fréchet}{Soit $H$ de Hilbert. \[\forall F\in H', \exists ! \tau(F)\in H; \forall x\in H, \langle F,x\rangle_{H'H}=(\tau(F), x)_H\]
De plus, l'application \begin{eqnarray*} \Phi : H'&\to&H \\ F&\mapsto& \tau(F) \end{eqnarray*} est une isométrie.}

\subsection{Les espaces $L^p$}
Dans la suite, $O$ est un ouvert de $\mathbb{R}^N$, $N\geq 2$\\
$\Omega$ est un ouvert borné de $\mathbb{R}^N$\\
$dx$ la mesure de Lebesgue\\

\Def{}{Soit $1\leq p< +\infty$.
\[L^p(O)=\{f:O\to\mathbb{R} \text{ mesurable }; \int|f|^p dx<\infty\}\]
\[L^p(O)=\{f:O\to\mathbb{R} \text{ mesurable }; |f|<\infty \text{ p.p. dans } O\}\]
\[\forall 1\leq p\leq+\infty, L^p_{loc}(O)=\{f\in L^p(\omega), \forall \omega \text{ ouvert borné}, \bar{\omega}\subset O\}\]}

\Prop{}{$L^p(O)$ est de Banach muni de la norme : \[\|f\|_{L^p(O)}=\left| \begin{array}{r c l} 
	\left(\int_O |f|^p dx \right)^{\frac{1}{p}} &\text{ si }& p<\infty\\
	\inf\{C; |f|\leq C \text{ pp}\} &\text{ si }& p=\infty
\end{array}\right.\]}

\Rem{}{Si $p=2$, $L^2(O)$ est un Hilbert par rapport au produit scalaire \[(f,g)_{L^2(O)}=\int_O f(x)g(x)dx\]}

\Prop{inégalité de Holder}{Soit $1\leq p\leq +\infty$. On pose 
\[p'=\left| \begin{array}{c c c}
	\frac{p}{p-1} &\text{ si }& 1<p<+\infty\\
	1 &\text{ si }& p=+\infty\\
	+\infty &\text{ si }& p=1
\end{array}\right.\]
appelé le conjugué.\\
\[\forall f\in L^p(O), \forall g\in L^{p'}(O), \int_O |f(x)g(x)|dx \leq \|f\|_{L^p(O)} \|g\|_{L^{p'}(O)}\]}

\Coro{}{$1\leq p\leq +\infty$, $p'$ son conjugué.\\
Si $f_n\to f$ dans $L^p(O)$ et $g\in L^{p'}(O)$ alors : \[\lim_{n\to +\infty} \int_O f_n g dx = \int_O fg dx\]}

\Coro{}{$1\leq p < q \leq +\infty$, $\Omega$ ouvert borné de $\mathbb{R}^N$. Alors $L^q(\Omega)\subset L^p(\Omega)$ et $\|f\|_{L^p(\Omega)}\leq c \|f\|_{L^q(\Omega)}$ où $c=c(|\Omega|, p, q)$.}

\Lem{inégalité de Young}{Soient $a,b\geq 0$ et $1<p<+\infty$. Alors \[ab\leq \frac{1}{p}a^p + \frac{1}{p'}b^{p'}\] avec $p'$ le conjugué de $p$.}

\Theo{inégalité d'interpolation}{Soit $1\leq p\leq r<+\infty$.\\
Si $f\in L^p(O)\cap L^r(O)$ alors $f\in L^q(O)$, $\forall p\leq q\leq r$.\\
De plus, \[\|f\|_{L^q(O)}\leq \|f\|^{\alpha}_{L^p(O)}\|f\|^{1-\alpha}_{L^r(O)}\]avec $\alpha\in[0,1]$ tel que $\frac{\alpha}{p} + \frac{1-\alpha}{r} = \frac{1}{q}$}

\subsection{2 rappels de mesure}
\Lem{de Fatou}{Soit $\{f_n\}\subset L^1(O)$ positives bornées dans $L^1(O)$. On pose \[f(x)=\liminf_{n\to +\infty} f_n(x) \text{ p.p. dans } O\]
Alors $f\in L^1(O)$ et \[\|f\|_{L^1(O)} \leq \liminf_{n\to +\infty} \|f_n\|_{L^1(O)}\]}

\Theo{convergence dominée de Lebesgue}{$\{f_n\}\subset L^1(O)$ telle que : 
\begin{enumerate}
	\item $f_n \to f$ presque partout dans $O$
	\item $\exists h\in L^1(O)$ telle que $|f_n(x)|\leq h(x)$ presque partout dans $O$, $\forall n\in \mathbb{N}$.
\end{enumerate}
alors $f_n \xrightarrow{L^1(O)} f$.}

\Prop{}{$1\leq p \leq +\infty$ tel que $f_n \xrightarrow{L^p} f$.\\
Alors $\exists \{f_{n_k}\}$ une sous-suite telle que $f_{n_k}\to f$ presque partout dans $O$.}

\subsection{Supportabilité}
\Def{Séparable}{Soit $B$ un espace de Banach.\\
$B$ est dit séparable s'il existse $A\subset B$ avec $A$ au plus dénombrable tel que $\overline{A}=B$.}

\Prop{}{$L^p(O)$ est séparable si $1\leq p < +\infty$.}

\subsection{Caractérisation du dual}
\Theo{représentation de Green}{$1\leq p<+\infty$, $p'$ son conjugué.\\
Si $f\in \left(L^p(O) \right)'$, alors $\exists ! g_f\in L^p(O)$ tel que \[\forall v\in L^{p'}(O), \langle f,v\rangle_{(L^p(O))'L^p(O)} = \int_O g_f(x) v(x) dx\]
De plus, \[\begin{array}{c c c c} \Phi : & \left(L^p(O)\right)' &\to& L^p(O) \\ &f&\mapsto& g_f \end{array}\] est une isométrie.}

\textbf{Remarque : } On peut donc identifier $f$ avec $g_f$.\\
De plus, $\Phi$ est surjective. On identifie donc $(L^p)'$ avec $L^{p'}$ si $1\leq p\leq +\infty$.
\begin{itemize}
	\item $1<p<+\infty$, $(L^p)'=L^{p'}$
	\item $p=1$, $(L^1)'=L^{\infty}$
	\item $p=+\infty$, $L^1 \subset (L^{\infty})'$
\end{itemize}
Ceci implique en particulier que $L^p(O)$ reflexif si $1<p<+\infty$. Mais $L^1$ et $L^{\infty}$ non reflexifs.

\section{Densité dans $L^p$}
\subsection{Notion de support}
\Def{}{$\phi : O \to \mathbb{R}$ continue. \[supp(\phi)=\{x\in O ; \phi(x)\neq 0\}\] (fermé de $O$)}

\Def{}{\[\mathcal{D}(O)=\{v:O\to \mathbb{R}; v\in\mathcal{C}^{\infty}(O) \text{ et } supp(v) \text{ est un compact de } \mathbb{R}^n \text{ contenu dans } O\}\]
\[\mathcal{C}^0_C(O)=\{v:O\to \mathbb{R}; v\in\mathcal{C}^0(O) \text{ et } supp(v) \text{ est un compact de } \mathbb{R}^n \text{ contenu dans } O\}\]}

\Prop{}{$1\leq p\leq +\infty$, $f\in L^p(O)$.\\
On pose \[\mathcal{A}=\{A \text{ ouvert de } O; f=0 \text{ p.p. dans } A\}\]
Alors si $w=\bigcup_{A\in\mathcal{A}} A$, on a $f=0$ p.p. dans $A$.}

\Def{}{On pose alors $supp(f)=O\backslash w$.}

\Def{}{\[L^p_c(O)=\{f\in L^p(O); supp(f) \text{ est un compact de } \mathbb{R}^n \text{ inclu de } O\}\]}

\subsection{Convolution}
\Def{}{$1\leq p\leq +\infty$, $f\in L^1(\mathbb{R}^N)$, $g\in L^p(\mathbb{R}^n)$. 
On définit le produit de convolution par : \[\forall x\in \mathbb{R}^n, (f*g)(x)=\in_{\mathbb{R}}f(x-y)g(y)dy \text{ p.p.}\]}

\Prop{}{\begin{enumerate}
	\item $f\in L^1(\mathbb{R}^N)$, $g\in L^p(\mathbb{R}^N)$. \\
$f*g$ est bien définie et $f*g\in L^p(\mathbb{R}^N)$, et : \[\|f*g\|_{L^p(\mathbb{R}^n)} \leq \|f\|_{L^1(\mathbb{R}^N)} \|g\|_{L^p(\mathbb{R}^N)}\]
	\item $f,g\in L^1(\mathbb{R}^N)$, $f*g=g*f$
	\item Si $f\in\mathcal{D}(\mathbb{R}^N)$, $g\in L^p(\mathbb{R}^N)$, alors $f*g\in \mathcal{C}^{\infty}(\mathbb{R}^N)$ (mais pas nécessairement à support compact).
		\[\frac{\partial}{\partial x_i} (f*g) = \frac{\partial f}{\partial x_i} * g\]
Si de plus, $g\in L^p_c(\mathbb{R}^N)$, alors $f*g\in\mathcal{D}(\mathbb{R}^N)$ et $supp(f*g)\subset supp(f) + supp(g)$.
\end{enumerate}}

\subsubsection{Suites régularisantes}
\Def{}{$B(0,1)\subset \mathbb{R}^N$. Soit $\rho \in \mathcal{D}(\mathbb{R}^N)$, $\rho\geq 0$, $\|\rho\|_{L^1(\mathbb{R}^N)}=1$, $supp(\rho)\subset \overline{B(0,1)}$.\\
$\forall n\in\mathbb{N}$, on pose $\rho_n(x)=n^N \rho(nx)$, $\forall x\in\mathbb{R}^N$. $\{\rho_n\}_n$ s'appelle une suite régularisante.}

\Theo{}{$1\leq p<+\infty$, $f\in L^p(\mathbb{R}^N)$. $\forall \{\rho\}_n$ suite régularisante : \[\underbrace{\rho_n * f}_{\in\mathcal{C}^{\infty}(\mathbb{R}^N)} \to f \text{ dans } L^p(\mathbb{R}^N)\]}

\Theo{}{$\mathcal{D}(\mathbb{R}^N)$ est dense dans $L^p(\mathbb{R}^N)$, $\forall 1\leq p<+\infty$. (Faux pour $L^{\infty}$ !)}

\Lem{de Urysohn}{$O$ ouvert de $\mathbb{R}^N$, $K$ compact de $\mathbb{R}^N$, $K\subset O$.\\
Alors $\exists \psi\in\mathcal{D}(0)$ telle que $\psi\equiv 1$ sur $K$ et $0\leq \psi<1$.}

\Coro{}{$\forall O\subset \mathbb{R}^N$, $\exists \{\psi_n\}\subset \mathcal{D}(O)$ tel que \[\forall n\in\mathbb{N}, 0\leq \psi_n\leq 1, \psi_n \to 1 \text{ p.p. dans } O\]}

\Theo{}{$1\leq p<\infty$.\\
Soit $v\in L^p(\mathbb{R}^N)$. On prolonge $v$ par zéro : \[\tilde{v}=\left\{ \begin{array}{c c c}
v &\text{ dans }& O \\
0 \text{ sinon}&
\end{array}\right.\]
Donc $\tilde{v}\in L^p(\mathbb{R}^N)$}

\Theo{}{$f\in L^1_{loc}(O)$ tel que \[\int_O f(x)\phi(x) dx = 0\ \forall \phi\in\mathcal{D}(O)\]
alors $f=0$ presque partout dans $O$.}

\section{Distributions}
\Def{Convergence des suites dans $\mathcal{D}(O)$}
{$\{\phi_n\}\subset \mathcal{D}(O)$, $\phi\in\mathcal{D}(O)$\\
$\phi_n\to \phi$ dans $\mathcal{D}(O)$ si : \begin{enumerate}
	\item $\exists K$ compact, $K\subset O$; \[\forall n, supp(\phi_n)\subset K\] \[supp(\phi)\subset K\]
	\item $\forall \alpha=(\alpha_1,...,\alpha_n)\in\mathbb{N}^n$, $\partial^{\alpha} \phi_n \to \partial^{\alpha} \phi$ uniformément dans $K$
\end{enumerate}}

\textbf{Remarque :} $\mathcal{D}(O)$ n'est pas métrisable, cela ne définit pas une topologie mais on peut en définir une telle que la convergence des suites dans cette topologie soit celle-ci.\\

\Def{}{Une application $T:\mathcal{D}(O)\to\mathbb{R}$ est une distribution si : \begin{enumerate}
	\item $T$ linéaire
	\item Si $\phi_n\to\phi$ dans $\mathcal{D}(O)$, alors $T(\phi_n)\to T(\phi)$
\end{enumerate}
L'ensemble des distributions sur $O$ est noté $\mathcal{D}'(O)$. On notera : \[\langle T,\phi\rangle_{\mathcal{D}'(O)\mathcal{D}(O)}=T(\phi)\]}

\textbf{Remarque :} L'application $\Phi : f\in L^1_{loc}(O) \to T_f\in\mathcal{D}'(O)$ est injective et linéaire car si $T_f(\phi)=O \forall \phi\in\mathcal{D}(O)$ alors $f=0$\\
Donc on identifie $f$ et $T_f$ et on écrit : \[L^1_{loc}(O)\subset \mathcal{D}'(O)\]

\Def{Distribution régulière}{$T\in\mathcal{D}'(O)$ est une régulière si : \[\exists f\in L^1_{loc}(O); T=T_f\]}

\textbf{Remarque :} On peut montrer qu'il existe des distributions non régulières.

\Def{Dérivée d'une distribution}{Soit $T\in\mathcal{D}'(O)$. On appelle dérivée de $T$ (au sens des distributions) par rapport à la ième variable et on la note $\frac{\partial T}{\partial x_i}$ la distribution définie par : \[\forall \phi\in\mathcal{D}(O), \langle \frac{\partial T}{\partial x_i} \rangle_{\mathcal{D}'(O)\mathcal{D}(O)}=-\langle T, \frac{\partial \phi}{\partial x_i}\rangle_{\mathcal{D}'(O)\mathcal{D}(O)}\]}

%\section{Linéarisation}
\subsection{Linéarisation dans l'espace d'état}
On s'intéresse aux systèmes affines : \begin{equation} \tag{$\Sigma$} \label{systAff} \Sigma : \dot{x}=f(x)+\sum_{i=1}^m u_ig_i(x),\ x\in X\end{equation}
On prend $\phi : X\to \tilde{X}$ un difféomorphisme et on transforme \ref{systAff}.
\[\dot{\tilde{x}}(t)=\tilde{f}(\tilde{x})+\sum_{i=1}^m u_i \tilde{g}_i(\tilde{x})\]
avec \begin{equation} \label{diffSEqu} \left\{\begin{array}{c c c} 
\tilde{f}(\tilde{x})&=&\frac{\partial \phi}{\partial x}(\phi^{-1}(\tilde{x}))f(\phi^{-1}(\tilde{x}))\\ \\
\tilde{g}_i(\tilde{x})&=&\frac{\partial \phi}{\partial x}(\phi^{-1}(\tilde{x}))g_i(\phi^{-1}(\tilde{x}))\\
\end{array}\right.\end{equation}

\Def{S-équivalent}{$\Sigma$ et 
\begin{equation} \label{systAffTr} \tag{$\tilde{\Sigma}$} \tilde{\Sigma}:\dot{\tilde{x}}(t)=\tilde{f}(\tilde{x})+\sum_{i=1}^m u_i \tilde{g}_i(\tilde{x})\end{equation}
 sont dits équivalents dans l'espace d'état (state space equivalent, ou S-equivalent) si (\ref{diffSEqu}).}

\Def{localement S-équivalent}{\ref{systAff} et \ref{systAffTr} sont localement, en $x_0$ et $\tilde{x}_0$, S-équivalent, si $\exists V_{x_0}$ et $V_{\tilde{x}_0}$ et $\phi : V_{x_0}\to V_{\tilde{x}_0}$ un difféomorphisme tel que $\phi$ transforme $\restriction{\Sigma}{V_{x_0}}$ en $\restriction{\tilde{\Sigma}}{V_{\tilde{x}_0}}$}

\Def{S-linéarisable}{\ref{systAff} est S-linéarisable si $\exists \phi: X\to\mathbb{R}^n$ un difféormorphisme tel que $\Sigma$ et \[\Lambda : \dot{\tilde{x}}=A\tilde{x} + \sum_{i=1}^m a_i b_i\]
$A\in \mathcal{M}_{n\times n}(\mathbb{R})$, $b_i\in\mathbb{R}^n$, sont S-équivalent. }

\Def{}{\[ad_f^0g=g,\ ad_fg=[f,g],\ ad_f^ng=[f,ad_f^{n-1}g]\]}


\Theo{}{Supposons $f(x_0)=0$, ie $x_0$ un point d'équilibre.\\
\ref{systAff} est localement autour de $x_0$ et $0_{\mathbb{R}^n}$ S-linéarisable si et seulement si :
\begin{description}
	\item[(SL1) :] dim vect$\{ad_f^qg_i(x_0),\ 1\leq i\leq m,\ 0\leq q\leq n-1\}=n$
	\item[(SL2) :] $[ad_f^qg_i, ad_f^rg_j]=0$, $\forall 1\leq i,j\leq m$, $\forall 0\leq q\leq n-1$, $\forall 0\leq r\leq n$
\end{description}}

\Theo{}{\ref{systAff} est S-linéarisable si et seulement si :
\begin{description}
	\item[(SL1) :] dim vect$\{ad_f^qg_i(x),\ 1\leq i\leq m,\ 0\leq q\leq n-1\}=n$
	\item[(SL2) :] $[ad_f^qg_i, ad_f^rg_j]=0$, $\forall 1\leq i,j\leq m$, $\forall 0\leq q\leq n-1$, $\forall 0\leq r\leq n$
	\item[(SL3) :] les champs de vecteurs $f, g_1,...,g_m$ sont complets, ie les flots $\gamma^f_t(p)$, $\gamma^{g_i}_t(p)$ existent $\forall p\in\mathbb{R}^n$, $\forall t\in\mathbb{R}$ $\Leftrightarrow$ $ad_f^qg_i$ complets, $0\leq q\leq n-1$, $0\leq i\leq m$
\end{description}}

\subsection{Linéarisation par bouclage}
\Def{F-équivalence}{\[\Sigma : \dot{x}=f(x)+g(x)u \text{ et } \tilde{\Sigma} : \dot{\tilde{x}}=\tilde{f}(\tilde{x})+\tilde{g}(\tilde{x})\tilde{u}\] sont dits équivalents par bouclage (F-equivalent) si $\exists \psi:X\to\tilde{X}$ un difféomorphisme et $\alpha=(\alpha_1,...,\alpha_m)^T$ et $\beta=(\beta_{ij})_{1\leq i,j\leq m}$ tel que :$\alpha_i, \beta_{ij}\in\mathcal{C}^{\infty}(X)$, $\beta(x)$ inversible, et : \begin{eqnarray*}
\phi_*(f+g\alpha)&=&\tilde{f}\\
\phi_*(g\beta)&=&\tilde{g}
\end{eqnarray*}}

\Def{}{On note : \[\mathcal{D}^j=span\{ad_f^qg_i,\ 1\leq i\leq m,\ 1\leq q\leq j-1\}\]}

\Theo{Jakuleczyk-Respondek}{Supposons $f(x_0)=0$, ie $x_0$ un point d'équilibre.\\
\ref{systAff} est localement autour de $x_0$ F-équivalent à \[\Lambda : \dot{\tilde{x}}=A\tilde{x}+\sum_{i=1}^m \tilde{u_i} b_i\]
contrôlable, si et seulement si : 
\begin{description}
	\item[(FL1) :] rg $\mathcal{D}^n(x_0)=n$ $\Leftrightarrow$ (SL1)
	\item[(FL2) :] rg $\mathcal{D}^j(x)=cste$, $\forall j=1,...,n$
	\item[(FL3) :] $\mathcal{D}^j$ involutive, $\forall j=1,...,n$
\end{description}}

\textbf{Remarque :} Pour $m=1$, on a équivalence avec :
\begin{description}
	\item[(FL1') :] $g$, $ad_fg$,...,$ad_f^{n-1}g$ indépendants en $x_0$
	\item[(FL2') :] $\mathcal{D}^{n-1}$ involutive.
\end{description}

\Def{Forme de Brunovsky}{On appelle forme de Brunovsky le système linéarisant par bouclage le système $\Sigma$ : 
\begin{eqnarray*} 
	\tilde{x}_1&=&h\\
	&\vdots&\\
	\tilde{x}_n&=&L_f^{n-1}h
\end{eqnarray*} 
où $h$ est la paramétrisation de la variété involutive de dimension $n-1$. Ce système vérifie : 
\begin{eqnarray*}
	\dot{\tilde{x}}_1&=&\tilde{x}_2\\
	&\vdots&\\
	\dot{\tilde{x}}_{n-1}&=&\tilde{x}_n\\
	\dot{\tilde{x}}_n&=&\tilde{u}
\end{eqnarray*} }

\subsection{Contrôlabilité}
On veut une trajectoire sur $[0,T]$ telle que $\tilde{x}_0$ soit relié par celle-ci à $\tilde{x}_T$.\\
Choisissons $\phi(t)$ tel que : \[\begin{array}{c c c c c c c}
\phi(0)&=&\tilde{x}_{0_1} & & \phi(T)&=&\tilde{x}_{T_0}\\
   &\vdots& &\text{ et }&  &\vdots&\\
\phi^{(n-1)}(0)&=&\tilde{x}_{0_n} & & \phi^{(n-1)}(T)&=&\tilde{x}_{T_n}
\end{array}\]
Alors $\tilde{u}(t)=\phi^{(n)}(t)$ résoud le problème.

\section{Observabilité}
\begin{equation}\label{probComp} \tag{$\Pi$}
\left\{ \begin{array}{c c c}
	\dot{x}&=&F(x,u)\\
	y&=&h(x)
\end{array} \right.\end{equation}
$y\in Y\subset\mathbb{R}^p$, $hX\to Y$, dim $Y=p<$dim $X=n$\\
$\forall 1\leq i\leq p, h_i\in \mathcal{C}^{\infty}(X)$.

\Def{Indistingables}{$x_0\in X$ et $\tilde{x}_0\in X$ sont dits indistingables si $\forall u(t)\in \mathcal{U}$, \[y(t,x_0,u)\equiv y(t,\tilde{x}_0,u)\]}

\Def{Observable}{\ref{probComp} est observable si : \[x_0 \text{ et } \tilde{x}_0 \text{ indistingables }\Rightarrow x_0=\tilde{x}_0\]}

\Def{Localement observable}{\ref{probComp} est localement observable autour de $p\in X$ si $\exists V_p\subset X$; $\restriction{\Pi}{V_p}$ est observable.}

\Def{Espace d'observation}{On définit $\mathcal{O}$, l'espace d'observation, comme étant le plus petit espace vectoriel sur $\mathbb{R}$ tel que :
\begin{enumerate}
	\item $h_i\in\mathcal{O}$, $1\leq i\leq p$
	\item Si $\phi\in\mathcal{O}$, alors $L_{F_u}\phi\in\mathcal{O}$, $\forall u\in \mathcal{U}$
\end{enumerate}
On a \[\mathcal{O}=vect\{L_{F_{u_1}}...L_{F_{u_k}}h_i, 1\leq i\leq p, u_j\in \mathcal{U}, 1\leq j\leq k, k\geq 0\}\]}

\Def{Codistribution}{On appelle codistibution : \[\mathcal{H}=span\{d\phi, \phi\in\mathcal{O}\}\]
(Notons que $d\phi\in\mathcal{M}_{1\times n}$).\\
En notant $T_p^*X=(T_pX)^*$ l'espace dual à l'espace tangent, appelé espace cotangent : \[\mathcal{H}:p\in X \mapsto \mathcal{H}(p)\subset T_p^*X\]}

\Theo{Hermann-Kremer}{Soit $p\in X$\\
Si dim $\mathcal{H}(p)=n$, alors \ref{probComp} est localement observable en $p$.}

\Theo{}{Supposons que dim$\mathcal{H}(p)=cste=k$ $\forall p\in X$.
\begin{enumerate}
	\item \ref{probComp} est localement observable si et seulement si $k=n$
	\item La distribution 
\begin{eqnarray*}
	\mathcal{D}&=&span\{f\in V^{\infty}(X); \langle d\phi, f\rangle=0, \forall \phi\in \mathcal{O}\}\\
		&=&span\{f\in V^{\infty}(X); \langle \mathcal{H}, f\rangle=0\}\\
\end{eqnarray*}
	est involutive.
	\item Autour de chaque $p\in X$, $\exists z^1=(z_1^1,...,z_k^1)$ et $z^2=(z_{k+1}^2,...,z_n^2)$ tel que 
		\[\mathcal{D}=span\{\frac{\partial}{\partial z^2_{k+1}},...,\frac{\partial}{\partial z^2_n}\}\]
	De plus, en coordonnées $(z^1,z^2)$ :
		\[\begin{array}{c c c}
			\dot{z}^1&=&f^1(z^1,u)\\
			\dot{z}^2&=&f^2(z^1,z^2,u)\\
			y&=&h^1(z^1)
		\end{array}\]
	\item Dans $V_p$ (où les coordonnées $(z^1,z^2)$ sont définies), $z_0$ et $\tilde{z}_0$ sont indistingables si $z_0^1=\tilde{z}_0^1$
\end{enumerate}}

\bigskip
On considère un système affine avec la même sortie (mais tout se généralise avec les systèmes non affines).\\
On prend juste le même nombre de sortie que de contrôle. $(y_i, 1\leq i\leq m)$
\Def{Découplable}{\ref{systAff} est découplable entrée-sortie (I-O-decouplable) si $\exists z=\phi(x)$, $u=\alpha(x)+\beta(x) v$ tel que :
	\[\dot{z}=\tilde{f}(z)+\sum_{i=1}^m v_i\tilde{g}_i(z)\]
$Z=Z_0\times...\times Z_m$, avec $z=(z^0,z^1,...,z^m)$
\[\begin{array}{c c c c c c c}
	\dot{z}^0&=&\tilde{f}^0(z)+\sum_{i=1}^m v^i \tilde{g}_i^0(z) \\
	\dot{z}^1&=&\tilde{f}^1(z^1)+\tilde{g}_1(z^1)v^1 & & y_1&=&h^1(z^1)\\
		&\vdots&\\
	\dot{z}^m&=&\tilde{f}^m(z^m)+\tilde{g}_m(z^m)v^m & & y_m&=&h^m(z^m)
\end{array}\]
où
	\[\tilde{f}=\phi_*(f+g\alpha)=\phi_*\left( g+\sum_{i=1}^m g_i\alpha_i\right)\]
	\[\tilde{g}=\phi_*(g\beta)\ \tilde{g}_i=\phi_*\left(\sum_{i=1}^m g_j\beta_{ji}\right)\]
}

\Def{}{Pour chaque $1\leq i\leq m$, soit $\rho_i$ le plus petit nombre tel que \[\frac{d^{\rho_i}y_i}{dt^{\rho_i}}\] dépend exclusivement de $u$, ie \[\exists j; L_{g_j}L_f^{\rho_i-1} y_i\neq 0\]}

\Theo{}{Supposons rg$D(x)=cst$, où \[D_{ij}=L_{g_j}L_f^{\rho_i-1}h_i\] appelée matice de découplage.\\
Le système \ref{systAff} est découplable en $x_0$ si et seulement si : \[rgD(x_0)=m\]
De plus, on pose $z_{ij}=L_f^{j-1}h_i, 1\leq i\leq m,\ 1\leq j\leq \rho_i$}


\end{document}
