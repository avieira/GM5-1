\documentclass{article}
\input{../../preambule}

\title{English : essay\\What could threaten your private life ?}
\author{Alexandre Vieira}
\date{21/09/14}

\hypersetup{colorlinks=true, urlcolor=bleu, linkcolor=red}

%Def = Definition
%Theo = Théorème
%Prop = Propriété
%Coro = Corollaire
%Lem = Lemme

\makeatletter
\@addtoreset{section}{part}
\makeatother

\begin{document}

\maketitle

Most people I know want to protect their private life. It is actually something that I can easily understand (and a thought that I also share) : I don't really want people to know what I'm doing, what I prefer, or worse, what I'm thinking anytime - even though some people already do that, but that's fine, since I trust them and they learned to know me.\\
Many little things could easily pierce this thin barrier between your private and your public life. A first example, which is commonly used, is mobile phones, or smartphones as we hear nowadays.These small devices are actually pretty useful : wherever you are, you can call who you want anytime, you can have some information on the place where you are, find the perfect place to buy something in the area, or just to drink a tea or a coffee. You can also see timetables for a movie you would like to see tonight, or when your delivery is scheduled.
But it has of course some drawbacks too. If you can call anybody anytime, that means you can also be called anytime, even by people you would not like to hear about. If your phone can know where you are, that means other people could also know it, even if you don't want them to do so. Could you even say that you didn't know your delivery was coming, since you could have seen it on your phone ? \\
Another example is social networks. They go viral for a few years now with the explosion of the most commonly used networks, such as Facebook or Twitter. More and more people now have and account on one of these websites - some people even create an account for their pet - and spend hours and hours online just looking at posts of people telling their adventures to whom agree to read them. The trouble is : are these posts public or private ? Are these words still yours and do you still have all your rights on them, or do they now belong to the social network you used ? Of course you can set some boundaries : if you don't want anybody to read your messages, you can easilly do that and make your messages private. But nonetheless, all of these are stored somewhere on a server, and you don't have any way to have any control on these data. And this could be quite scarry.

\end{document}
