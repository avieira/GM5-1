\documentclass{article}
\input{../../preambule}

\title{Game makers eager to pick your brain with neurogaming :\\ Summary and opinion}
\author{Alexandre Vieira}
\date{11/10/14}

\hypersetup{colorlinks=true, urlcolor=bleu, linkcolor=red}

%Def = Definition
%Theo = Théorème
%Prop = Propriété
%Coro = Corollaire
%Lem = Lemme

\makeatletter
\@addtoreset{section}{part}
\makeatother

\begin{document}

\maketitle
The principle of neurogaming is simple : capture the signals emitted by someone's brain and translate it into instructions for a game. That means it does not rely on a mouse or a joystick whereas traditionnal games. Game makers seem to show a lot of interest in this new kind of gaming, many people talking about it and even many companies being really eager to develop further this new technology. There was even a neurogaming conference gathering about 550 company last May.\\
This technology uses a headset developed in California, measuring brainware frequencies. It brings forward a million ideas, even though only a few games have appeared so far. The actions possible seem limited for now, but it is hoped to be enhanced for more challenging games. In the future, we could imagine going even further, by including physiological factors like heart rate or hand gestures. The games could even change depending on how the players feels. Games could then have impact beyond entertainment ; people playing these games seem to be more focus, relaxed and they meditate better. Other companies hope to use this to treat some disease, like Alzheimer's disease or hyperactivity disorder.\\
And clients also seem to be interested, or at least find this intriguing : "that would be kind of cool" said a regular player.

\bigskip
For me, this article is kind of "narrow-minded". I had a project last year, concerning EEG and recognition of movements through these. It is actually a big area of research these days, and people are always searching for new ways to enhance this recognition. And when we were talking about why we were making it, what came mainly was oriented to correction of handicap. We could actually figure out a wide range of application for this : a wheelchair activated thanks to one's mind only could be a great advantage for tetraplegic people for instance. I feel like this article presents this technological progress just usefull for fun or psychological disease (in a minor part). From what I saw, it could have much more to offer, and it seems like it has been forgotten.\\
Besides, there is something I also thought : there is actually no danger in this. I don't mean this article intended to say that, but it is a question that could be raised. The finest methods used on EEG that I saw were based on calibration. In order to know what you are thinking, the device already have to know you, or sort of. And it would be able to recognise only precise signals you were asked before, and rather simple. When they become more complex, it is, in the state of the art, almost impossible to know what you wanted to do only through these signals. That means this technology is not able to know everything about you. It won't know when you lie, when you're scared or what you actually prefer, unless it has learned it from you. And in that case, it will only know how to recognise what he has learned and only with you. Each person has its own signals, its own "way to think". I clearly see no danger in this.\\
And finally, I also think this could be a good progress for video games. I used to play video games - quite a lot, actually. And even if a part of me wants to keep this old way of gaming - mainly through a joystick for me - I think this could be quite fun. But it won't replace the old way of gaming - at least not totally for me - and I think it won't replace the traditionnal way for most people in the next years. But maybe later !
\end{document}
