\documentclass{article}
\input{../../preambule}

\title{Ways to boost IQ : Summary and opinion}
\author{Alexandre Vieira}
\date{4/11/14}

\hypersetup{colorlinks=true, urlcolor=bleu, linkcolor=red}

%Def = Definition
%Theo = Théorème
%Prop = Propriété
%Coro = Corollaire
%Lem = Lemme

\makeatletter
\@addtoreset{section}{part}
\makeatother

\begin{document}

\maketitle
This article presents the leading edge technology used by U.S. Military in order to enhance intelligence. They do this because new technologies are growing dramatically, and they need people able to deal with these.\\
Some programs are funded by the Navy, the Army and the Air Force. One big exemple of these programs is the SHARP program, costing \$12 million. It is studying techniques based on meditation an electrical stimulation of the brain in order to make analysts more intelligent.\\
Other programs showed that transcranial stimulation also improves cognitive capacities by 200 percent. The effect seem however to be ephemeral. But some mental benefits might last longer, and lead to some training programs, as it already exists for physical training programs. It could also interest people outside the military : children, teens or even eldery people.\\
While these programs aims to strengthen the brain, other programs tend to prefer the company of computers to help people. It uses brain scanners, which could be seen as mind reading. The only goal of such technology is only to detect when people are tired, distracted or under stress.\\
But is it really working ? The history of IQ-raising interventions is already marked with disapointment. Recently published studies already show technical errors, and people are asking for a bigger study that is carefully designed.

\bigskip
For me, this article is actually well written : it only presents facts, and it does not want to argue for a side or for another one. It is even quite complete, presenting the pros in a large majority, but also the cons. And I just wanted to underline this point.\\
Now, on the subject on itself, I find this research rather ridiculous. Rather than funding schools or educational programs, they prefer searching ways to enhance intelligence through electrical impulses. And the results seem ridiculous : cognitive capacities are apparently enhanced by 200 percent : compared to what ? And how are these capacities mesured ? And on top of that, these results are only available for a few hours, when an actual traning program should last for way more than a few hours. \\
And there is also something this article never mentioned : what are the long term consequences ? Because I don't think electrical impulses given to the brain only have benefits. It seems to me like a burn on your brain. I think further research should be maid to seek for the potential drawbacks of these technologies.\\
Besides that, seeking ways in order to enhance attentiveness could be a good thing. For instance, the research made on meditation could also underline how our brain actually works and how it could lead to ways to teach more efficiently, for instance. Our brain still is something we don't fully understand, and some research on it are not necessarily wrong !

\end{document}
