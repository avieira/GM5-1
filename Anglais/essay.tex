\documentclass{article}
\input{../../preambule}

\title{English : essay\\What could threaten your private life ?}
\author{Alexandre Vieira}
\date{21/09/14}

\hypersetup{colorlinks=true, urlcolor=bleu, linkcolor=red}

%Def = Definition
%Theo = Théorème
%Prop = Propriété
%Coro = Corollaire
%Lem = Lemme

\makeatletter
\@addtoreset{section}{part}
\makeatother

\begin{document}

\maketitle

Most people I know want to protect their private life. It is actually something that I can easily understand (and a thought that I also share) : I don't really want people to know what I'm doing, what I prefer, or worse, what I'm thinking anytime - even though some people already do that, but that's fine, since I trust them and they learned to know me.\\
Many little things could easily pierce this thin barrier between your private and your public life. A first example, which is commonly used, is mobile phones, or smartphones as we hear nowadays.These small devices are actually pretty useful : wherever you are, you can call who you want anytime, you can have some information on the place where you are, find the perfect place to buy something in the area, or just to drink a tea or a coffee. You can also see timetables for a movie you would like to see tonight, or when your delivery is scheduled.
But it has of course some drawbacks too. If you can call anybody anytime, that means you can also be called anytime, even by people you would not like to hear about. If your phone can know where you are, that means other people could also know it, even if you don't want them to do so. Could you even say that you didn't know your delivery was coming, since you could have seen it on your phone ? \\
Another example is social networks. They go viral for a few years now with the explosion of the most commonly used networks, such as Facebook or Twitter. More and more people now have an account on one of these websites - some people even create an account for their pet - and spend hours and hours online just looking at posts of people telling their adventures to whom would agree to read them. The trouble is : are these posts public or private ? Are these words still yours and do you still have all your rights on them, or do they now belong to the social network you used ? Of course you can set some boundaries : if you don't want anybody to read your messages, you can easilly do that and make them private. But nonetheless, they are all stored somewhere on a server, and you don't have any way to have any control on these data. And this could be quite scary.

\bigskip
All these fears, are they all just a consequence of these technological progresses ? I think not.\\
This fear of being spied, watched or even monitored, is not new. For instance, intelligence exists for a very long time : the Iliad, attributed to Homer, mentions spying during antiquity. So why should it change now ? I think we are scared because it seems more close to us, more insidious than before, and on top of that, we don't feel like we could do anything about that without sacrifices. 
You don't want to be spied through your phone ? Okay, then do not buy a phone and forget about all the benefits of it. It is actually the same about social networks. The trouble is that people will often tend to change you on this, and it is not really enjoyable to fell like put aside of the society. We are not always aware of it, but we are full of reflexes, acquired through our life alongside others, even on our relation to technology. It is actually something common now, to ask someone to his or her phone number.\\
As a result, it seems like we are bound to let an access to our private life. Are we also bound to be monitored ? It is not necessary : we just have to know how to use these devices. For instance, on Facebook or Twitter, of you don't want people to know what you did what you exactely like, you just have to never mention it : these websites won't force you to tell it. Most mobile phones have options you can deactivate so you can enjoy the benefits you want and avoid some drawbacks.\\
To conlcude, we should not fear all progresses : all we have to do is to know how to control them.  
 
\end{document}
