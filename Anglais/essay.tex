\documentclass{article}
\input{../../preambule}

\title{Game makers eager to pick your brain with neurogaming}
\author{Alexandre Vieira}
\date{21/09/14}

\hypersetup{colorlinks=true, urlcolor=bleu, linkcolor=red}

%Def = Definition
%Theo = Théorème
%Prop = Propriété
%Coro = Corollaire
%Lem = Lemme

\makeatletter
\@addtoreset{section}{part}
\makeatother

\begin{document}

\maketitle
The principle of neurogaming is simple : capture the signals emitted by someone's brain and translate it into instructions for a game. That means it does not rely on a mouse or a joystick whereas traditionnal games. Game makers seem to show a lot of interest in this new kind of gaming, many people talking about it and even many companies being really eager to develop further this new technology. There was even a neurogaming conference gathering about 550 company last May.\\
This technology uses a headset developed in California, measuring brainware frequencies. It brings forward a million ideas, even though only a few games have appeared so far. The actions possible seem limited for now, but it is hoped to be enhanced for more challenging games. In the future, we could imagine going even further, by including physiological factors like heart rate or hand gestures. The games could even change depending on how the players feels. Games could then have impact beyond entertainment ; people playing these games seem to be more focus, relaxed and they meditate better. Other companies hope to use this to treat some disease, like Alzheimer's disease or hyperactivity disorder.\\
And clients also seem to be interested, or at least intriguing : "that would be kind of cool" said a regular player.

\end{document}
