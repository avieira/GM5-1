\documentclass[handout]{beamer}

\usepackage[frenchb]{babel}
\usepackage[T1]{fontenc}
\usepackage[utf8x]{inputenc}
 
\usetheme{Berkeley}
\usecolortheme{crane}
\useinnertheme{rounded}

\title[Class. Running Fronts]{Pr\'esentation d'un article : \\ Classification of existence and non-existence of running fronts in case of fast diffusion\\\small{Messoud Efendiev \& Johannes M\"uller}}
\author{Alexandre \bsc{Vieira}}
\institute{INSA de Rouen}
\date{\today}


\AtBeginSection[]
{
	\begin{frame}
		\frametitle{Sommaire}
		\tableofcontents[currentsection, hideothersubsections]
	\end{frame}
}

\begin{document}

\begin{frame}
\titlepage
\end{frame}

\begin{frame}
	\frametitle{Sommaire}
	\tableofcontents
\end{frame}

\section{Resultats}
\begin{frame}
	\frametitle{Équation étudiée}
	\begin{equation} \label{eqnG} u_t=(D(u)u_x)_x + f(u)\end{equation}
	\[D(u)=\frac{u^a}{(1-u)^b}\bar{D}(u)\in\mathcal{C}^2[0,1[\]
	\[f(u)=u(1-u)^\alpha \bar{f}(u)\in\mathcal{C}^2[0,1[\]
	\[a>1,\hspace{1em} b>0 \hspace{1em} \alpha\geq 0\]

\bigskip
	On cherche la solution $u(x,t)$ sous la forme \[u(x,t)=w(ct-x)\]
\end{frame}

\begin{frame}
	\frametitle{Théorème}
	\[u_t=(D(u)u_x)_x + f(u)\]

	\begin{block}{Théorème :}
	Si $\alpha-b\leq -1$, il n'y a aucun front.\\
	Si $\alpha-b>-1$, il existe une vitesse minimale $c^*$ telle que :
	\begin{itemize}
		\item Pour $c<c^*$, il n'y a pas de solution de la forme $u(x,t)=w(ct-x)$ non négative
		\item Pour $c=c^*$, un unique front d'onde solution existe, qui tend vers $0$ quand $x\to-\infty$
		\item Pour $c>c^*$, il y a une infinité de fronts d'ondes solutions, ordonnés. La solution minimal tend elle aussi vers 0 quand $x\to -\infty$, les autres sont strictement positives.
	\end{itemize}
	\end{block}
\end{frame}

\section{Preuve}
\subsection{Transformation du système}

\begin{frame}
	\frametitle{Transformation du système}
	Soit $u(x,t)=w(ct-x)$. En reprenant l'équation (\ref{eqnG}) et en y introduisant $w$, on obtient : \begin{equation} \label{eqn1} cw'=(D(w)w')'+f(w) \end{equation}
	On définit à présent $v$ tel que : \[v=\frac{D(w)w'}{w}\]
	En multipliant (\ref{eqn1}) par $\frac{D(w)}{w}$, on obtient : \[cv=v'D(w)+v\frac{w'D(w)}{w}+\frac{D(w)f(w)}{w}\]
\end{frame}

\begin{frame}
	\frametitle{Transformation du système}
D'où le système d'équation : \begin{equation} \label{eqn2} \left\{ \begin{array}{c c c}
	D(w)w' &=& vw \\
	D(w)v' &=& v(c-v)-\frac{D(w)f(w)}{w}
\end{array} \right. \end{equation}

En faisant le changement de variable suivant (rescaling time) : \[\frac{dt}{d\tau}=D(w(t(\tau)))\] \[\tilde{w}(\tau)=w(t(\tau))\] \[\tilde{v}(\tau)=v(t(\tau))\]
\end{frame}

\begin{frame}
	\frametitle{Transformation du système}
On obtient : 
\begin{equation} \left\{ \begin{array}{c c c}
	\tilde{w}' &=& \tilde{v}\tilde{w} \\
	\tilde{v}' &=& \tilde{v}(c-\tilde{v})-g(\tilde{w})
\end{array} \right. \end{equation}

où \[g(\tilde{w})=\frac{D(\tilde{w})f(\tilde{w})}{\tilde{w}}\]
\end{frame}

\begin{frame}
	\frametitle{Problème dans la transformation}
\[\frac{dt}{d\tau}=D(w)=\frac{w^a}{(1-w)^b}\bar{D}(w)\]
La transformation devient singulière pour $w\to 0$ et $w\to 1$

\bigskip
Pour $t_0\in \mathbb{R}\cup\{-\infty\}$, on doit avoir : \[0\leq w(t)\leq 1\] et \[\lim_{t\to t_0^+} w(t)=0, \hspace{2em} \lim_{t\to +\infty}w(t)=1\]
$\Rightarrow$ Problème !
\end{frame}

\subsection{Méthode du plan de phase}
\subsubsection{$w=0$ dans le système transformé}

\begin{frame}
	\frametitle{Analyse du linéarisé}
On prend $w=0$ $\Rightarrow$ Analyse peut être faite avec le temps rééchelonné, on utilise donc le système (\ref{eqn2}) : 
\[\left\{ \begin{array}{c c c}
	\tilde{w}' &=& \tilde{v}\tilde{w} \\
	\tilde{v}' &=& \tilde{v}(c-\tilde{v})-g(\tilde{w})
\end{array} \right.\]
On a deux points d'équilibre : \[(0,0) \text{ et } (0,c)\]
\end{frame}

\begin{frame}
	\frametitle{Analyse de (0,c)}
Jacobien du système en $(0,c)$ : \[\begin{pmatrix} c & 0 \\ -g'(0) & -c \end{pmatrix} = \begin{pmatrix} c & 0 \\ 0 & -c \end{pmatrix}\]
On peut vérifier que $g'(0)=0$.\\
\[\lambda_1=c \text{ et } \lambda_2=-c\]

Dans l'espace des phases : point selle. Axe $v$ invariant, stable. \\
On a approximativement : $\tilde{w}'\approx \lambda_1 \tilde{w} = c\tilde{w}$, d'où : 
\[w'\approx \frac{cw}{D(w)} \approx cw^{-(a-1)} \text{ pour } w \text{ petit}\]
Comme $a-1>0$, la trajectoire atteint le point stationnaire en un temps négatif fini.
\end{frame}

\begin{frame}
	\frametitle{Analyse de (0,0)}
Jacobien du système en $(0,0)$ : \[\begin{pmatrix} 0 & 0 \\ -g'(0) & c \end{pmatrix} = \begin{pmatrix} 0 & 0 \\ 0 & c \end{pmatrix}\]
\[\lambda_1=0 \text{ et } \lambda_2=c\]
On a directement que l'axe $v$ sera instable.\\
Deuxième direction : nécessite plus de discussion.
\end{frame}

\begin{frame}
	\frametitle{Variété centrale pour (0,0)}
\begin{block}{Propriété}
Le champ de vecteur étant $\mathcal{C}^2$ près de $(0,0)$, il existe une variété centrale tangente au vecteur propre $\begin{pmatrix} 0\\ 1 \end{pmatrix}$
\end{block}

Soit $\{v=h(w)\}$ une paramétrisation locale de cette variété. On a $h(0)=0$, $h'(0)=0$ et $v'=h'(w)w'$. En multipliant cette dernière égalité par $D(w)$ : 
\[v(c-v)-g(w)=h'(w)vw\]
\[h'(w)=\frac{c-h(w)}{w}-\frac{g(w)}{wh(w)}\]
\end{frame}

\begin{frame}
	\frametitle{Variété centrale pour (0,0)}
En écrivant $h(w)=c_1 w^\beta+o(w^{\beta+1})$, on obtient :
\[c_1\beta w^{\beta-1}=\frac{c}{w}-c_1w^{\beta-1}-\frac{\bar{g}(0)}{c_1}w^{a-\beta-1}+o(w^{\beta+1})\]
où $\bar{g}(0)=\bar{D}(0)\bar{f}(0)>0$.
\end{frame}
\end{document}
