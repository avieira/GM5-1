\section{Opérateurs compacts}
\subsection{Définitions}
Soient $E$ et $F$ deux espaces de Banach.\\
On désigne par $B_E$ la boule unité centrée à l'origine, ie \[B_E=\{x\in E;\ \|x\|\leq 1\}\]
et par $\mathscr{L}(E,F)$ l'espace des opérateurs linéaires continues de $E$ dans $F$ muni de la norme \[\forall f\in \mathscr{L}(E,F),\ \|f\|_{\mathscr{L}(E,F)}=\sup_{\|x\|\neq 0} \frac{\|f(x)\|_F}{\|x\|_E}\]

\Def{Opérateur compact}{On dit qu'un opérateur $T\in \mathscr{L}(E,F)$ est compact si l'image de la boule unité par $T$ est relativement compact pour la topologie forte, ie :
\[\overline{T\left(\{x\in E;\ \|x\|\leq 1\}\right)}\subset F \text{ compact}\]\\
On désigne par $\mathscr{H}(E,F)$ l'ensemble des opérateurs compacts de $E$ dans $F$, et $\mathscr{H}(E)=\mathscr{H}(E,E)$.}

\Theo{}{$\mathscr{H}(E,F)$ est un sous-espace vectoriel fermé de $\mathscr{L}(E,F)$ (pour la norme $\|\bullet\|_{\mathscr{L}(E,F)}$).}

\begin{dem}
Il est clair que la somme de deux opérateurs compacts est un opérateur compact.\\
Supposons que $(T_n)\subset\mathscr{H}(E,F)$, $T\in\mathscr{L}(E,F)$, et $\|T_n-T\|_{\mathscr{L}(E,F)}\to 0$. Montrons que $T\in\mathscr{H}(E,F)$. Comme $F$ est complet, il suffit de vérifier que pour tout $\varepsilon >0$, $T(B_E)$ peut être recouvert par un nombre fini de boules $B(f_i,\varepsilon)$ dans $F$. \\
Pour $n$ assez grand, on a $\|T_n-T\|_{\mathscr{L}(E,F)}<\frac{\varepsilon}{2}$. Comme $T_n(B_E)$ est relativement compact, on a pour $I$ fini \[T_n(B_E)\subset \bigcup_{i\in I} B\left( f_i,\frac{\varepsilon}{2}\right)\]
Donc par force, \[T(B_E)\subset \bigcup_{i\in I} B(f_i, \varepsilon)\]
\end{dem}

\Def{Rang fini}{On dit qu'un opérateur $T\in\mathscr{L}(E,F)$ est de rang fini si $R(T)<\infty$}
Il est clair qu'un opérateur continu de rang fini est compact (car les compacts dans un espace de dimension finie sont les sous-espaces fermés bornés).

\Coro{}{Soit $(T_n)$ une suite d'opérateurs de rangs finis de $E$ dans $F$ et soit $T\in\mathscr{L}(E,F)$ tels que $\|T_n-T\|_{\mathscr{L}(E,F)}\to 0$. Alors $T\in\mathscr{H}(E,F)$.}

\Propo{}{Soient $E$, $F$ et $G$ trois espaces de Banach. Si $T\in\mathscr{L}(E,F)$ et $S\in\mathscr{H}(F,G)$ (ou $T\in\mathscr{H}(E,F)$ et $S\in\mathscr{L}(F,G)$), alors $S\circ T\in\mathscr{H}(E,G)$. }

\Theo{Schauder}{Si $T\in\mathscr{H}(E,F)$, alors $T^*\in\mathscr{H}(F',E')$, et réciproquement.}

\begin{dem}
On aura pour cela besoin du théorème d'Ascoli :\\
\textbf{Théorème :} Soit $K$ un espace métrique compact et soit $\mathcal{H}$ un sous-ensemble borné de $\mathcal{C}(K)$, l'ensemble des fonctions continues sur $K$.\\
On suppose que $\mathcal{H}$ est uniformément équicontinu, ie : \[\forall \varepsilon>0, \exists\delta>0;\ d(x_1,x_2)<\delta\Rightarrow |f(x_1)-f(x_2)|<\varepsilon\ \forall f\in\mathcal{H}\]
Alors $\mathcal{H}$ est relativement compact dans $\mathcal{C}(K)$.

\bigskip
Montrons que $T^*(B_{F'})$ est relativement compact dans $E'$. Soit $(v_n)$ une suite de $B_{F'}$. Montrons que l'on peut extraire une sous-suite telle que $T^*(v_{n_k})$ converge. Soit $K=\overline{T(B_E)}$ (métrique compact) et soit $\mathcal{H}\subset\mathcal{C}(K)$ défini par : 
\[\mathcal{H}=\{\phi_n:x\in K\to\langle v_n, x\rangle;\ n=1,2,...\}\]
Par le théorème d'Ascoli, on peut extraire une sous-suite notée $\phi_{n_k}$ qui converge dans $\mathcal{C}(K)$ vers une fonction $\phi\in\mathcal{C}(K)$. En particulier : 
	\[\sup_{u\in B_E}|\langle v_{n_k},Tu\rangle - \phi(Tu)|\xrightarrow[k\to+\infty]{} 0\]
Donc
	\[\sup_{u\in B_E}|\langle v_{n_k},Tu\rangle - \langle v_{n_l},Tu\rangle|\xrightarrow[k,l\to+\infty]{} 0\]
ie
	\[\|T^* v_{n_k}-T^* v_{n_l}\|_{E'}\xrightarrow[k,l\to +\infty]{} 0\]
Par conséquent, $T^*v_{n_k}$ converge dans $E'$.

\bigskip
\textit{Réciproquement}, supposons que $T^*\in\mathscr{H}(F',E')$. D'après ce qui précède, $T^{**}\in\mathscr{H}(E'',F'')$ et en particulier, $T^{**}(B_E)$ est relativement compact dans $F''$. Or, $T(B_E)=T^{**}(B_E)$ et $F$ fermé dans $F''$. Par conséquent, $T(B_E)$ est relativement compact dans $F$.
\end{dem}

\subsection{Théorie de Riesz-Fredholm}
\Lem{de Riesz}{Soit $E$ un e.v.n. et soit $M\subset E$ un sous-espace fermé tel que $M\neq E$. Alors 
\[\forall \varepsilon>0\ \exists u\in E;\ \|u\|=1 \text{ et } d(u,M)\ge 1-\varepsilon\]}

\begin{dem}
Soit $v\in E\backslash M$. Comme $M$ est fermé, alors $d=d(v,M)>0$. On choisit $m_0\in M$ tel que \[d\leq \|v-m_0\|\leq \frac{d}{1-\varepsilon}\]
Alors \[u=\frac{v-m_0}{\|v-m_0\|}\]
répond à la question. En effet, si $m\in M$, on a : \[\|u-m\|=\left\|\frac{v-m_0}{\|v-m_0\|} -m\right\|\geq 1-\varepsilon\]
puisque \[m_0+\|v-m_0\|m\in M\]
\end{dem}

\Theo{Riesz}{Soit $E$ un e.v.n. tel que $B_E$ soit compact. Alors $E$ est de dimension finie.}

\begin{dem}
Raisonnons par l'absurde. Si $E$ est de dimension infinie, il existe une suite $(E_n)$ de sous-espaces de dimension finie tels que $E_{n-1}\subsetneq E_n$. Grâce au lemme ptécédent, on peut construire une suite $(u_n)$ avec $u_n\in E_n$, $\|u_n\|=1$ et $d(u_n,E_{n-1})\geq \frac{1}{2}$. En particulier, $\|u_n-u_m\|\geq\frac{1}{2}$ pour $m<n$. Donc la suite $(u_n)$ n'admet aucune sous-suite convergente - ce qui est contraire à l'hypothèse $B_E$ compact.
\end{dem}

\Theo{Alternative de Fredholm}{Soit $T\in\mathscr{H}(E)$. Alors : \begin{enumerate}
	\item $N(I-T)$ est de dimension finie
	\item $R(I-T)$ est fermé, et plus précisément \[R(I-T)=N(I-T^*)^\perp\]
	\item $N(I-T)=\{0\}\Leftrightarrow R(I-T)=E$
	\item dim $N(I-T)=$dim $N(I-T^*)$
\end{enumerate}}

\begin{dem}
\begin{enumerate}
	\item Soit $E_1=N(I-T)$. Alors $B_{E_1}\subset T(B_E)$ et donc $B_{E_1}$ est compact. D'après le théorème de Riesz précédent, $E_1$ est de dimension finie.
	\item Soit $f_n=u_n-Tu_n\to f$. Il faut montrer que $f\in R(I-T)$. Posons $d_n=d(u_n,N(I-T))$. Comme $N(I-T)$ est de dimension finie, il existe $(v_n)\subset N(I-T)$ tel que $d_n=\|u_n-v_n\|$. On a : \begin{equation} \label{fn} f_n=(u_n-v_n)-T(u_n-v_n) \end{equation}
Montrons que $\|u_n-v_n\|$ reste borné. Raisonnons par l'absurde et supposons qu'il existe une sous-suite telle que $\|u_{n_k}-v_{n_k}\|\to\infty$. En posant \[w_n=\frac{u_n-v_n}{\|u_n-v_n\|}\] on aurait grâce à (\ref{fn}) $w_{n_k}-T(w_{n_k})\to 0$. En extrayant une sous-sous-suite (encore notée $(w_{n_k})$ pour simplifier) on peut supposer que $Tw_{n_k}\to z$. Donc $w_{n_k}\to z$ et $z\in N(I-T)$. D'autre part :
	\[d(w_n,N(I-T))=\frac{d(u_n,N(I-T))}{\|u_n-v_n\|}=1\]
puisque $v_n\in N(I-T)$. À la limite on obtient $d(z,N(I-T))=1$ - ce qui est absurde, vu que $z\in N(I-T)$.\\
Par conséquent, $\|u_n-v_n\|$ reste borné et comme $T$ est compact, on peut extraire une sous-suite telle que $T(u_{n_k}-v_{n_k})\to l$.\\
On déduit de (\ref{fn}) que $u_{n_k}-v_{n_k}\to f+l$; posant $g=f+l$, on a $g-Tg=f$, ie $f\in R(I-T)$. On a donc montré que l'opérateur $I-T$ est à image fermée. On peut alors appliquer un théorème précédent sur la fermeture de l'ensemble image, et en conclure : 
	\[R(I-T)=N(I-T^*)^\perp \text{ et } R(I-T^*)=N(I-T)^\perp\]

	\item Prouvons d'abord l'implication $\Rightarrow$.\\
Raisonnons par l'absurde et supposons que \[E_1=R(I-T)\neq E\]
$E_1$ est un espace de Banach et $T(E_1)\subset E_1$. Donc $\restriction{T}{E_1}\in\mathscr{H}(E_1)$ et $E_2=(I-T)(E_1)$ est un sous-espace fermé de $E_1$. De plus, $E_2\neq E_1$ (puisque (I-T) injectif). En posant $E_n=(I-T)^n(E)$, on obtient ainsi une suite strictement décroissant de sous-espaces fermés. D'après le lemme de Riesz, il existe une suite $(u_n)$ telle que $u_n\in E_n$, $\|u_n\|=1$ et $d(u_n,E_{n+1})\geq \frac{1}{2}$. On a : \[Tu_n-Tu_m=-(u_n-Tu_n)+(u_m-Tu_m)+(u_n-u_m)\]
Notons que si $n>m$, $E_{n+1}\subset E_n\subset E_{m+1}\subset E_m$ et par conséquent : \[-(u_n-Tu_n)+(u_m-Tu_m)+u_n\in E_{m+1}\]
Donc $\|Tu_n-Tu_m\|\geq\frac{1}{2}$ - ce qui est absurde puisque $T$ est compact. Donc $R(I-T)=E$.

\bigskip
\textit{Inversement}, supposons que $R(I-T)=E$. Alors par corollaire précédent, $N(I-T^*)=R(I-T)^\perp=\{0\}$. Puisque $T^*\in\mathscr{H}(E')$, on peut appliquer ce qui précède à $T^*$ et conclure que $R(I-T^*)=E'$. Or, par le même corollaire, $N(I-T)=R(I-T^*)^\perp=\{0\}$.

	\item Soit $d=$dim $N(I-T)$ et $d^*=$dim $N(I-T^*)$. On va d'abord montrer que $d^*\leq d$. Raisons par l'absurde et supposons que $d<d^*$. Comme $N(I-T)$ est de dimension finie, il admet un supplémentaire topologique dans $E$; il exuste donc un projecteur continue $P$ de $E$ sur $N(I-T)$.\\
D'autre part, $R(I-T)=N(I-T)^\perp$ est de codomension finie $d^*$ et par conséquent, $R(I-T)$ admet dans $E$ un supplémentaire topologique, noté $F$ de dimension $d^*$. Comme $d<d^*$, il existe une application linéaire $\Lambda:N(I-T)\to F$ qui est injective et non surjective. POsons $S=T+(\Lambda\circ P)$; alors $R\in\mathscr{H}(E)$ puisque $\Lambda\circ P$ est de rang fini.\\
Montrons que $N(I-S)=\{0\}$. En effet, si \[0=u-Su=(u-Tu)-(\Lambda\circ Pu)\]
alors \[u-Tu=0 \text{ et } \Lambda\circ Pu=0\]
ie $u\in N(I-T)$ et $\Lambda u=0$, donc $u=0$\\
En appliquant (3) à l'opérateur $S$, on voit que $R(I-S)=E$. Ceci est absurde puisqu'uk existe $f\in F$, $f\not\in R(\Lambda)$; l'équation $u-Su=f$ n'admet pas de solution.\\
Par conséquent, on a prouvé que $d^*\leq d$. En appliquant ce résultat à $T^*$, on voit que \[\text{dim }N(I-T^{**})\leq \text{ dim }N(I-T^*)\leq \text{ dim }N(I-T)\]
Or, $N(I-T^{**})\supset N(I-T)$ - ce qui permet de conclure que $d=d^*$.
\end{enumerate}
\end{dem}

\Rem{}{\begin{enumerate}
	\item L'Alternative de Fredholm concerne la résulution de l'équation $u-Tu=f$. Elle exprime que : \begin{itemize}
		\item \textbf{Ou bien} pour tout $f\in E$, l'équation $u-Tu=f$ admet une solution unique
		\item \textbf{Ou bien} l'équation homogène $u-Tu=0$ admet $n$ solutions linéairement indépendantes et dans ce cas, l'équation non homogène $u-Tu=f$ est résoluble si et seulement si $f$ vérifie $n$ conditions d'orthogonalité (i.e. $f\in N(I-T^*)^\perp$).
\end{itemize}
	\item La propriété (3) est familière en dimension finie. Si dim $E<\infty$, un opérateur linéaire de $E$ dans lui-même est injectif si et seulement s'il est surjectif.
\end{enumerate}}

\subsection{Spectre d'un opérateur - Décomposition spéctrale}
\subsubsection{Spectre d'un opérateur compact}
\Def{Ensemble résolvant, spectre, espace propre}{Soit $T\in\mathscr{L}(E)$\\
L'ensemble résolvant est \[\rho(T)=\{\lambda\in\mathbb{R}; (T-\lambda I) \text{ est bijectif de } E \text{ dans } E\}\]
Le spectre $\sigma(T)$ est le complémentaire de l'ensemble résolvant \[\sigma(T)=\mathbb{R}\backslash \rho(T)\]
On dit que $\lambda$ est valeur propre - et on note $\lambda\in VP(T)$ - si \[N(T-\lambda I)\neq \{0\}\]
$N(T-\lambda I)$ est l'espace propre associé à $\lambda$.}

\textbf{Remarque :} Il est clair que $VP(T)\subset \sigma(T)$. En général, l'inclusion est stricte (sauf bien sûr en dimension finie). Il peut exister $\lambda$ tel que 
\[N(T-\lambda I)=\{0\} \text{ et } R(T-\lambda I)\neq E\]
(un tel $\lambda$ appartient au spectre mais n'est pas valeur propre).\\
Par exemple, prenons dans $E=l^2$, $Tu=(0,u_1,u_2,...)$ où $u=(u_1,u_2,...)$ (T est appelé le shift à droite). Alors $0\in\sigma(T)$ et $0\not\in VP(T)$.

\Propo{}{Le spectre $\sigma(T)$ est un ensemble compact et \[\sigma(T)\subset [-\|T\|, +\|T\|]\]}

\begin{dem}
Soit $\lambda\in\mathbb{R}$ avec $|\lambda|>\|T\|$. Montrons que $T-\lambda I$ est bijectif - ce qui prouvera $\sigma(T)\subset [-\|T\|, +\|T\|]$.\\
Étant donné $f\in E$, l'équation $Tu-\lambda u=f$ admet une solution unique car elle s'écrit $u=\frac{1}{\lambda}(Tu-f)$ et on peut lui appliquer le théorème du point fixe de Banach (en effet, il est simple de vérifier que l'application $u\mapsto\frac{1}{\lambda}(Tu-f)$ définit une contraction : il suffit de majorer par $\frac{\|T\|}{\lambda}$).\\
Montrons maintenant que $\rho(T)$ est ouvert - ainsi $\sigma(T)$ sera par complémentaire fermé, et donc compact. Soit $\lambda_0\in\rho(T)$. Étant donnés $\lambda\in\mathbb{R}$ (voisin de $\lambda_0$) et $f\in E$, on cherche à résoudre :
	\begin{equation} \label{VI7.5} Tu-\lambda u=f \end{equation}
Or, on peut réécrire (\ref{VI7.5}) $Tu-\lambda_0 u=f+(\lambda-\lambda_0)u$, ie : 
	\begin{equation} \label{VI7.6} u=(T-\lambda_0 I)^{-1}[f+(\lambda-\lambda_0)u] \end{equation}
En appliquant à nouveau le théorème du point fixe de Banach, on voit que (\ref{VI7.6}) possède une solution unique si \[|\lambda-\lambda_0|\ \|(T-\lambda_0 I)^{-1}\|<1\]
On définit donc une boule ouverte autour de $\lambda_0$ incluse dans $\rho(T)$. Donc $\rho(T)$ est un ouvert.
\end{dem}

\Theo{}{Soit $T\in\mathscr{H}(E)$ avec dim $E=+\infty$. Alors on a :
\begin{enumerate}
	\item $0\in\sigma(T)$
	\item $\sigma(T)\backslash\{0\}=VP(T)\backslash\{0\}$
	\item l'une des situations suivantes : 
	\begin{itemize}
		\item ou bien $\sigma(T)=0$
		\item ou bien $\sigma(T)\backslash\{0\}$ est fini
		\item ou bien $\sigma(T)\backslash\{0\}$ est une suite qui tend vers $0$
	\end{itemize}
\end{enumerate}}

\begin{dem}
\begin{enumerate}
	\item Supposons que $0\not\in\sigma(T)$. Alors $T$ est bijectif et $I=T\circ T^{-1}$ est compact. Donc $B_E$ est compact et par force, dim $E<\infty$ par le théorème de Riesz précédent.

	\item Soit $\lambda\in\sigma(T)$, $\lambda\neq 0$. Montrons que $\lambda\in VP(T)$. Raisonnons par l'absurde et supposons que $N(T-\lambda I)=\{0\}$. Alors d'après l'alternative de Fredholm, on sait que $R(T-\lambda I)=E$, et donc $\lambda\in\rho(T)$ - ce qui est absurde.

	\item On va avoir besoin du lemme suivant :
\Lem{}{Soit $(\lambda_n)_{n\geq 1}\subset\sigma(T)\backslash\{0\}$ une suite de réels tous distincts telle que \[\lambda_n\to\lambda\]
Alors $\lambda=0$.}

On sait que $\lambda_n\in VP(T)$; soit $e_n\neq 0$ tel que $(T-\lambda_n)e_n=0$. Soit $E_n=vect\{e_1,...,e_n\}$. Montrons que $\forall n\ E_n\subsetneq E_{n+1}$.\\
Il suffit de vérifier que, pour tout $n$, les vecteurs $e_1,...,e_n$ sont linéairement indépendants. Raisonnons par récurrence sur $n$. Admettons le résultat à l'ordre $n$ et supposons que $e_{n+1}=\sum_{i=1}^n \alpha_i e_i$. Alors :
	\[Te_{n+1}=\sum_{i=1}^n \alpha_i\lambda_ie_i=\sum_{i=1}^n \alpha_i\lambda_{n+1}e_i\]
Par suite, $\alpha_i(\lambda_i-\lambda_{n+1})=0$ pour tout $i=1,2,...,n$, et donc $\alpha_i=0$ pour tout $i=1,...,n$ - ce qui est absurde. Donc $E_n\subsetneq E_{n+1}$ pour tout $n$.\\
D'autre part, il est clair que $(T-\lambda_n)E_n\subset E_{n-1}$. En appliquant le lemme de Riesez, on construit une suite $(u_n)_{n\geq 1}$ telle que $u_n\in E_n$, $\|u_n\|=1$ et $d(u_n,E_{n-1})\geq \frac{1}{2}$ pour $n\geq 2$. Soient $2\leq m<n$ de sorte que \[E_{m-1}\subset E_m\subset E_{n-1}\subset E_n\]
On a :
\[\left\| \frac{Tu_n}{\lambda_n}-\frac{Tu_m}{\lambda_m}\right\|=\left\| \frac{Tu_n-\lambda_nu_n}{\lambda_n}-\frac{Tu_m-\lambda_mu_m}{\lambda_m}+u_n-u_m\right\|\geq d(u_n,E_{n-1})\geq \frac{1}{2}\]
Si $\lambda_n\to\lambda$, on aboutit à une contradiction, puisque $(Tu_n)$ admet une sous-suite convergente.

\bigskip
\textit{Retour à la démonstration du théorème :} Pour tout entier $n\geq 1$, l'ensemble \[\sigma(T)\cap\{\lambda\in\mathbb{R}, |\lambda|\geq\frac{1}{n}\}\]
est vide ou fini (s'il contenait une infinité de points distincts, on aurait un point d'accumulation - puisque $\sigma(T)$ est compact - et on aboutirai à une contradiction avec le lemme démontré précédemment). Lorsque $\sigma(T)\backslash\{0\}$ contient une infinité de points distincts, on peut donc les ranger en une suite qui tend vers $0$.
\end{enumerate}
\end{dem}

\subsubsection{Décomposition spectrale des opérateurs autoadjoints}
n suppose dans la suite que $E=H$ est un espace de Hilbert et que $T\in\mathscr{L}(H)$. En identifiant $H$ et $H'$ (grâce au théorème de représentation de Riesz), on peut considérer que $T^*\in\mathscr{L}(H)$.

\Def{Autoadjoint}{On dit qu'un opérateur $T\in\mathscr{L}(H)$ est autoadjoint su $T^*=T$, ie \[(Tu,v)=(u,Tv)\ \forall u,v\in H\]}

\Propo{}{Soit $T\in\mathscr{L}(H)$ un opérateur autoadjoint. On pose :
	\[m=\inf_{u\in H,\ |u|=1}(Tu,u) \text{ et } M=\sup_{u\in H,\ |u|=1}(Tu,u)\]
Alors $\sigma(T)\subset[m,M]$, avec $m\in\sigma(T)$ et $M\in\sigma(T)$.}

\begin{dem}
Soit $\lambda>M$; montrons que $\lambda\in\rho(T)$. On a : \[(Tu,u)\leq M|u|^2\ \forall u\in H\]
et par conséquent \[(\lambda u-Tu,u)\geq (\lambda-M)|u|^2=\alpha|u|^2\ \forall u\in H \text{ avec } \alpha>0\]
Appliquant le théorème de Lax-Milgram, on voit que $\lambda I-T$ est bijectif.\\
Montrons que $M\in\sigma(T)$. La forme $a(u,v)=(Mu-Tu,v)$ est bilinéaire, symétrique et \[a(v,v)\geq 0\ \forall v\in H\]
Appliquant l'inégalité de Cauchy-Schwarz à la forme $a(u,v)$, il vient : \[|(Mu-Tu,v)|\leq (Mu-Tu,u)^{\frac{1}{2}}(Mv-Tv,v)^\frac{1}{2}\ \forall u,v\in H\] (Il faut m'expliquer où est CS là...?)\\
D'où il résulte en particulier \begin{equation}\label{VI.9.7} |Mu-Tu|\leq C(Mu-Tu,u)^\frac{1}{2}\ \forall u\in H \end{equation}
Soit $(u_n)$ une suite telle que $|u_n|=1$ et $(Tu_n,u_n)\to M$. Grâce à (\ref{VI.9.7}), on voit que $|Mu_n-Tu_n|\to 0$ et donc $M\in\sigma(T)$ (car si $M\in\rho(T)$, alors $u_n=(MI-T)^{-1}(Mu_n-Tu_n)\to 0$)\\
Les propriétés de $m$ s'obtiennent en remplaçant $T$ par $-T$
\end{dem}

\Coro{}{Soit $T\in\mathscr{L}(H)$ un opérateur autoadjoint tel que $\sigma(T)=\{0\}$. Alors $T=0$.}
\begin{dem}
	D'après la proposition précédente, on sait que \[(Tu,u)=0\ \forall u\in H\]
Il en résulte que :
	\[2(Tu,v)=(T(u+v),u+v)-(Tu,u)-(Tv,v)=0 \forall u,v\in H\]
Donc $T=0$
\end{dem}

\Theo{Diagonalisation}{On suppose que $H$ est séparable. Soit $T$ un opérateur autoadjoint compact.\\
Alors $H$ admet une base hilbertienne formée de vecteurs propres de $T$.}

\begin{dem}
Soit $(\lambda_n)_{n\geq 1}$ la suite des valeurs propres distincte de $T$, excepté $0$; on note $\lambda_0=0$.\\
On pose $E_0=N(T)$ et $E_n=N(T-\lambda_n I)$; rappelons que \[0\leq dim E_0\leq \infty \text{ et que } 0<dim E_n<\infty\]
Montrons d'abord que $H$ est comme hilbertienne des $(E_n)_{n\geq 0}$ :
\begin{enumerate}
	\item Les $(E_n)_{n\geq 0}$ sont deux à deux orthogonaux. En effet, si $u\in E_{m}$ et $v\in E_n$ avec $m\neq n$ alors 
	\[Tu=\lambda_mu \text{ et } Tv=\lambda_n v\]
et \[(Tu,v)=\lambda_m(u,v)=(u,Tv)=\lambda_n(u,v)\]
Donc $(u,v)=0$

	\item Soit $F$ l'espace vectoriel engendré par les $(E_n)_{n\geq 0}$. Vérifions que $F$ est dense dans $H$.\\
Il est claire que $T(F)\subset F$. Il s'en suit que $T(F^\perp)\subset F^\perp$; en effet, si $u\in F^\perp$ et $v\in F$, alors $(Tu,v)=(u,Tv)=0$. L'opérateur $T_0=\restriction{T}{F}$ est autoadjoint compact. D'autre part, $\sigma(T_0)=\{0\}$; en effet, si
	\[\lambda\in\sigma(T_0)\backslash\{0\}\text{, alors } \lambda\in VP(T_0)\]
et donc il existe $u\in F^\perp$, $u\neq 0$ tel que $T_0u=\lambda u$. Par conséquent, $\lambda$ est l'une des valeurs propres $\lambda_n$ de $T$ et $u\in F^\perp\cap E_n$. Donc $u=0$, ce qui est absurde.\\
Il résulte du corollaire précédent que $T_0=0$; par suite
	\[F^\perp\subset N(T)\subset F \text{ et } F^\perp=\{0\}\]
Donc $F$ est dense dans $H$.\\
Enfin, on choisit dans chaque $E_n$ une base hilbertienne. La réunion de ces bases est une base hilbertienne de $H$ formée de vecteurs propres de $T$.
\end{enumerate}
\end{dem}

\section{Théorème de Hille-Yosida}
\subsection{Opérateurs maximaux monotones}
Dans toute la suite, $H$ désigne un espace de Hilbert.

\Def{Maximal et monotone}{Soit $A:D(A)\subset H\to H$ un opérateur linéaire non-borné. On dit que $A$ est monotone (ou accrétif ou dissipatif) si \[(Av,v)\geq 0\ \forall v\in D(A)\]
$A$ est maximal monotone si de plus, $R(I+A)=H$, ie 
	\[\forall f\in H, \exists u\in D(A); u+Au=f\]}

\Propo{}{Soit $A$ un opérateur maximal monotone. Alors : \begin{enumerate}
	\item $D(A)$ est dense dans $H$
	\item $A$ est fermé
	\item Pour tout $\lambda>0$, $(I+\lambda A)$ est bijectif de $D(A)$ sur $H$, et $(I+\lambda A)^{-1}$ est un opérateur borné de norme $\|(I+\lambda A)^{-1}\|_{\mathscr{L}(H)}\leq 1$
\end{enumerate}}

\begin{dem} \begin{enumerate}
	\item Soit $f\in H$ tel que $(f,v)=0$ pour tout $v\in D(A)$. Vérifions que $f=0$. En effet, il existe $v_0\in D(A)$ tel que $v_0+Av_0=f$. On a :
	\[0=(f,v_0)=|v_0|^2+(Av_0,v_0)\geq |v_0|^2\]
Donc $v_0=0$ et par suite $f=0$.

	\item Notons d'abord que pour tout $f\in H$ il existe $u\in D(A)$ unique tel que $u+Au=f$. En effet, si $\tilde{U}$ désigne une autre solution, alors on a $(u-\tilde{u})+A(u-\tilde{u})=0$. Prenant le produit scalaire avec $(u-\tilde{u})$ et appliquant la monotonie de $A$, on voit que $u-\tilde{u}=0$.\\
D'autre part, on a $|u|^2+(Au,u)=(f,u)$ et par suite, $|u|\leq |f|$. L'opérateur $f\mapsto u$ noté $(I+A)^{-1}$ est donc un opérateur linéaire borné de $H$ dans $H$ et $\|(I+A)^{-1}\|_{\mathscr{L}(H)}\leq 1$.\\
Montrons que $A$ est fermé. Soit $(u_n)\subset D(A)$ une suite telle que $u_n\to u$ et $Au_n\to f$. Il faut vérifier que $u\in D(A)$ et que $Au=f$. On a $u_n+Au_n\to u+f$ et donc
	\[u_n=(I+A)^{-1}(u_n+Au_n)\to(I+A)^{-1}(u+f)\]
Par conséquent, $u=(I+1)^{-1}(u+f)$, ie $u\in D(A)$ et $u+Au=f$.

	\item Supposons que pour un certain $\lambda_0>0$, on ait $R(I+\lambda_0 A)=H$. On va montrer que pour tout $\lambda>\frac{\lambda_0}{2}$, on a $R(I+\lambda A)=H$. \\
Commençons par notrer que pour tout $f\in H$ il existe $u\in D(A)$ unique tel que $u+\lambda_0 Au=f$ ; l'opérateur $f\mapsto u$ est noté $(I+\lambda_0A)^{-1}$ et l'on a $\|(I+\lambda_0A)^{-1}\|_{\mathscr{L}(H)}\leq 1$. On cherche à résoudre l'équation
\[u+\lambda Au=f \text{ avec } \lambda >0\]
On réécrit l'équation sous la forme
	\[u+\lambda_0Au=\frac{\lambda_0}{\lambda}f+\left(1-\frac{\lambda_0}{\lambda}\right)u\]
Ou encore :
\begin{equation} \label{VII1.2} u=(I+\lambda_0 A)^{-1}\left[\frac{\lambda_0}{\lambda}f+\left( 1-\frac{\lambda_0}{\lambda}\right)u \right] \end{equation}

On voit alors que si $\left| 1-\frac{\lambda_0}{\lambda}\right|<1$, ie $\lambda>\frac{\lambda_0}{2}$, alors (\ref{VII1.2}) admet une solution grâce au théorème du point fixe de Banach.\\
Si $A$ est maximal monotone, alors $I+A$ est surjectif. D'après ce qui précède, $I+\lambda A$ est surjectif pour $\lambda>\frac{1}{2}$ donc aussi pour $\lambda>\frac{1}{4}$, etc. Par récurrence on voit que $I+\lambda A$ est surjectif pour tout $\lambda>0$.
\end{enumerate}\end{dem}

\Def{Résolvante et résgularisée}{Soit $A$ un opérateur maximal monotone. On pose, pour tout $\lambda>0$, 
	\[J_\lambda=(I+\lambda A)^{-1} \text{ et } A_\lambda=\frac{1}{\lambda}(I-J_\lambda)\]
$J_\lambda$ est la résolvante de $A$ et $A_\lambda$ est la régularisée Yosida de $A$.}

On retiendra que $\|J_\lambda\|_{\mathscr{L}(H)}\leq 1$.

\Propo{}{Soit $A$ un opérateur maximal monotone. On a :\begin{enumerate}
	\item $A_\lambda v=A(J_\lambda v)$ $\forall v\in H$ et $\forall \lambda>0$
	\item $A_\lambda v=A(J_\lambda v)$ $\forall v\in D(A)$ et $\forall \lambda>0$
	\item $|A_\lambda v|\leq |Av|$ $\forall v\in D(A)$ et $\forall \lambda>0$
	\item $\lim_{\lambda\to 0} J_\lambda v=v$ $\forall v\in H $
	\item $\lim_{\lambda\to 0} J_\lambda v=v$ $\forall v\in D(A) $
	\item $(A_\lambda v,v)\geq 0$ $\forall v\in H$, $\forall \lambda>0$
	\item $|A_\lambda v|\leq \frac{1}{\lambda}|v|$ $\forall v\in H$, $\forall \lambda>0$
\end{enumerate}}

\begin{dem}\begin{enumerate}
	\item Équivaut à $v=(J_\lambda v)+\lambda A(J_\lambda v)$, qui résulte de la définition de $J_\lambda$
	\item On a : \[Av=\frac{1}{\lambda} [(I+\lambda A)v-v]=\frac{1}{\lambda}(I+\lambda A)(v-J_\lambda v)\]
et donc \[J_\lambda Av=\frac{1}{\lambda}(v-J_\lambda v)\]
	\item Résulte de 2)
	\item Supposons d'abord que $v\in D(A)$. Alors \[|v-J_\lambda v|=\lambda|A_\lambda v|\leq \lambda|Av|\]
Donc $\lim_{\lambda\to 0} J_\lambda v=v$.

\bigskip
Passons au cas général. Soit $v\in H$ et soit $\varepsilon>0$. Comme $\overline{D(A)}=H$, il existe $v_1\in D(A)$ tel que $|v-v_1|\leq\varepsilon$. On a : 
\begin{eqnarray*}
|J_\lambda v-v|&\leq& |J_\lambda v-J_\lambda v_1|+|J_\lambda v_1-v_1|+|v_1-v|\\
		&\leq& 2|v-v_1|+|J_\lambda v_1-v_1|\\
		&\leq& 2\varepsilon + |J_\lambda v_1-v_1|
\end{eqnarray*}

Par conséquent \[\limsup_{\lambda\to 0} |J_\lambda v-v|\leq 2\varepsilon\ \forall\varepsilon>0\]
et donc \[\lim_{\lambda\to 0} |J_\lambda v-v|=0\]

	\item Appliquer 2. et 4.
	\item On a \begin{eqnarray*}
(A_\lambda v,v)&=&(A_\lambda v,v-J_\lambda v)+(A_\lambda v,J_\lambda v)\\
		&=&\lambda|A_\lambda v|^2+(A_\lambda(J_\lambda v),J_\lambda v)
\end{eqnarray*}
Donc \[(A_\lambda v,v)\geq \lambda|A_\lambda v|^2\geq 0\]

	\item Viens de la dernière inégalité et de Cauchy-Schwarz.
\end{enumerate}\end{dem}

\subsection{Problème d'évolution}
\subsubsection{Existence et unicité}
On s'intéresse au problème général suivant :
\begin{equation} \label{PbEv} \tag{PbEv} \left\{ \begin{array}{c c c c}
	\frac{du}{dt}+Au &=& 0 &\text{ sur } [0,+\infty[\\
	u(0)&=&u_0
\end{array}\right. \end{equation}

On rappelle le résultat classique suivant :
\Theo{Cauchy-Lipschitz-Picard}{Soit $E$ un espace de Banach et soit $F:E\to E$ une application telle que \[\|Fu-Fv\|\leq L\|u-v\|\ \forall u,v\in E\ (L\geq 0)\]
Alors pour tout $u_0\in E$, il existe $u\in \mathscr{C}^1([0,\infty[;E)$ unique telle que
\begin{equation} \label{CLP} \tag{CLP} \left\{ \begin{array}{c c c c}
	\frac{du}{dt}&=& Fu &\text{ sur } [0,+\infty[\\
	u(0)&=&u_0
\end{array}\right. \end{equation}}

%\Lem{1}{Soit $w\in\mathscr{C}^1([0,+\infty[);H)$ une fonction vérifiant : \begin{equation} \label{lem1} \frac{dw}{dt} + A_\lambda w=0 \text{ sur } [0,+\infty[ \end{equation}
%Alors les fonctions $t\mapsto |w(t)|$ et $t\mapsto \left|\frac{dw}{dt}(t)\right|=|A_\lambda w(t)|$ sont décroissantes sur $[0,+\infty[$.}
%
%\begin{dem}
%	On a \[\left(\frac{dw}{dt},w\right) + (A_\lambda w,w)=0\]
%Or, comme on l'a vu dans une propriété précédente (prop. 6), $(A_\lambda w, w)\geq 0$ et par suite $\frac{1}{2} \frac{d}{dt}|w|^2\leq 0$.\\
%D'autre part, comme $A_\lambda$ est un opérateur linéaire borné, on déduit de (\ref{lem1}) que $w\in\mathscr{C}^\infty$ et que \[\frac{d}{dt}\left(\frac{dw}{dt}\right)+A_\lambda\left(\frac{dw}{dt}\right)=0\]
%On applique alors ce qui précède à $\frac{dw}{dt}$.
%\end{dem}

\Theo{Hille-Yosida}{Soit $A$ un opérateur maximal monotone dans un espace de Hilbert $H$. Alors pour tout $u_0\in D(A)$, il existe une fonction \[u\in\mathscr{C}^1([0,+\infty[;H)\cap\mathscr{C}([0,+\infty[;D(A))\]
unique vérifiant (\ref{PbEv}). De plus, on a \[|u(t)\leq |u_0| \text{ et } \left| \frac{du}{dt}(t)\right|=|Au(t)|\leq |Au_0|\ \forall t\geq 0\]}

\Rem{}{\begin{enumerate}
	\item Soit $t\geq 0$; on considère l'application linéaire $S_A(t):u_0\mapsto u(t)$ de $D(A)$ dans $D(A)$ où $u(t)$ est la solution de (\ref{PbEv}). Étant donné que $|S_A(t)u_0|\leq|u_0|$, on peut prolonger $S_A(t)$ par continuité et densité en un opérateur linéaire continue de $H$ dans lui-même, qu'on désigne toujours par $S_A(t)$. On vérifie facilement que $S_A(t)$ possède les propriétés suivantes :
\begin{enumerate}
	\item Pour chaque $t\geq 0$, $S_A(t):H\to H$ est un opéarateur linéaire continue et $S_A(t)_{\mathscr{L}(H)}\leq 1$
	\item $S_A(t_1+t_2)=S_A(t_1)\circ S_A(t_2)$ $\forall t_1,t_2\geq 0$ et $S_A(0)=Id$
	\item $\lim_{t\to 0^+} |S_A(t)u_0-u_0|=0$ $\forall u_0\in H$
\end{enumerate}
Une famille $\{S(T)\}_{t\geq 0}$ d'opérateurs de $\mathscr{L}(H)$ définie pour chaque valeur du paramètre $t\geq 0$ et vérifiant ces trois points est par définition un semi-groupe continu de contractions.\\
On montre qu'inversement, étant donné un semi-groupe continu de contractions $S(t)$, il existe un opérateur $A$ maximal monotone unique tel que $S(T)=S_A(t)$ pour tout $t\geq 0$.\\
On étabilit ainsi une correspondance bijective entre les opérateurs maximaux monotones et les semi-groupes continus de contraction.

	\item Soit $A$ un opérateur maximal monotone et soit $\lambda\in\mathbb{R}$. La résolution de l'équation \[\left\{\begin{array}{c c c c} \frac{du}{dt}+Au+\lambda u&=&0 &\text{ sur } [0,+\infty[ \\ u(0)=u_0 \end{array}\right.\]
se ramène très simplement à la résolution de (\ref{PbEv}) grâce à l'artifice classique suivant. On pose \[v(t)=e^{\lambda t} u(t)\]
Alors $v$ vérifie 
\begin{equation} \left\{ \begin{array}{c c c c}
	\frac{dv}{dt}+Av&=&0 &\text{ sur } [0,+\infty[\\
	v(0)&=&u_0
\end{array}\right. \end{equation}
\end{enumerate}}

\subsubsection{Régularité}
\Def{}{On définition par récurrence l'espace \[D(A^k)=\{v\in D(A^{k-1});\ Av\in D(A^{k-1})\},\ k \text{ entier }\geq 2\]
On vérifie aisément que $D(A^k)$ est un espace de Hilbert pour le produit scalaire
	\[(u,v)_{D(A^k)}=\sum_{j=0}^k (A^ju,A^jv)\]
}

\Theo{}{On suppose que $u_0\in D(A^k)$ avec $k\geq 2$. Alors la solution $u$ du problème (\ref{PbEv}) vérifie de plus : \[u\in\mathscr{C}^{k-j}([0,+\infty[;D(A^j))\ \text{ pour } j=0,1,...,k\]}

\subsubsection{Dans les espaces de Banach}
Soit $E$ un espace de Banach.
\Def{m-accrétif}{Soit $A:D(A)\subset E\to E$ un opérateur linéaire non-borné. On dit que $A$ est m-accrétif si $\overline{D(A)}=E$ et si pour tout $\lambda>0$, $I+\lambda A$ est bijectif de $D(A)$ sur $E$, avec $\|(I+\lambda A)^{-1}\|_{\mathscr{L}(E)}\leq 1$}

\Theo{Hille-Yosida dans les espaces de Banach}{Soit $A$ un opérateur m-accrétif dans $E$. Alors pour tout $u_0\in D(A)$, il existe une fonction \[u\in\mathscr{C}^1([0,+\infty[;H)\cap\mathscr{C}([0,+\infty[;D(A))\]
unique vérifiant (\ref{PbEv}). De plus, on a \[|u(t)\leq |u_0| \text{ et } \left| \frac{du}{dt}(t)\right|=|Au(t)|\leq |Au_0|\ \forall t\geq 0\]}

\Theo{}{On suppose que $A$ est m-accrétif. Alors pour tout $u_0\in D(A)$, la solution $u$ de (\ref{PbEv}) est donnée par la formule exponentielle \[u(t)=\lim_{n\to+\infty} \left[\left(I+\frac{t}{n} A\right)^{-1}\right]^n u_0\]}
