\part{Opérateurs}
\section{Définitions et résultats préliminaires}
\Theo{du graphe fermé}{Soient $E$ et $F$ deux espaces de Banach. Soit $T$ un opérateur linéaire de $E$ dans $F$. On suppose que le graphe de $T$ est fermé dans $E\times F$. Alors $T$ est continue.}

\Lem{de Baire}{Soit $X$ un espace métrique complet. Soit $(X_n)_{n\geq 1}$ une suite de fermés. On suppose que \[\forall n\geq 1,\ \widering{X_n}=\emptyset\]
Alors \[\widering{\bigcup_{i=1}^{\infty} X_i}=\emptyset\]}

\begin{dem}
	On pose $O_n=X_n^C$ le complémentaire de $X_n$, de sorte que $O_n$ est un ouvert dense. Il s'agit de montrer que $G=\cap_{i=1}^\infty O_i$ est dense dans $X$.\\
	Soit $\omega$ un ouvert non vide de $X$. On va prouver que $\omega\cap G\neq\emptyset$. \\
	On choisit $x_0\in\omega$ et $r_0>0$ arbitraires tels que \[\overline{B(x_0,r_0)}\subset\omega\]
	On choisit ensuite $x_1\in B(x_0,r_0)\cap O_1$ et $r_1>0$ tels que : 
	\[\left\{\begin{array}{c}
		\overline{B(x_1,r_1)}\subset B(x_0,r_0)\cap O_1\\
		0<r_1<\frac{r_0}{2}
	\end{array}\right.\]
	Ceci est possible car $O_1$ est ouvert et dense. Ainsi de sute, on construit par récurrence deux suites $(x_n)$ et $(r_n)$ telles que :	
	\[\left\{\begin{array}{c}
		\overline{B(x_{n+1},r_{n+1})}\subset B(x_n,r_n)\cap O_{n+1}\\
		0<r_{n+1}<\frac{r_n}{2}
	\end{array}\right.\]
	Il en résulte que la suite $(x_n)$ est de Cauchy. Soit $x_n\to l$. Comme $x_{n+p}\in B(x_n,r_n)$ pour tous $n, p\geq 0$, on obtient à la limite (quand $p\to +\infty$) :
	\[l\in \overline{B(x_n,r_n)}\ \forall n\geq 0\]
	En particulier, $l\in \omega\cap G$.
\end{dem}

\Def{Orthogonal d'un ev}{Soit X un espace de Banach.\\
Si $M\subset X$ est un sev, on pose \[M^{\perp}=f\in X'; \langle f,x\rangle =0 \forall x\in M\}\]
Si $N\subset X'$ est un sev, on pose \[N^{\perp}=x\in X; \langle f,x\rangle =0 \forall f\in N\}\]
$M^{\perp}$ (resp. $N^{\perp}$) est l'orthogonal de $M$ (resp. $N$), qui est un sev fermé de $X'$ (resp. $X$).}

\Propo{}{Soit $M\subset X$ un sev. On a alors \begin{equation}\label{orthoTwice1}\left( M^{\perp}\right)^{\perp}=\overline{M}\end{equation}
Soit $N\subset X'$ un sev. On a alors \begin{equation}\label{orthoTwice2}\left( N^{\perp}\right)^{\perp}\supset \overline{N}\end{equation}}

\Propo{}{Soient $G$ et $L$ deux sous-espaces fermés de X. On a : 
	\begin{eqnarray} \label{Prop1}
		G\cap L=\left(G^{\perp}+L^{\perp}\right)^{\perp}\\
		G^\perp\cap L^\perp=\left(G+L\right)^{\perp}
	\end{eqnarray}
}

\Theo{}{Soient $G$ et $L$ deux sous-espaces fermés de $X$. Les propriétés suivantes sont équivalentes :
\begin{eqnarray}\label{Theo1.1}
G+L \text{ est fermé dans X}\\
G^\perp+L^\perp \text{ est fermé dans X}\\
G+L=\left(G^\perp+L^\perp\right)^\perp\\
\label{Theo1.2} G^\perp+L^\perp=(G\cap L)^\perp
\end{eqnarray}}

\section{Opérateurs non-bornés}
\subsection{Définitions et propositions}

\Def{Opérateur}{Soient $E$ et $F$ deux espaces de banach. On appelle opérateur linéaire non borné de $E$ dans $F$ toute application linéaire \[A:D(A)\subset E\to F\] définie sur un sous-espace vectoriel $D(A)\subset E$ à valeur dans $F$. $D(A)$ est le domaine de $A$.\\
On dit que $A$ est borné s'il existe une constante $c\geq 0$ telle que \[\|Au\|\leq c\|u\|\ \forall u\in D(A)\]
(Oui, avec cette définition, un opérateur non borné peut être... Borné)}

\Def{Graphe, Image et Noyau}{On appelle Graphe de $A$ l'ensemble \[G(A)=\bigcup_{u\in D(A)} [u,Au]\subset E\times F\]
On appelle Image de $A$ l'ensemble \[R(A)=\bigcup_{u\in D(A)} Au\subset F\]
On appelle Noyau de $A$ l'ensemble \[N(A)=\{u\in D(A);\ Au=0\}\subset E\]}

\Def{fermé}{On dit qu'un opérateur $A$ est fermé si $G(A)$ est fermé dans $E\times F$.}

\Rem{}{\begin{enumerate}
	\item Pour prouver qu'un opérateur $A$ est femré, on procède en général de la manière suivante : on prend une suite $(u_n)$ dans $D(A)$ telle que $u_n\to u$ dans $E$ et $Au_n\to f$ dans $F$. Il s'agit ensuite de vérifier que 
		\begin{enumerate}
			\item $u\in D(A)$
			\item $f=Au$
		\end{enumerate}
	\item Si $A$ est fermé, alors $N(A)$ est fermé.
\end{enumerate}}

\Def{Adjoint}{Soit $A:D(A)\subset E\to F$ un opérateur linéaire à domaine dense.\\ 
L'opérateur $A^*:D(A^*)\subset F'\to E'$, appelé adjoint de $A$, est l'unique opérateur vérifiant : \[\langle v, Au\rangle_{F'F}=\langle A^*v,u\rangle_{E'E}\hspace{2em} \forall u\in D(A),\ v\in D(A^*)\]}

L'existence et l'unicité de cet opérateur vient principalement du théorème de Hahn-Banach dans sa forme analytique. On pose : \[D(A^*)=\{v\in F';\ \exists c\geq 0 ; |\langle v,Au\rangle|\leq c\|u\|\ \forall u\in D(A)\}\]
Il est clair que $D(A^*)$ est un sous-espace vectoriel de $F'$. On va maintenant définir $A^*v$ pour $v\in D(A^*)$. On considère l'application $g:D(A)\to\mathbb{R}$ définie pour $v\in D(A^*)$ par \[g(u)=\langle v,Au\rangle_{F'F}\]
On a \[|g(u)|\leq c\|u\| \forall u\in E\]
On peut alors appliquer le théorème de Hahn-Banach : on sait que $g$ peut être prolongée en une application linéaire $f:E\to \mathbb{R}$ telle que \[|f(u)|\leq c\|u\|\ \forall u\in E\]
Par suite, $f\in E'$. On remarquera que le prolongement de $g$ est unique puisque $f$ est continue sur $E$ et que $D(A)$ est dense. On pose enfin : \[A^*v=f\]

\Propo{}{Soit $A:D(A)\subset E\to F$ un opérateur non borné à domaine dense. Alors $A^*$ est fermé.}

\begin{dem}
Soit $(v_n)\subset D(A^*)$ telle que $v_n\to v$ dans $F'$ et $A^*v_n\to f$ dans $E'$. Il s'agit de prouver que $v\in D(A^*)$ et $A^*v=f$. Or : 
	\[\langle v_n, Au\rangle=\langle A^*v_n,u\rangle\ \forall u\in D(A)\]
D'où à la limite, il vient :
	\[\langle v, Au\rangle = \langle A^*v, u\rangle\]
Par conséquent, $v\in D(A^*)$ par définition du domaine et $A^*v=f$.
\end{dem}

\Coro{}{Soit $A:D(A)\subset E\to F$ un opérateur non borné, fermé, avec $\overline{D(A)}=E$ (dense). Alors on a : 
\begin{enumerate}
	\item $N(A)=R(A^*)^\perp$
	\item $N(A^*)=R(A)^\perp$
	\item $N(A)^\perp \supset \overline{R(A^*)}$
	\item $N(A^*)^\perp = \overline{R(A)}$
\end{enumerate}}

\begin{dem}
On peut très facilement vérifier les égalités suivantes :
\begin{equation}\label{timesEgal1}N(A)\times\{0\}=G(A)\cap (E\times\{0\})\end{equation}
\begin{equation}\label{timesEgal2} E\times R(A)=G(A)+(E\times\{0\})\end{equation}
\begin{equation}\label{timesEgal3} \{0\}\times N(A^*)=G(A)^\perp\cap(E\times\{0\})^\perp\end{equation}
\begin{equation}\label{timesEgal4} R(A^*)\times F'=G(A)^\perp+(E\times\{0\})^\perp\end{equation}

En utilisant (\ref{Prop1}), on a donc directement :
\begin{eqnarray*} R(A^*)^\perp\times\{0\}&=&(R(A^*)\times F')^\perp\\
					&=&\left(G(A)^\perp+(E\times\{0\})^\perp\right)^\perp\\
					&=&G(A)\cap(E\times\{0\}\\
					&=&N(A)\times \{0\}
\end{eqnarray*}
D'où le premier résultat.\\
Pour le deuxième, on fait de même : 
\begin{eqnarray*}
	\{0\}\times R(A)^\perp&=&\left(G(A)+(E\times\{0\})\right)^\perp\\
				&=&G(A)^\perp\cap(E\times\{0\})^\perp\\
				&=&\{0\}\times N(A^*)
\end{eqnarray*}

Pour les deux derniers résultats, on utilise les deux premiers avec (\ref{orthoTwice1}) et (\ref{orthoTwice2}).
\end{dem}

\subsection{Opérateurs bornés}
\subsubsection{Opérateurs à image fermée}
\Theo{}{Soit $A:D(A)\subset E\to F$ un opérateur non-borné, fermé, avec le support de $A$ dense dans E. Les propriétés suivantes sont équivalentes :
\begin{enumerate}
	\item $R(A)$ est fermé
	\item $R(A^*)$ est fermé
	\item $R(A)=N(A^*)^\perp$
	\item $R(A^*)=N(A)^\perp$
\end{enumerate}}

\begin{dem}
$(1)\Leftrightarrow G(A)+(E\times\{0\})$ fermé dans $X$ (\ref{timesEgal2})\\
$(2)\Leftrightarrow G(A)^\perp+(E\times\{0\})^\perp$ fermé dans $X'$ (\ref{timesEgal4})\\
$(3)\Leftrightarrow G(A)+(E\times\{0\})=\left(G(A)^\perp\cap(E\times\{0\})^\perp\right)^\perp$ (\ref{timesEgal2}) et (\ref{timesEgal3})\\
$(4)\Leftrightarrow \left(G(A)\cap(E\times\{0\}\right)^\perp=G(A)^\perp+(E\times\{0\})^\perp$ (\ref{timesEgal1}) et (\ref{timesEgal4})\\

La conclusion nous vient directement du théorème (\ref{Theo1.1})-(\ref{Theo1.2}).
\end{dem}

\subsubsection{Opérateurs bornés}
\Theo{}{Soit $A:D(A)\subset E\to F$ un opérateur non-borné, fermé, avec son domaine dense dans $E$. Les propriétés suivantes sont équivalentes :
\begin{enumerate}
	\item $D(A)=E$
	\item $A$ est borné
	\item $D(A^*)=F'$
	\item $A^*$ est borné
\end{enumerate}
Dans ces conditions, on a : \[\|A\|_{\mathcal{L}(E,F)}=\|A^*\|_{\mathcal{L}(F',E')}\]}


\begin{dem}
$(1)\Rightarrow (2)$ : il suffit d'applquer le théorème du graphe fermé.\\
$(2)\Rightarrow (3)$ : par définition de $D(A^*)$ donnée après la définition de $A^*$\\
$(3)\Rightarrow (4)$ : On applique la proposition précédente sur une caractérisation de $A^*$ fermée et à l'aide du théorème du graphe fermé.\\
$(4)\Rightarrow (1)$ : Plus délicat. Notons d'abord que$D(A^*)$ est fermé. En effet, soit $(v_n)\subset D(A^*)$ avec $v_n\to v$ dans $F'$. On a : 
\[\|A^*(v_n-v_m)\|\leq c\|v_n-v_m\|\]
Par conséquent, $(A^*v_n)$ converge vers une limite $f$. Comme $A^*$ est fermé, $v\in D(A^*)$ et $A^*v=f$. Dans l'espace $X=E\times F$, on considère les sous-espaces $G=G(A)$ et $L=\{0\}\times F$ de sorte que \[G+L=D(A)\times F \text{ et } G^\perp+L^\perp=E'\times D(A^*)\]
Par conséquent, $G^\perp+L^\perp$ est fermé dans $X'$. Le théorème (\ref{Theo1.1})-(\ref{Theo1.2}) permet de conclure que $G+L$ est fermé, donc que $D(A)$ est fermé. Comme $\overline{D(A)}=E$, on en déduite que $D(A)=E$.

\bigskip
Prouvons maintenant que $\|A\|_{\mathcal{L}(E,F)}=\|A^*\|_{\mathcal{L}(F',E')}$. On a : 
	\[\langle v, Au\rangle=\langle A^*v, u\rangle\ \forall u\in E,\ \forall v\in F'\]
Donc \[|\langle v, Au\rangle| \leq \|A^*\|\|v\|\|u\|\] et 
	\[\|Au\|=\sup_{\|v\|\leq 1}|\langle v, Au\rangle| \leq \|A^*\|\|u\|\]
Par suite, $\|A\|\leq \|A^*\|$. Inversement, on a : 
\[\|A^* v\|=\sup_{\|u\|\leq 1}|\langle A^*v, u\rangle|=\sup_{\|u\|\leq 1}|\langle v, Au\rangle|\leq \|A\|\|v\|\]
Par conséquent, $\|A^*\|\leq \|A\|$.
\end{dem}

\section{Topologie faible}
Soit $E$ un espace de Banach, $E'$ son dual. Pour $f\in E'$, on définit $\phi_f:E\to\mathbb{R}$ tel que $\phi_f(x)=\langle f,x\rangle$. On définit ainsi une famille $(\phi_f)_{f\in E'}$ d'applications de $E$ dans $\mathbb{R}$. 
\Def{Topologie faible}{La topologie faible $\sigma(E,E')$ sur $E$ est la topologie la moins fine sur $E$ rendant continues toutes les applications $(\phi_f)_{f\in E'}$ continues, ie la topologie sur $E$ avec un nombre minimal d'ouvert rendant les $\phi_f$ continues.\\
On note par $\rightharpoonup$ la convergence pour la topologie faible.}

\Propo{}{Soit $(x_n)_n$ une suite de $E$. On a :
\begin{enumerate}
	\item $x_n\rightharpoonup x$ $\Leftrightarrow$ $\forall f\in E',\ \langle f,x_n\rangle \to \langle f,x\rangle$
	\item Si $x_n\to x$, alors $x_n\rightharpoonup x$
	\item Si $x_n\rightharpoonup x$ alors $\|x_n\|$ est bornée et $\|x\|\leq \liminf \|x_n\|$
	\item Si $x_n\rightharpoonup x$ et si $f_n\to f$ dans $E'$, alors $\langle f_n, x_n\rangle\to \langle f,x\rangle$.
\end{enumerate}}

\begin{dem}
\begin{enumerate}
\item Admis
\item Résulte de (1), puisque $|\langle f, x_n\rangle - \langle f,x\rangle|\leq \|f\|\|x_n-x\|$
\item On utilise pour cela le corollaire du théorème de Banach-Steinhaus suivant : \\
	\textbf{Corollaire :} Soit $G$ un espace de Banach et soit $B$ un sous-ensemble de $G$. On suppose que pour tout $f\in G'$, l'ensemble $f(B)=\bigcup_{x\in B}\langle f,x\rangle$ est borné. Alors $B$ est borné.
Il suffit donc de vérifier que pour chaque $f\in E'$, l'ensemble $(\langle f, x_n\rangle)_n$ est borné. Or, pour chaque $f\in E'$, la suite $\langle f,x_n\rangle$ converge vers $\langle f,x\rangle$ (en particulier, elle est bornée). Soit $f\in E'$, on a : \[|\langle f,x_n\rangle\leq \|f\|\|x_n\|\]
et à la limite : \[|\langle f,x\rangle\leq \|f\|\liminf \|x_n\|\]
Par conséquent : \[\|x\|=\sup_{\|f\|\leq 1} |\langle f,x\rangle|\leq \liminf \|x_n\|\]
\item On a :
	\[|\langle f_n,x_n\rangle-\langle f,x\rangle|\leq |\langle f_n-f,x_n\rangle|+|\langle f,x_n-x\rangle|\leq\|f_n-f\|\|x\|+|\langle f,x_n-x\rangle|\]
On conclut grâce à (1) et (3).
\end{enumerate}
\end{dem}

\Propo{}{Lorsque $E$ est de dimension finie, la topologie faible $\sigma(E,E')$ et la topologie usuelle conïncident. En particulier, une suite $(x_n)$ converge faiblement si et seulement si elle converge fortement.}
